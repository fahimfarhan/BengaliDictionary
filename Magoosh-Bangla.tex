\documentclass{article}
\usepackage{fontspec}



\usepackage{geometry}
 \geometry{
 a4paper,
 total={170mm,257mm},
 left=10mm,
 top=10mm,
 }


\setmainfont{[kalpurush.ttf]}

\begin{document} 

\title{Magoosh Bangla}
\maketitle



\section{COMMON WORDS}
{aberrant}\\
{অস্বাভাবিক}\\
{(adjective): markedly different from an accepted norm\\When the financial director started screaming and throwing food at his co-workers, the police had to\\come in to deal with his aberrant behavior.\\}\\
{aberration}\\
{সত্যপথ হতে স্খলন}\\
{(noun): a deviation from what is normal or expected\\Aberrations in climate have become the norm: rarely a week goes by without some meteorological\\phenomenon making headlines.\\}\\
{abstain}\\
{মদ্যপান না করা}\\
{(verb): choose not to consume or take part in (particularly something enjoyable)\\Considered a health nut, Jessica abstained from anything containing sugar--even chocolate.\\}\\
{abstruse}\\
{নিগূঢ়}\\
{(adjective): difficult to understand; incomprehensible\\Physics textbooks can seem so abstruse to the uninitiated that readers feel as though they are looking at\\hieroglyphics.\\}\\
{accolade}\\
{প্রশংসা,,}\\
{(noun): an award or praise granted as a special honor\\Jean Paul-Sartre was not a fan of accolades, and as such, he refused to accept the Nobel Prize for\\Literature in 1964.\\}\\
{acerbic}\\
{}\\
{(adjective): harsh in tone\\Most movie critics are acerbic towards summer blockbusters, often referring to them as garbage.\\}\\
{acrimony}\\
{"স্বভারের রুক্ষতা, মেজাজের রুক্ষতা, তীব্র বিদ্বেষভরে আক্রমণ করা",,}\\
{(noun): bitterness and ill will\\The acrimonious dispute between the president and vice-president sent a clear signal to voters: the\\health of the current administration was imperiled.\\}\\
{adamant}\\
{জেদী}\\
{(adjective): refusing to change one's mind\\Civil rights icon Rosa Parks will forever be remembered for adamantly refusing to give up her seat on a\\public bus--even after the bus driver insisted, she remained rooted in place.\\}\\
{admonish}\\
{সতর্ক করা}\\
{(verb): to warn strongly, even to the point of reprimanding\\Before the concert began, security personel admonished the crowd not to come up on stage during the\\performance.\\}\\
{admonitory}\\
{সতর্কতা সূচক}\\
{(adjective): serving to warn; expressing reproof or reproach especially as a corrective\\At the assembly, the high school vice-principal gave the students an admonitory speech, warning them of\\the many risks and dangers of prom night.\\}\\
{aesthete}\\
{কলাবিদ্যবিশারদ,,}\\
{(noun): one who professes great sensitivity to the beauty of art and nature\\A true aesthete, Marty would spend hours at the Guggenheim Museum, staring at the same Picasso.\\\\                                                                                \\}\\
{aesthetic}\\
{সৌন্দর্য বিজ্ঞান বিষয়ক}\\
{(adjective): concerned with the appreciation of beauty\\The director, not known for his aesthetic sensibilities, decided not to use costumes at all, and put on the\\play in everyday clothing.\\}\\
{aesthetic}\\
{সৌন্দর্য বিজ্ঞান বিষয়ক}\\
{(noun): a set of principles underlying and guiding the work of a particular artist or artistic\\movement.\\The artist operated according to a peculiar aesthetic, not considering any photograph to be worth\\publishing unless it contained a marine mammal.\\}\\
{amalgam}\\
{পারদ মিশ্রিত ধাতু}\\
{(noun): a mixture of multiple things\\The band's music was an amalgam of hip-hop, flamenco and jazz, blending the three styles with\\surprising results.\\}\\
{ambiguous}\\
{দ্ব্যর্থক}\\
{(adjective): open to more than one interpretation\\The coach told his team, "Move towards that side of the field"; because he did not point, his directions\\were ambiguous, and the team had no idea to which side he was referring.\\}\\
{ambivalent}\\
{বিরোধী,,}\\
{(adjective): mixed or conflicting emotions about something\\Sam was ambivalent about studying for the exam because doing so ate up a lot of his time, yet he was\\able to improve his analytical skills.\\}\\
{ameliorate}\\
{উন্নতি সাধন করা}\\
{(verb): make something bad better\\Three Cups of Tea tells the story of western man who hopes to ameliorate poverty and the lack of\\education in Afghanistan.\\}\\
{amenable}\\
{শাসনযোগ্য}\\
{(adjective): easily persuaded\\Even though she did not like the outdoors, Shirley was generally amenable and so her brother was able\\to persuade her to go camping.\\}\\
{amorphous}\\
{নিরাকার,,}\\
{(adjective): shapeless\\His study plan for the GRE was at best amorphous; he would do questions from random pages in any one\\of seven test prep books.\\}\\
{anomalous}\\
{ব্যতিক্রমী,,}\\
{(adjective): not normal\\According to those who do not believe in climate change, the extreme weather over the last five years is\\simply anomalous--average temps should return to average, they believe.\\}\\
{anomaly}\\
{ব্যত্যয়}\\
{(noun): something that is not normal, standard, or expected\\After finding an anomaly in the data, she knew that she would have to conduct her experiment again.\\}\\
{antipathy}\\
{বিদ্বেষ,,}\\
{(noun): an intense feeling of dislike or aversion\\Maria had an antipathy for tour groups, often bolting to the other side of the museum as soon as she\\saw a chaperone leading a group of wide-eyed tourists.\\\\                                                                                 \\}\\
{antithetical}\\
{বিরূদ্ধ,,}\\
{(adjective): sharply contrasted in character or purpose\\His deep emotional involvement with these ideas is, in fact, antithetical to the disattachment Buddhism\\preaches.\\}\\
{apathetic}\\
{বিমুখ,,}\\
{(adjective): marked by a lack of interest\\Mr. Thompson was so talented at teaching math that even normally apathetic students took interest.\\}\\
{apathy}\\
{অনীহা}\\
{(noun): an absence of emotion or enthusiasm\\Widespread apathy among voters led to a very small turnout on election day.\\}\\
{apocryphal}\\
{অপ্রামাণিক,,}\\
{(adjective): being of questionable authenticity\\The web is notorious for sandwiching apocryphal stories between actual news.\\}\\
{appease}\\
{শান্ত করা,,}\\
{(verb): pacify by acceding to the demands of\\Neville Chamberlain, the British prime minister during WWII, tried to appease Hitler and in doing so sent\\a clear message: you can walk all over us.\\}\\
{arbitrary}\\
{খামখেয়ালী}\\
{(adjective): based on a random, groundless decision\\One of the arbitrary decrees in place during the emperor's rule is that all citizens pay him weekly homage\\at his palace.\\This word has other definitions but this is the most important one for the GRE\\}\\
{arcane}\\
{গোপনীয়,,}\\
{(adjective): requiring secret or mysterious knowledge\\Most college fraternities are known for arcane rituals that those hoping to the join the fraterntiy must\\learn.\\}\\
{arduous}\\
{শ্রমসাধ্য,,}\\
{(adjective): demanding considerable mental effort and skill; testing powers of endurance\\In order to deal with the arduous cross-country journey, truck drivers often survive on a string of\\caffeinated drinks, staying awake for up to 30 hours at a time.\\}\\
{artful}\\
{ছলনাময়,,}\\
{(adjective): exhibiting artistic skill\\Picasso is generally considered the most artful member of the Cubist movement.\\}\\
{artful}\\
{ছলনাময়,,}\\
{(adjective): clever in a cunning way\\Bernie Madoff's artful Ponzi scheme stole billions of dollars from investors and is considered the largest\\financial fraud in U.S. history.\\}\\
{ascetic}\\
{তপস্বী,,}\\
{(adjective): practicing self-denial\\His ascetic life is the main reason he inspired so many followers, especially since he gave up wealth and\\power to live in poverty.\\}\\
{ascetic}\\
{তপস্বী,,}\\
{(noun): one who practices great self-denial\\Historically, ascetics like Ghandi are often considered wise men partially because of their restraint.\\\\                                                                                   \\}\\
{askance}\\
{বক্রভাবে,,}\\
{(adverb): with a look of suspicion or disapproval\\The old couple looked askance on the teenagers seated next to them, whispering to each other, "They've\\got rings through their noses and purple hair!"\\}\\
{audacious}\\
{দুঃসাহসী,,}\\
{(adjective): willing to be bold in social situations or to take risks\\As all of the other campers cowered in their tents, Bill, armed only with a flashlight, audaciously tracked\\down the bear that had raided their food.\\}\\
{audacity}\\
{স্পর্ধা,,}\\
{(noun): aggressive boldness in social situations\\She surprised her colleagues by having the audacity to publically criticize the findings of an distinguished\\scientist.\\}\\
{auspicious}\\
{মঙ্গলজনক}\\
{(adjective): favorable, the opposite of sinister\\Despite an auspicious beginning, Mike's road trip became a series of mishaps, and he was soon stranded\\and penniless, leaning against his wrecked automobile.\\}\\
{austere}\\
{উগ্র,,}\\
{(adjective): practicing self-denial\\His lifestyle of revelry and luxurious excess could hardly be called austere.\\}\\
{austere}\\
{উগ্র,,}\\
{(adjective): unadorned in style or appearance\\Late Soviet architecture, although remaining largely austere, moved into experimental territory that\\employed previously unused shapes and structures.\\}\\
{austere}\\
{উগ্র,,}\\
{(adjective): harsh in manner of temperament\\The principal of my elementary school was a cold, austere woman; I could never understand why she\\chose to work with children.\\}\\
{avaricious}\\
{ধনলোভী,,}\\
{(adjective): excessively greedy\\Since avaricious desire is similar to gluttony or lust--sins of excess--it was listed as one of the seven\\deadly sins by the Catholic church.\\}\\
{banal}\\
{গতানুগতিক,,}\\
{(adjective): repeated too often; overfamiliar through overuse\\The professor used such banal expression that many students in the class either fell asleep from bordeom\\or stayed awake to complete his sentences and humor friends.\\}\\
{banality}\\
{তুচ্ছতা,,}\\
{(noun): a trite or obvious remark\\Herbert regarded the minister's remark as a mere banality until Sharon pointed out profound\\implications to the seemingly obvious words.\\}\\
{belie}\\
{মিথ্যা বর্ণনা প্রদান করা,,}\\
{(verb): to give a false representation to; misrepresent\\The smile on her face belies the pain she must feel after the death of her husband.\\\\                                                                               \\}\\
{belligerent}\\
{যুধ্যমান,,}\\
{(adjective): characteristic of one eager to fight\\Tom said that he was arguing the matter purely for philosophical reasons, but his belligerent tone\\indicated an underlying anger about the issue.\\}\\
{betray}\\
{বিশ্বাসঘাতকতা করা,,}\\
{(verb): to reveal or make known something, usually unintentionally\\With the gold medal at stake, the gymnast awaited his turn, his quivering lip betraying his intense\\emotions.\\}\\
{blatant}\\
{ভয়ানক,,}\\
{(adjective): without any attempt at concealment; completely obvious\\Allen was often punished in school for blatantly disrespecting teachers.\\}\\
{bolster}\\
{তাকিয়া,,}\\
{(verb): support and strengthen\\The case for the suspect's innocence was bolstered considerably by the fact that neither fingerprints nor\\DNA were found at the scene.\\This word has other definitions but this is the most important one for the GRE\\}\\
{brazen}\\
{নির্লজ্জ,,}\\
{(adjective): unrestrained by convention or propriety\\Their large "donations" to the local police department gave the drug cartel the brazen confidence to do\\their business out in the open.\\}\\
{bucolic}\\
{রাখালী,,}\\
{(adjective): relating to the pleasant aspects of the country\\The noble families of England once owned vast expanses of beautiful, bucolic land.\\}\\
{bumbling}\\
{আনাড়ি,,}\\
{(adjective): lacking physical movement skills, especially with the hands\\Within a week of starting, the bumbling new waiter was unceremoniously fired.\\}\\
{burgeon}\\
{পাতা ধরা,,}\\
{(verb): grow and flourish\\China's housing market is burgeoning, but some predict that the growth is merely a bubble and will burst\\much like the U.S. real estate bubble of 2008.\\}\\
{calumny}\\
{দুর্নাম,,}\\
{(noun): making of a false statement meant to injure a person"s reputation\\With the presidential primaries well under way, the air is thick with calumny, and the mud already waist-\\high.\\}\\
{capricious}\\
{খামখেয়ালী,,}\\
{(adjective): determined by chance or impulse or whim rather than by necessity or reason\\Nearly every month our capricious CEO had a new plan to turn the company around, and none of them\\worked because we never gave them the time they needed to succeed.\\}\\
{castigate}\\
{শাস্তি দেওয়া}\\
{(verb): to reprimand harshly\\Drill sergeants are known to castigate new recruits so mercilessly that the latter often break down during\\their first week in training.\\}\\
{censure}\\
{ভর্ত্সনা,,}\\
{(verb): to express strong disapproval\\After being caught in bed with a mistress, the mayor was quickly censured by the city council.\\\\                                                                               \\}\\
{chastise}\\
{তাড়ন করা,,}\\
{(verb): to reprimand harshly\\Though chastised for his wanton abuse of the pantry, Lawrence shrugged off his mother's harsh words,\\and continued to plow through jars of cookies and boxes of donuts.\\}\\
{chortle}\\
{খলখল শব্দ,,}\\
{(verb): to chuckle, laugh merrily\\Walking past the bar, I could hear happy, chortling people and the blast of horns from a jazz band.\\}\\
{circumscribe}\\
{সংযত করা,,}\\
{(verb): restrict or confine\\Their tour of South America was circumscribed so that they saw only popular destinations and avoided\\the dangerous parts of cities.\\This word has other definitions but this is the most important one for the GRE\\}\\
{circumvent}\\
{ফাঁদে ফেলা,,}\\
{(verb): cleverly find a way out of one's duties or obligations\\One way of circumventing the GRE is to apply to a grad school that does not require GRE scores.\\}\\
{commensurate}\\
{তুল্য,,}\\
{(adjective): to be in proportion or corresponding in degree or amount\\The convicted felon's life sentence was commensurate to the heinousness of his crime.\\}\\
{concede}\\
{মেনে নেওয়া}\\
{(verb): acknowledge defeat\\I concede. You win!\\}\\
{concede}\\
{মেনে নেওয়া}\\
{(verb): admit (to a wrongdoing)\\After a long, stern lecture from her father, Olivia conceded to having broken the window.\\}\\
{concede}\\
{মেনে নেওয়া}\\
{(verb): give over; surrender or relinquish to the physical control of another\\The Spanish were forced to concede much of the territory they had previously conquered.\\}\\
{confound}\\
{তালগোল পাকান}\\
{(verb): be confusing or perplexing to\\Though Harry loved numbers, he found calculus confounding.\\}\\
{confound}\\
{তালগোল পাকান}\\
{(verb): mistake one thing for another\\Americans often confound sweet potatoes with yams, and refer to both vegetables by the same name.\\}\\
{conspicuous}\\
{প্রসিদ্ধ,,}\\
{(adjective): without any attempt at concealment; completely obvious\\American basketball players are always conspicuous when they go abroad--not only are they American,\\but some are over seven feet tall.\\}\\
{constituent}\\
{নির্বাচনকারী}\\
{(noun): a citizen who is represented in a government by officials for whom he or she votes\\The mayor's constituents are no longer happy with her performance and plan to vote for another\\candidate in the upcoming election.\\}\\
{constituent}\\
{নির্বাচনকারী}\\
{(noun): an abstract part of something\\The constituents of the metal alloy are nickle, copper, and tin.\\\\                                                                                 \\}\\
{construe}\\
{ভাষান্তরিত করা,,}\\
{(verb): interpreted in a particular way\\The author's inability to take a side on the issue was construed by both his opponents and supporters as\\a sign of weakness.\\}\\
{contingent}\\
{সাপেক্ষ,,}\\
{(noun): a gathering of persons representative of some larger group\\A small contingent of those loyal to the king have gathered around the castle to defend it.\\}\\
{contingent}\\
{সাপেক্ষ,,}\\
{(adjective): possible but not certain to occur\\Whether the former world champions can win again this year is contingent upon none of its star players\\getting injured.\\}\\
{contrition}\\
{অনুতাপ,,}\\
{(noun): the feeling of remorse or guilt that comes from doing something bad\\Those who show contrition during their prison terms--especially when under review by a parole board--\\often get shortened sentences.\\}\\
{contrive}\\
{মতলব আঁটা,,}\\
{(verb): to pull off a plan or scheme, usually through skill or trickery\\Despite a low GPA, he contrived to get into college, going so far as to write his own glowing letters of\\recommendation.\\This word has other definitions but this is the most important one for the GRE\\}\\
{copious}\\
{প্রচুর,,}\\
{(adjective): in abundant supply\\In midsummer, there are copious popiscle stands at the beach; in the winter, there are none.\\}\\
{craven}\\
{নীচ,,}\\
{(adjective): pathetically cowardly\\Though the man could have at least alerted the police, he crouched cravenly in the corner as the old\\woman was mugged.\\}\\
{cryptic}\\
{রহস্যপূর্ণ,,}\\
{(adjective): mysterious or vague, usually intentionally\\Since Sarah did not want her husband to guess the Christmas present she had bought him, she only\\answered cryptically when he would ask her questions about it.\\}\\
{culminate}\\
{চরম মাত্রায় পৌঁছানো,,}\\
{(verb): reach the highest or most decisive point\\Beethoven's musical genius culminated in the 9th Symphony, which many consider his greatest work.\\}\\
{culpability}\\
{নিন্দনীয়তা,,}\\
{(noun): a state of guilt\\Since John had left his banana peel at the top of the stairwell, he accepted culpability for Martha's\\broken leg.\\}\\
{decorous}\\
{সুন্দর,,}\\
{(adjective): characterized by good taste in manners and conduct\\Sally's parties are decorous affairs, and instead of the usual beer and music, there is tea and intellectual\\conversation.\\}\\
{decorum}\\
{শালীনতা,,}\\
{(noun): propriety in manners and conduct\\\\                                                                                 \\You will obey the rules of decorum for this courtroom or spend the night in a jail cell, said the judge to\\the prosecutor.\\}\\
{deferential}\\
{শ্রদ্ধাবনত,,}\\
{(adjective): showing respect\\If you ever have the chance to meet the president, stand up straight and be deferential.\\}\\
{deleterious}\\
{ক্ষতিকর,,}\\
{(adjective): harmful to living things\\The BP oil spill in the Gulf of Mexico was deleterious to the fishing industry in the southern states.\\}\\
{delineate}\\
{অঙ্কিত করা,,}\\
{(verb): describe in detail\\After a brief summary of proper swimming technique, the coach delineated the specifics of each stroke,\\spending 30 minutes alone on the backstroke.\\}\\
{demur}\\
{গম্ভীর করা,,}\\
{(verb): to object or show reluctance\\Wallace disliked the cold, so he demurred when his friends suggested they going skiing in the Alps.\\}\\
{denigrate}\\
{কলঙ্কিত করা,,}\\
{(verb): charge falsely or with malicious intent; attack the good name and reputation of\\someone\\Count Rumford denigrated the new theory of heat, demonstrating that it was wholly inadequate to\\explain the observations.\\}\\
{denote}\\
{বোঝান,,}\\
{(verb): be a sign or indication of; have as a meaning\\Even if the text is not visible, the red octagon denotes "stop" to all motorists in America.\\}\\
{derivative}\\
{উদ্ভূত}\\
{(adjective): (or a creative product, e.g. music, writing, etc.) not original but drawing on the\\work of another person\\Because the movies were utterly derivative of other popular movies, they did well at the box office.\\This word has other definitions but this is the most important one for the GRE\\}\\
{derive}\\
{উৎপত্তি নির্ণয় করা}\\
{(verb): come from; be connected by a relationship of blood, for example\\Many words in the English language are derived from Latin, including the word "derive."\\}\\
{derive}\\
{উৎপত্তি নির্ণয় করা}\\
{(verb): reason by deduction; establish by deduction\\From the multiple set of footprints in the living room, the investigator derived an important clue: Sheila\\was not alone in the room at the time of the murder.\\}\\
{dictatorial}\\
{স্বৈর,,}\\
{(adjective): expecting unquestioning obedience; characteristic of an absolute ruler\\The coach was dictatorial in his approach: no players could ever argue or question his approach.\\}\\
{didactic}\\
{নীতিমূলক,,}\\
{(adjective): instructive (especially excessively)\\Tolstoy's The Death of Ivan Illyich is a didactic novel, instructing the reader on how to live a good life.\\}\\
{diffident}\\
{অবিশ্বাসী,,}\\
{(adjective): showing modest reserve; lacking self-confidence\\As a young girl she was diffident and reserved, but now as an adult, she is confident and assertive.\\\\                                                                                  \\}\\
{dilatory}\\
{দীর্ঘসূত্রী,,}\\
{(adjective): wasting time\\Lawyers use dilatory tactics so that it takes years before the case is actually decided.\\}\\
{dilettante}\\
{সমঝদার,,}\\
{(noun): an amateur who engages in an activity without serious intentions and who pretends\\to have knowledge\\Fred has no formal medical training; while he likes to claim authority on medical issues, he is little more\\than a dilettante\\}\\
{disaffected}\\
{বীতরাগ,,}\\
{(adjective): discontented as toward authority\\After watching his superior take rations from the soliders, he quickly became disaffected and rebeled.\\}\\
{discrete}\\
{বিযুক্ত,,}\\
{(adjective): constituting a separate entity or part\\What was once known as Czechoslovakia has since split into two discrete, independent nations.\\}\\
{disinterested}\\
{নির্লিপ্ত,,}\\
{(adjective): unbiased; neutral\\The potential juror knew the defendant, and therefore could not serve on the jury, which must consist\\only of disinterested members.\\}\\
{dispassionate}\\
{নিষ্কাম,,}\\
{(adjective): unaffected by strong emotion or prejudice\\A good scientist should be dispassionate, focusing purely on what the evidence says, without personal\\attachment.\\}\\
{disseminate}\\
{প্রচার করা,,}\\
{(verb): cause to become widely known\\Before the effects of anaethesia were disseminated, patients had to experience the full pain of a surgery.\\}\\
{dogmatic}\\
{উদ্ধত,,}\\
{(adjective): highly opinionated, not accepting that your belief may not be correct\\Bryan is dogmatic in his belief that the earth is flat, claiming that all pictures of a spherical earth are\\computer generated.\\}\\
{duress}\\
{কয়েদ,,}\\
{(noun): compulsory force or threat\\The witness said he signed the contract under duress and argued that the court should cancel the\\agreement.\\}\\
{eclectic}\\
{সঙ্কীর্ণতামুক্ত}\\
{(adjective): comprised of a variety of styles\\Joey was known for his eclectic tastes in music, one moment dancing to disco the next "air conducting"\\along to Beethoven's 9th symphony.\\}\\
{economical}\\
{মিতব্যয়ী,,}\\
{(adjective): avoiding waste, efficient\\Journalists favor an economical style of writing, in which no unnecessary words are used and every\\sentence is as short as possible.\\}\\
{edifying}\\
{শিক্ষাপ্রদ,,}\\
{(adjective): enlightening or uplifting so as to encourage intellectual or moral improvement\\\\                                                                               \\I recently read an article in the Times about whether good literature is edifying or not; specifically, does\\reading more make a person more moral.\\}\\
{efficacious}\\
{ফলপ্রদ,,}\\
{(adjective): producing the intended result\\Since Maggie's cough syrup, which had expired five years back, was no longer efficacious, she coughed\\through the night.\\}\\
{egregious}\\
{গুরুতর,,}\\
{(adjective): standing out in negative way; shockingly bad\\The dictator's abuse of human rights was so egregious that many world leaders asked that he be tried in\\an international court for genocide.\\}\\
{elicit}\\
{প্রকাশ করা,,}\\
{(verb): call forth (emotions, feelings, and responses)\\Just smiling--even if you are depressed--can elicit feelings of pleasure and happiness.\\}\\
{elucidate}\\
{স্পষ্ট করা,,}\\
{(verb): make clearer and easier to understand\\Youtube is great place to learn just about anything--an expert elucidates finer points so that even a\\complete novice can learn.\\}\\
{eminent}\\
{বিশিষ্ট,,}\\
{(adjective): standing above others in quality or position\\Shakespeare is an eminent author in the English language, but I find his writing uninteresting and\\melodramatic.\\}\\
{enervate}\\
{দুর্বল করা,,}\\
{(verb): to sap energy from\\John preferred to avoid equatorial countries; the intense sun would always leave him enervated after\\he'd spent the day sightseeing.\\}\\
{engender}\\
{জন্ম দেওয়া}\\
{(verb): give rise to\\The restrictions of the Treaty of Versailles were so severe that they engendered deep hatred and\\resentment in the German people.\\}\\
{entrenched}\\
{জড়িত,,}\\
{(adjective): fixed firmly or securely\\By the time we reach 60-years old, most of our habits are so entrenched that it is difficult for us to\\change.\\}\\
{ephemeral}\\
{ক্ষণজীবী,,}\\
{(adjective): lasting a very short time\\The lifespan of a mayfly is ephemeral, lasting from a few hours to a couple of days.\\}\\
{equivocal}\\
{সন্দেহজনক,,}\\
{(adjective): confusing or ambiguous\\The findings of the study were equivocal--the two researchers had different opinions on what the results\\signified.\\}\\
{eradicate}\\
{সমূলে উৎপাটিত করা}\\
{(verb): to completely destroy\\I tried eradicating the mosquitos in my apartment with a rolled up newspaper, but there were too many\\\\                                                                              \\of them.\\}\\
{erudite}\\
{জ্ঞানী,,}\\
{(adjective): having or showing profound knowledge\\Before the Internet, the library was typically were you would find erudite readers.\\}\\
{eschew}\\
{পরিহার করা,,}\\
{(verb): avoid and stay away from deliberately; stay clear of\\Politicians are the masters of eschewing morals; academics are the masters of eschewing clarity.\\}\\
{esoteric}\\
{গূঢ়,,}\\
{(adjective): confined to and understandable by only an enlightened inner circle\\Map collecting is an esoteric hobby to most, but to geography geeks it is a highly enjoyable pasttime.\\}\\
{espouse}\\
{সমর্থন দান করা}\\
{(verb): to adopt or support an idea or cause\\As a college student, Charlie espoused Marxism, growing his beard out and railing against the evils of the\\free-market.\\}\\
{exacerbate}\\
{বর্ধিত করা,,}\\
{(verb): make worse\\Her sleeplessness exacerbated her cold--when she woke up the next day, her sinuses were completely\\blocked.\\}\\
{exacting}\\
{ছিদ্রান্বেষী,,}\\
{(adjective): requiring and demanding accuracy\\Though his childhood piano teacher was so exacting, Max is thankful now, as a professional pianist.\\}\\
{exalt}\\
{মহিমান্বিত করা}\\
{(verb): praise or glorify\\The teenagers exalted the rock star, covering their bedrooms with posters of him.\\}\\
{exonerate}\\
{পুনর্বাসন করা,,}\\
{(verb): pronounce not guilty of criminal charges\\The document clearly indicated that Nick was out of the state at the time of the crime, and so served to\\exonerate him of any charges.\\}\\
{expound}\\
{মানে করা,,}\\
{(verb): add details or explanation; clarify the meaning; state in depth\\The CEO refused to expound on the decision to merge our department with another one, and so I quit.\\}\\
{extant}\\
{অদ্যাপি বর্তমান,,}\\
{(adjective): the opposite of extinct\\Despite many bookstores closing, experts predict that some form of book dealing will still be extant\\generations from now.\\}\\
{fallacious}\\
{প্রতারণাপূর্ণ,,}\\
{(adjective): of a belief that is based on faulty reasoning\\The widespread belief that Eskimos have forty different words for snow is fallacious, based on one false\\report.\\}\\
{fastidious}\\
{খুঁতখুঁতে,,}\\
{(adjective): overly concerned with details; fussy\\Whitney is fastidious about her shoes, arranging them on a shelf in a specific order, each pair evenly\\spaced.\\\\                                                                                \\}\\
{flux}\\
{গলান,,}\\
{(noun): a state of uncertainty about what should be done (usually following some important event)\\Ever since Elvira resigned as the head of marketing, everything about our sales strategy has been in a\\state of flux.\\}\\
{foment}\\
{গজগজ করা}\\
{(verb): try to stir up public opinion\\After having his pay cut, Phil spread vicious rumors about his boss, hoping to foment a general feeling of\\discontent.\\}\\
{forlorn}\\
{পরিত্যক্ত}\\
{(adjective): marked by or showing hopelessness\\After her third pet dog died, Marcia was simply forlorn: this time even the possibility of buying a new dog\\no longer held any joy.\\}\\
{forthcoming}\\
{আসন্ন,,}\\
{(adjective): available when required or as promised\\The President announced that the senators were about to reach a compromise, and that he was eager to\\read the forthcoming details of the bill.\\}\\
{forthcoming}\\
{আসন্ন,,}\\
{(adjective): at ease in talking to others\\As a husband, Larry was not forthcoming: if Jill didn't demand to know details, Larry would never share\\them with her.\\}\\
{fortuitous}\\
{আধিদৈবিক,,}\\
{(adjective): occurring by happy chance; having no cause or apparent cause\\While the real objects are vastly different sizes in space, the sun and the moon seem to have the same\\fortuitous size in the sky.\\}\\
{frivolous}\\
{অসার,,}\\
{(adjective): not serious in content or attitude or behavior\\Compared to Juliet's passionate concern for human rights, Jake's non-stop concern about football seems\\somewhat frivolous.\\}\\
{frugal}\\
{মিতব্যয়ী,,}\\
{(adjective): not spending much money (but spending wisely)\\Monte was no miser, but was simply frugal, wisely spending the little that he earned.\\}\\
{frustrate}\\
{পরাভূত করা,,}\\
{(verb): hinder or prevent (the efforts, plans, or desires) of\\I thought I would finish writing the paper by lunchtime, but a number of urgent interruptions served to\\frustrate my plan.\\This word has other definitions but this is the most important one for the GRE\\}\\
{furtive}\\
{অলক্ষিত,,}\\
{(adjective): marked by quiet and caution and secrecy; taking pains to avoid being observed\\While at work, George and his boss Regina felt the need to be as furtive as possible about their romantic\\relationship.\\}\\
{gainsay}\\
{প্রতিবাদ করা,,}\\
{(verb): deny or contradict; speak against or oppose\\I can't gainsay a single piece of evidence James has presented, but I still don't trust his conclusion.\\\\                                                                                \\}\\
{gall}\\
{পিত্ত,,}\\
{(noun): the trait of being rude and impertinent\\Even though Carly was only recently hired, she had the gall to question her boss's judgment in front of\\the office.\\}\\
{gall}\\
{পিত্ত,,}\\
{(noun): feeling of deep and bitter anger and ill-will\\In an act of gall, Leah sent compromising photos of her ex-boyfriend to all his co-workers and\\professional contacts.\\}\\
{galvanize}\\
{রাংঝালাই করা,,}\\
{(verb): to excite or inspire (someone) to action\\At mile 23 of his first marathon, Kyle had all but given up, until he noticed his friends and family holding\\a banner that read, "Go Kyle"; galvanized, he broke into a gallop, finishing the last three miles in less\\than 20 minutes.\\}\\
{garrulous}\\
{ফচকে,,}\\
{(adjective): full of trivial conversation\\Lynne was garrulous: once, she had a fifteen minute conversation with a stranger before she realized the\\woman didn't speak English.\\}\\
{gauche}\\
{আনাড়ি,,}\\
{(adjective): lacking social polish\\Sylvester says the most gauche things, such as telling a girl he liked that she was much prettier when she\\wore makeup.\\}\\
{germane}\\
{সঙ্গত,,}\\
{(adjective): relevant and appropriate\\The professor wanted to tell the jury in detail about his new book, but the lawyer said it wasn't germane\\to the charges in the cases.\\}\\
{glut}\\
{গোগ্রাসে গেলা,,}\\
{(noun): an excessive supply\\The Internet offers such a glut of news related stories that many find it difficult to know which story to\\read first.\\}\\
{glut}\\
{গোগ্রাসে গেলা,,}\\
{(verb): supply with an excess of\\In the middle of economic crises, hiring managers find their inboxes glutted with resumes.\\}\\
{gossamer}\\
{লূতাতন্তু,,}\\
{(adjective): characterized by unusual lightness and delicacy\\The gossamer wings of a butterfly, which allow it to fly, are also a curse, so delicate that they are often\\damaged.\\}\\
{gregarious}\\
{যূথচর,,}\\
{(adjective): to be likely to socialize with others\\Often we think that great leaders are those who are gregarious, always in the middle of a large group of\\people; yet, as Mahatma Gandhi and many others have shown us, leaders can also be introverted.\\}\\
{guileless}\\
{সরল,,}\\
{(adjective): free of deceit\\At first I thought my niece was guileless, but I then found myself buying her ice cream every time we\\passed a shop.\\\\                                                                               \\}\\
{hackneyed}\\
{বস্তাপচা,,}\\
{(adjective): lacking significance through having been overused\\Cheryl rolled her eyes when she heard the lecturer's hackneyed advice to "be true to yourself."\\}\\
{haphazard}\\
{এলোমেলো,,}\\
{(adjective): marked by great carelessness; dependent upon or characterized by chance\\Many golf courses are designed with great care, but the greens on the county golf course seem entirely\\haphazard.\\}\\
{harangue}\\
{বাগাড়ম্বরপূর্ণ বক্তৃতা,,}\\
{(noun): a long pompous speech; a tirade\\Dinner at Billy's was more a punishment than a reward, since anyone who sat at the dinner table would\\have to listen to Billy's father's interminable harangues against the government.\\}\\
{harangue}\\
{বাগাড়ম্বরপূর্ণ বক্তৃতা,,}\\
{(verb): to deliver a long pompous speech or tirade\\Tired of his parents haranguing him about his laziness and lack of initiative, Tyler finally moved out of\\home at the age of thirty-five.\\}\\
{harried}\\
{নিপীড়িত,,}\\
{(adjective): troubled persistently especially with petty annoyances\\With a team of new hires to train, Martha was constantly harried with little questions and could not\\focus on her projects.\\}\\
{haughty}\\
{উদ্ধত,,}\\
{(adjective): having or showing arrogant superiority to and disdain of those one views as\\unworthy\\The haughty manager didn't believe that any of his subordinates could ever have an insight as brilliant\\his own.\\}\\
{hegemony}\\
{কর্তৃত্ব,,}\\
{(adjective): dominance over a certain area\\Until the Spanish Armada was defeated in 1587, Spain had hegemony over the seas, controlling waters\\stretching as far as the Americas.\\}\\
{heretic}\\
{পাষণ্ড,,}\\
{(noun): a person who holds unorthodox opinions in any field (not merely religion)\\Though everybody at the gym told Mikey to do cardio before weights, Mikey was a heretic and always\\did the reverse.\\}\\
{iconoclast}\\
{কালাপাহাড়,,}\\
{(noun): somebody who attacks cherished beliefs or institutions\\Lady Gaga, in challenging what it means to be clothed, is an iconoclast for wearing a "meat dress" to a\\prominent awards show.\\}\\
{iconoclastic}\\
{কালাপাহাড়ী,,}\\
{(adjective): defying tradition or convention\\Jackson Pollack was an iconoclastic artist, totally breaking with tradition by splashing paint on a blank\\canvas.\\}\\
{idiosyncrasy}\\
{মানসিক গঠন,,}\\
{(noun): a behavioral attribute that is distinctive and peculiar to an individual\\Peggy's numerous idiosyncrasies include wearing mismatched shoes, laughing loudly to herself, and\\\\                                                                                \\owning a pet aardvark.\\}\\
{ignoble}\\
{নীচ,,}\\
{(adjective): dishonorable\\In the 1920s, the World Series was rigged--an ignoble act which baseball took decades to recover from.\\}\\
{ignominious}\\
{কলঙ্কজনক,,}\\
{(adjective): (used of conduct or character) deserving or bringing disgrace or shame\\Since the politician preached ethics and morality, his texting of revealing photographs was ignominious,\\bringing shame on both himself and his party.\\}\\
{immutable}\\
{অপরিবর্তনীয়}\\
{(adjective): not able to be changed\\Taxes are one of the immutable laws of the land, so there is no use arguing about paying them.\\}\\
{impartial}\\
{নিরপেক্ষ,,}\\
{(adjective): free from undue bias or preconceived opinions\\The judge was not impartial since he had been bribed by the witness's family.\\}\\
{impertinent}\\
{অপ্রাসঙ্গিক,,}\\
{(adjective): being disrespectful; improperly forward or bold\\Dexter, distraught over losing his pet dachshund, Madeline, found the police officer's questions\\impertinent--after all, he thought, did she have to pry into such details as to what Madeline's favorite\\snack was?\\}\\
{implacable}\\
{নির্দয়,,}\\
{(adjective): incapable of making less angry or hostile\\Win or lose, the coach was always implacable, never giving the athletes an easy practice or a break.\\}\\
{implausible}\\
{অভাবনীয়,,}\\
{(adjective): describing a statement that is not believable\\The teacher found it implausible that the student was late to school because he had been kidnapped by\\outlaws on horseback.\\}\\
{imprudent}\\
{হঠকারী,,}\\
{(adjective): not wise\\Hitler, like Napoleon, made the imprudent move of invading Russia in winter, suffering even more\\casualties than Napoleon had.\\}\\
{impudent}\\
{বেহায়া,,}\\
{(adjective): improperly forward or bold\\In an impudent move, the defendant spoke out of order to say terribly insulting things to the judge.\\}\\
{incisive}\\
{ব্যঙ্গকারী,,}\\
{(adjective): having or demonstrating ability to recognize or draw fine distinctions\\The lawyer had an incisive mind, able in a flash to dissect a hopelessly tangled issue and isolate the\\essential laws at play.\\}\\
{incongruous}\\
{বেমানান,,}\\
{(adjective): lacking in harmony or compatibility or appropriateness\\The vast economic inequality of modern society is incongruous with America's ideals.\\}\\
{incorrigible}\\
{সংশোধনাতীত,,}\\
{(adjective): impervious to correction by punishment\\Tom Sawyer seems like an incorrigible youth until Huck Finn enters the novel; even Sawyer can't match\\\\                                                                                \\his fierce individual spirit.\\}\\
{indecorous}\\
{অনুচিত,,}\\
{(adjective): not in keeping with accepted standards of what is right or proper in polite\\society\\Eating with elbows on the table is considered indecorous in refined circles.\\}\\
{indifference}\\
{অযত্ন,,}\\
{(noun): the trait of seeming not to care\\In an effort to fight indifference, the president of the college introduced a new, stricter grading system.\\}\\
{inexorable}\\
{নির্দয়,,}\\
{(adjective): impossible to stop or prevent\\The rise of the computer was an inexorable shift in technology and culture.\\}\\
{ingenuous}\\
{অকৃত্রিম,,}\\
{(adjective): to be naïve and innocent\\Two-years in Manhattan had changed Jenna from an ingenuous girl from the suburbs to a jaded\\urbanite, unlikely to fall for any ruse, regardless of how elaborate.\\}\\
{ingratiate}\\
{অনুগ্রহ ভাজন করান,,}\\
{(verb): gain favor with somebody by deliberate efforts\\Even though Tom didn't like his new boss, he decided to ingratiate himself to her in order to advance his\\career.\\}\\
{inimical}\\
{শত্রুভাবাপন্ন,,}\\
{(adjective): hostile (usually describes conditions or environments)\\Venus, with a surface temperature that would turn rubber to liquid, is inimical to any form of life.\\}\\
{innocuous}\\
{নির্দোষ,,}\\
{(adjective): harmless and doesn"t produce any ill effects\\Everyone found Nancy's banter innocuous--except for Mike, who felt like she was intentionally picking on\\him.\\}\\
{inscrutable}\\
{অবর্ণনীয়,,}\\
{(adjective): not easily understood; unfathomable\\His speech was so dense and confusing that many in the audience found it inscrutable.\\}\\
{insidious}\\
{প্রতারণাপূর্ণ,,}\\
{(adjective): working in a subtle but destructive way\\Plaque is insidious: we cannot see it, but each day it eats away at our enamel, causing cavities and other\\dental problems.\\}\\
{insolent}\\
{দাম্ভিক,,}\\
{(adjective): rude and arrogant\\Lilian could not help herself from being insolent, commenting that the Queen's shoes were showing too\\much toe.\\}\\
{intimate}\\
{ঘনিষ্ঠ,,}\\
{(verb): to suggest something subtly\\At first Manfred's teachers intimated to his parents that he was not suited to skip a grade; when his\\parents protested, teachers explicitly told them that, notwithstanding the boy's precocity, he was simply\\too immature to jump to the 6th grade.\\This word has other definitions but this is the most important one for the GRE\\\\                                                                               \\}\\
{intransigent}\\
{একরোখা,,}\\
{(adjective): unwilling to change one's beliefs or course of action\\Despite many calls for mercy, the judge remained intransigent, citing strict legal precedence.\\}\\
{intrepid}\\
{নিরাতঙ্ক,,}\\
{(adjective): fearless\\Captain Ahab was an intrepid captain whose reckless and fearless style ultimate leads to his downfall.\\}\\
{inveterate}\\
{পাকা,,}\\
{(adjective): habitual\\He is an inveterate smoker and has told his family and friends that there is no way he will ever quit.\\}\\
{involved}\\
{জড়িত}\\
{(adjective): complicated, and difficult to comprehend\\The physics lecture became so involved that the undergraduate's eyes glazed over.\\}\\
{irrevocable}\\
{অনড়,,}\\
{(adjective): incapable of being retracted or revoked\\Once you enter your plea to the court, it is irrevocable so think carefully about what you will say.\\}\\
{itinerant}\\
{ভ্রাম্যমান,,}\\
{(adjective): traveling from place to place to work\\Doctors used to be itinerant, traveling between patients' homes.\\}\\
{jingoism}\\
{সংগ্রামপ্রি় দেশপ্রেম,,}\\
{(noun): fanatical patriotism\\North Korea maintains intense control over its population through a combination of jingoism and cult of\\personality.\\}\\
{jovial}\\
{আমুদে}\\
{(adjective): full of or showing high-spirited merriment\\The political candidate and his supporters were jovial once it was clear that she had won.\\}\\
{jubilant}\\
{আনন্দগর্বে মত্ত}\\
{(adjective): full of high-spirited delight because of triumph or success\\My hardwork paid off, and I was jubilant to receive a perfect score on the GRE.\\}\\
{juxtapose}\\
{পাশাপাশি স্থাপন করা,,}\\
{(verb): place side by side\\The meaning of her paintings comes from a classical style which juxtaposes modern themes.\\}\\
{laconic}\\
{স্বল্পবাক,,}\\
{(adjective): one who says very few words\\While Martha always swooned over the hunky, laconic types in romantic comedies, her boyfriends\\inevitably were very talkative--and not very hunky.\\}\\
{lambast}\\
{}\\
{(verb): criticize severely or angrily\\Showing no patience, the manager utterly lambasted the sales team that lost the big account.\\}\\
{languid}\\
{অবসন্ন}\\
{(adjective): not inclined towards physical exertion or effort; slow and relaxed\\As the sun beat down and the temperature climbed higher, we spent a languid week lying around the\\house.\\}\\
{largess}\\
{বদান্যতা,,}\\
{(noun): extreme generosity and giving\\\\                                                                               \\Uncle Frank was known for his largess, so his nephew was sad when he did not receive a present for his\\birthday.\\}\\
{laudable}\\
{প্রশংসনীয়,,}\\
{(adjective): worthy of high praise\\To say that Gandhi's actions were laudable is the greatest understatement; he overthrew an empire\\without violence.\\}\\
{lionize}\\
{}\\
{(verb): assign great social importance to\\Students in the U.S. learn to lionize Jefferson, Franklin, and Washington because they are the founding\\fathers of the nation.\\}\\
{magnanimous}\\
{মহানুভব,,}\\
{(adjective): noble and generous in spirit, especially towards a rival or someone less\\powerful\\He was a great sportsman: in defeat he was complementary and in victory he was magnanimous.\\}\\
{maintain}\\
{বজায় রাখা}\\
{(verb): to assert\\The scientist maintained that the extinction of dinosaurs was most likely brought about by a drastic\\change in climate.\\}\\
{maladroit}\\
{জবুথবু,,}\\
{(adjective): clumsy\\As a child she was quite maladroit, but as an adult, she has become an adept dancer.\\This word has other definitions but this is the most important one for the GRE\\}\\
{maverick}\\
{বাউণ্ডুলে,,}\\
{(noun): someone who exhibits great independence in thought and action\\Officer Kelly was a maverick, rarely following police protocols or adopting the conventions for speech\\common among his fellow officers.\\}\\
{mawkish}\\
{ঘৃণাজনক,,}\\
{(adjective): overly sentimental to the point that it is disgusting\\The film was incredibly mawkish, introducing highly likeable characters only to have them succumb to a\\devastating illnesses by the end of the movie.\\}\\
{mendacity}\\
{অসত্যতা,,}\\
{(noun): the tendency to be untruthful\\I can forgive her for her mendacity but only because she is a child and is seeing what she can get away\\with.\\}\\
{mercurial}\\
{পারাযুক্ত,,}\\
{(adjective): (of a person) prone to unexpected and unpredictable changes in mood\\The fact that Ella's moods were as mercurial as the weather was problematic for her relationships--it\\didn't help that she lived in Chicago.\\}\\
{meticulous}\\
{অতিসতর্ক,,}\\
{(adjective): marked by extreme care in treatment of details\\The Japanese noodle maker was meticulous in making his noodles and would never let another person\\take over the task.\\}\\
{misconstrue}\\
{ভুল অর্থ করা,,}\\
{(verb): interpret in the wrong way\\\\                                                                               \\The politician never trusted journalists because he though that they misconstrue his words and\\misrepresent his positions.\\}\\
{mitigate}\\
{উপশম করা,,}\\
{(verb): make less severe or harsh\\I can only spend so much time mitigating your disagreements with your wife, and at certain point, you\\need to do it on your own.\\}\\
{mitigate}\\
{উপশম করা,,}\\
{(verb): lessen the severity of an offense\\If it weren't for the mitigating circumstances, he would have certainly lost his job.\\}\\
{mollify}\\
{শান্ত করা,,}\\
{(verb): to make someone angry less angry; placate\\In the morning, Harriat was unable to mollify Harry, if he happened to become angry, unless he had his\\cup of coffee.\\}\\
{mundane}\\
{জাগতিক,,}\\
{(adjective): repetitive and boring; not spiritual\\Nancy found doing dishes a thorougly mundane task, although Peter found a kind of Zen pleasure in the\\chore.\\}\\
{mundane}\\
{জাগতিক,,}\\
{(adjective): relating to the ordinary world\\Though we think of the pope as someone always dealing in holy matters, he is also concerned with\\mundane events, such as deciding when to set his alarm each morning.\\}\\
{munificent}\\
{মুক্তহস্ত,,}\\
{(adjective): very generous\\Uncle Charley was known for his munificence, giving all seven of his nephews lavish Christmas presents\\each year.\\}\\
{myopic}\\
{ক্ষীণদৃষ্টি,,}\\
{(adjective): lacking foresight or imagination\\The company ultimately went out of business because the myopic managers couldn't predict the changes\\in their industry.\\This word has other definitions but this is the most important one for the GRE\\}\\
{myriad}\\
{অগণ্য}\\
{(noun): a large indefinite number\\There are a myriad of internet sites hawking pills that claim to boost energy for hours on end.\\}\\
{negligible}\\
{তুচ্ছ,,}\\
{(adjective): so small as to be meaningless; insignificant\\The GRE tests cumulative knowledge, so if you cram the night before it is, at best, likely to only have a\\negligible impact on your score.\\}\\
{nonplussed}\\
{আবিষ্ট,,}\\
{(verb): unsure how to act or respond\\Shirley was totally nonplussed when the angry motorist cut her off and then stuck his finger out the\\window.\\}\\
{nuance}\\
{সামান্য পার্থক্য,,}\\
{(noun): a subtle difference in meaning or opinion or attitude\\Because of the nuances involved in this case, I hired an outside consultant to advice us and help.\\\\                                                                                 \\}\\
{obscure}\\
{অখ্যাত,,}\\
{(verb): make unclear\\On the Smith's drive through the Grand Canyon, Mr. Smith's big head obscured much of Mrs. Robinson's\\view, so that she only saw momentary patches of red rock.\\This word has other definitions but this is the most important one for the GRE\\}\\
{obscure}\\
{অখ্যাত,,}\\
{(adjective): known by only a few\\Many of the biggest movie stars were once obscure actors who got only bit roles in long forgotten films.\\This word has other definitions but this is the most important one for the GRE\\}\\
{obsequious}\\
{চাটুকার,,}\\
{(adjective): attentive in an ingratiating or servile manner; attempting to win favor from\\influential people by flattery\\The obsequious waiter did not give the couple a moment's peace all through the meal, constantly\\returning to their table to refill their water glasses and to tell them what a handsome pair they made.\\}\\
{opaque}\\
{অস্বচ্ছ}\\
{(adjective): not clearly understood or expressed\\The meaning of the professor's new research was opaque to most people, so no one asked any questions.\\This word has other definitions but this is the most important one for the GRE\\}\\
{opulence}\\
{সমৃদ্ধি,,}\\
{(noun): wealth as evidenced by sumptuous living\\Russian oligarchs are famous for their opulence, living in fancy homes and dining on expensive cavier.\\}\\
{ostentatious}\\
{ভানপূর্ণ,,}\\
{(adjective): intended to attract notice and impress others; tawdry or vulgar\\Matt wanted to buy stone lions for front of the house, but Cynthia convinced him that such a display\\would be too ostentatious for a modest house in an unassuming neighborhood.\\}\\
{ostracize}\\
{নির্বাসিত করা,,}\\
{(verb): exclude from a community or group\\Later in his life, Leo Tolstoy was ostracized from the Russian Orthodox Church for his writings that\\contradicted church doctrine.\\}\\
{panache}\\
{শিরস্ত্রাণের উঅপরে ব্যবহৃত পাখির পালক,,}\\
{(noun): distinctive and stylish elegance\\Jim, with his typical panache, came to the wedding reception with a top hat, a cane, and a long cape\\covered in sequins.\\}\\
{parochial}\\
{সংকীর্ণ,,}\\
{(adjective): narrowly restricted in scope or outlook\\Jasmine was sad to admit it, but her fledgling relationship with Jacob did not work out because his\\culinary tastes were simply too parochial; "After all," she quipped on her blog, "he considered Chef\\Boyardee ethnic food."\\}\\
{parsimonious}\\
{মিতব্যয়ী,,}\\
{(adjective): extremely frugal; miserly\\Katie is so parsimonious that she only buys a pair of socks if all of her other socks have holes in them.\\}\\
{pedantic}\\
{পণ্ডিতিপনা}\\
{(adjective): marked by a narrow focus on or display of learning especially its trivial aspects\\Professor Thompson was regarded as an expert in his field, but his lectures were utterly pedantic,\\focused on rigorous details of the most trivial conventions in the field.\\\\                                                                                 \\}\\
{pedestrian}\\
{পথচারী,,}\\
{(adjective): lacking imagination\\While Nan was always engaged in philosophical speculation, her brother was occupied with far more\\pedestrian concerns: how to earn a salary and run a household.\\This word has other definitions but this is the most important one for the GRE\\}\\
{pejorative}\\
{মূল্যহানিকর,,}\\
{(adjective): expressing disapproval (usu. refers to a term)\\Most psychologists object to the pejorative term "shrink", believing that they expand the human mind,\\not limit it.\\This word has other definitions but this is the most important one for the GRE\\}\\
{perfidy}\\
{বিশ্বাসভঙ্গ,,}\\
{(noun): an act of deliberate betrayal; a breach of a trust\\The lowest circles in Dante's Inferno were for those who had practiced perfidy, and among these, the\\very lowest was for those, such as Judas, who had been treacherous to one of their benefactors.\\}\\
{pernicious}\\
{ক্ষতিকর,,}\\
{(adjective): exceedingly harmful; working or spreading in a hidden and injurious way\\The most successful viruses are pernicious: an infected person may feel perfectly healthy for several\\months while incubating and spreading the virus.\\}\\
{petulant}\\
{বিরক্ত,,}\\
{(adjective): easily irritated or annoyed\\When Ed first met Ruth, he didn't realize she was so petulant, but now that they are three months into\\their relationship, Ed feels a day doesn't go by in which she isn't whining about some minor issue.\\}\\
{placate}\\
{শান্ত করা,,}\\
{(verb): cause to be more favorably inclined; gain the good will of\\I was able to placate the angry mob of students by promising to bring cookies on Monday.\\}\\
{platitude}\\
{ধুয়া,,}\\
{(noun): a trite or obvious remark\\The professor argued that many statements regarded as wise in previous times, such as the Golden Rule,\\are now regarded as mere platitudes.\\}\\
{poignant}\\
{গ্লানিকর,,}\\
{(adjective): emotionally touching\\After the Montagues and Capulets discover the dead bodies of Romeo and Juliet, in the play's most\\poignant moment, the two griefstricken familes agree to end their feud once and for all.\\This word has other definitions but this is the most important one for the GRE\\}\\
{polemic}\\
{বিতর্কমূলক,,}\\
{(noun): a strong verbal or written attack on someone or something.\\The professor launched into a polemic, claiming that Freudian theory was a pack of lies that absolutely\\destroyed European literary theory.\\This word has other definitions but this is the most important one for the GRE\\}\\
{posit}\\
{যথাস্থানে রাখা,,}\\
{(verb): assume as fact\\Initially, Einstein posited a repulsive force to balance Gravity, but then rejected that idea as a blunder.\\}\\
{pragmatic}\\
{রাষ্ট্রীয়,,}\\
{(adjective): guided by practical experience and observation rather than theory\\Rather than make a philosophical appeal to the Congressmen, the Speaker decided to take a far more\\pragmatic approach, making small side-deals that would add votes to his bill.\\}\\
{precipitous}\\
{}\\
{(adjective): done with very great haste and without due deliberation\\\\                                                                               \\He was expecting a precipitous rise in the value of a "hot" tech stock, so he was disappointed when it\\only inched up a dollar or two each day.\\}\\
{preclude}\\
{প্রতিরোধ করা,,}\\
{(verb): keep from happening or arising; make impossible\\The manager specified that all other gates be locked, to preclude the possibility of persons without\\tickets entering the arena undetected.\\}\\
{precocious}\\
{অকালপক্ক}\\
{(adjective): characterized by or characteristic of exceptionally early development or maturity\\(especially in mental aptitude)\\Though only seven years old, she was a precocious chess prodigy, able to beat players twice her age.\\}\\
{predilection}\\
{পক্ষপাত,,}\\
{(noun): a strong liking\\Monte had a predilection for the fine things in life: Cuban cigars, 200 dollar bottles of wine, and trips to\\the French Riviera.\\}\\
{prescience}\\
{দূরদর্শিতা,,}\\
{(noun): the power to foresee the future\\Baxter's warnings about investing in technology stocks seemed like an act of prescience after the whole\\market declined significantly.\\}\\
{prevaricate}\\
{সত্যের অপলাপ করা,,}\\
{(verb): to speak in an evasive way\\The cynic quipped, "There is not much variance in politicians; they all seem to prevaricate".\\}\\
{prodigal}\\
{অমিতব্যয়ী,,}\\
{(adjective): rashly or wastefully extravagant\\Successful professional athletes who do not fall prey to prodigality seem to be the exception--most live\\decadent lives.\\}\\
{prodigious}\\
{বিস্ময়কর,,}\\
{(adjective): so great in size or force or extent as to elicit awe\\After the relatively small homerun totals in the "dead ball" era, Babe Ruth's homerun totals were truly\\prodigious: every year, he set a new all-time record.\\}\\
{profligate}\\
{লক্ষ্মীছাড়া,,}\\
{(adjective): spending money recklessly or wastefully\\}\\
{profligate}\\
{লক্ষ্মীছাড়া,,}\\
{(noun): someone who spends money recklessly or wastefully\\Most lottery winners go from being conservative, frugal types to outright profligates who blow millions\\on fast cars, lavish homes, and giant yachts.\\}\\
{prolific}\\
{উর্বর,,}\\
{(adjective): intellectually productive\\Schubert was the most prolific composer, producing hundreds of hours of music before he died at the age\\of 31.\\}\\
{propitious}\\
{প্রসন্ন,,}\\
{(adjective): presenting favorable circumstances; likely to result in or show signs of success\\The child's heartbeat is still weak, but I am seeing many propitious signs and I think that she may be\\healing.\\\\                                                                                \\}\\
{provincial}\\
{প্রাদেশিক}\\
{(adjective): characteristic of the a limited perspective; not fashionable or sophisticated\\Maggie's enthusiasm about her high school teams seemed provincial to her college classmates, all of\\whom were following a nationally ranked college team.\\}\\
{pundit}\\
{পণ্ডিত,,}\\
{(noun): someone who has been admitted to membership in a scholarly field\\Steven Pinker's credentials are unquestioned as a pundit; he has taught at MIT and Stanford, teaches at\\Harvard, and has published a number of influential books on cognition, language, and psychology.\\}\\
{qualify}\\
{যোগ্যতা,,}\\
{(adjective): to be legally competent or capable\\If James had made more than \$50,000 last year, then he wouldn't have qualified for the low-income\\scholarship.\\}\\
{qualify}\\
{যোগ্যতা,,}\\
{(verb): to make less severe; to limit (a statement)\\Chris qualified his love for San Francisco, adding that he didn't like the weather as much as the weather\\in Los Angeles.\\}\\
{querulous}\\
{অসন্তুষ্ট,,}\\
{(adjective): habitually complaining\\The querulous old woman was begining to wear down even the happier members of the staff with her\\ceaseless complaining.\\}\\
{quotidian}\\
{প্রাত্যহিক,,}\\
{(adjective): found in the ordinary course of events\\Phil gets so involved thinking about Aristotle's arguments that he totally forgets quotidian concerns, such\\as exercising and eating regularly.\\}\\
{ravenous}\\
{বুভুক্ষিত,,}\\
{(adjective): extremely hungry; devouring or craving food in great quantities\\John didn't each much at all during the week he had the flu, so now that he is regaining his health, it's\\not surpring that he has a ravenous appetite.\\}\\
{rebuke}\\
{তীব্র তিরস্কার করা}\\
{(verb): criticize severely or angrily; censure\\The police chief rebuked the two officers whose irresponsible decisions almost led to the deaths of seven\\innocent by-standers.\\}\\
{reconcile}\\
{মিলনসাধন করা}\\
{(verb): make (one thing) compatible with (another)\\Peggy was unable to reconcile her kind friend Jane with the cruel and merciless character Jane played on\\television.\\}\\
{recondite}\\
{দুর্বোধ্য,,}\\
{(adjective): difficult to penetrate; incomprehensible to one of ordinary understanding or\\knowledge\\I found Ulysses recondite and never finished the book, waiting instead to read it with someone else so we\\could penetrate its meaning together.\\}\\
{refractory}\\
{অবাধ্য,,}\\
{(adjective): stubbornly resistant to authority or control\\\\                                                                                \\Used to studious high school students, Martha was unprepared for the refractory Kindgergarteners who\\neither sat still nor listened to a single word she said.\\This word has other definitions but this is the most important one for the GRE\\}\\
{refute}\\
{খণ্ডন করা,,}\\
{(verb): prove to be false or incorrect\\No one could refute his theories or propositions, and that is why he was esteemed by all his colleagues in\\the philosophy department.\\}\\
{reproach}\\
{গাল পাড়া}\\
{(verb): to express criticism towards\\At first, Sarah was going to yell at the boy, but she didn't want to reproach him for telling the truth about\\the situation.\\}\\
{repudiate}\\
{অস্বীকার করা,,}\\
{(verb): reject as untrue or unfounded\\Many in the public believed the rumors of a UFO crash outside town, so the chief of police did everything\\he could to repudiate the rumors.\\}\\
{rescind}\\
{রদ করা,,}\\
{(verb): cancel officially\\The man's driver's license was rescinded after his tenth car accident, which meant he would never be\\allowed to legally drive again.\\}\\
{restive}\\
{অশান্ত,,}\\
{(adjective): restless\\The crowd grew restive as the comedian's opening jokes fell flat.\\}\\
{resurgent}\\
{পুনরুজ্জীবিত}\\
{(adjective): rising again as to new life and vigor\\The team sank to fourth place in June, but is now resurgent and about to win the division.\\}\\
{reticent}\\
{স্বল্পভাষী,,}\\
{(adjective): reluctant to draw attention to yourself; temperamentally disinclined to talk\\When asked about her father, Helen lost her outward enthusiasm and became rather reticent.\\}\\
{reverent}\\
{শ্রদ্ধাশীল}\\
{(adjective): feeling or showing profound respect or veneration\\The professor could speak objectively about the other composers, but he always lectured about Brahms\\with a particularly reverent air, unable to offer a single criticism of his compositions.\\}\\
{rudimentary}\\
{প্রাথমিক}\\
{(adjective): being in the earliest stages of development; being or involving basic facts or\\principles\\I would love to be able to present a fully polished proposal to the board, but right now, our plans for the\\product are still in the most rudimentary stages.\\}\\
{rustic}\\
{গ্রাম্য}\\
{(adjective): characteristic of rural life; awkwardly simple and provincial\\The vacation cabin had no electricity and no indoor plumbing, but despite these inconveniences, Nigel\\adored its rustic charm.\\}\\
{sanction}\\
{অনুমোদন  করা}\\
{(verb): give authority or permission to\\The authorities have sanctioned the use of the wilderness reserve for public use; many expect to see\\\\                                                                                 \\hikers an campers enjoying the park in the coming months.\\}\\
{sanction}\\
{অনুমোদন  করা}\\
{(noun): a legal penalty for a forbidden action\\International sanctions have been placed on certain shipping lanes that were thought to be involved in\\human trafficking.\\}\\
{scrupulous}\\
{বিবেকী,,}\\
{(adjective): characterized by extreme care and great effort\\Because of his scrupulous nature, Mary put him in charge of numbering and cataloging the entire\\collection of rare stamps.\\}\\
{scrupulous}\\
{বিবেকী,,}\\
{(adjective): having a sense of right and wrong; principled\\Everyone trusted what he said and followed his example because he was scrupulous and honest.\\}\\
{soporific}\\
{ঘুমপাড়ানি,,}\\
{(adjective): inducing mental lethargy; sleep inducing\\Although the professor is brilliant, his bland monotone gives his lectures a soporific effect.\\}\\
{specious}\\
{ন্যায্য,,}\\
{(adjective): based on pretense; deceptively pleasing\\Almost every image on TV is specious and not to be trusted.\\}\\
{specious}\\
{ন্যায্য,,}\\
{(adjective): plausible but false\\He made a career out of specious arguments and fictional lab results, but lost his job and reputation\\when his lies were exposed by an article in The New York Times.\\}\\
{sporadic}\\
{বিচ্ছিন্ন}\\
{(adjective): recurring in scattered and irregular or unpredictable instances\\The signals were at first sporadic, but now we detect a clear, consistent pattern of electromagnetic\\radiation eminating from deep space.\\}\\
{spurious}\\
{কৃত্রিম,,}\\
{(adjective): plausible but false\\When listening to a politician speak, it is hard to distinguish the spurious claims from the authentic ones.\\}\\
{staunch}\\
{বায়ুরোধী,,}\\
{(adjective): firm and dependable especially in loyalty\\No longer a staunch supporter of the movement, Todd now will openly question whether its goals are\\worthwhile.\\}\\
{stringent}\\
{কঠোর}\\
{(adjective): demanding strict attention to rules and procedures\\Most of the students disliked the teacher because of his stringent homework policy, but many students\\would later thank him for demanding so much from them.\\}\\
{subsume}\\
{অন্তর্ভূত করা,,}\\
{(verb): contain or include\\The rogue wave quickly subsumed the pier and boardwalk, destroying everything in its path.\\}\\
{subsume}\\
{অন্তর্ভূত করা,,}\\
{(verb): consider (an instance of something) as part of a general rule or principle\\Don Quixote of La Mancha subsumes all other modern novels, demonstrating modern literary devices\\\\                                                                             \\and predating even the idea of a postmodern, metanarrative.\\}\\
{subversive}\\
{ক্ষতিকর}\\
{(adjective): in opposition to a civil authority or government\\The ruling political party has begun a campaign to shut down subversive websites that it deems as a\\threat to "national safety."\\}\\
{sullen}\\
{অন্ধকারাচ্ছন্ন,,}\\
{(adjective): showing a brooding ill humor\\Herbert took board games too seriously, often appearing sullen after losing.\\}\\
{superfluous}\\
{অপর্যাপ্ত,,}\\
{(adjective): serving no useful purpose\\How can we hope to stay open if we don't eliminate all superfluous spending, like catered meetings and\\free acupucture Tuesday?\\}\\
{superfluous}\\
{অপর্যাপ্ত,,}\\
{(adjective): more than is needed, desired, or required\\Everything in this closet is superfluous and can be given to charity.\\}\\
{supplant}\\
{উচ্ছিন্ন করা,,}\\
{(verb): take the place or move into the position of\\For many, a cell phone has supplanted a traditional phone; in fact, most 20-somethings don't even have\\a traditional phone anymore.\\}\\
{sycophant}\\
{পরগাছা,,}\\
{(noun): a person who tries to please someone in order to gain a personal advantage\\The CEO was unaware of the damaging consequences of his choices, largely because he surrounded\\himself with sycophants who would never dare criticize him.\\}\\
{taciturn}\\
{অল্পভাষী,,}\\
{(adjective): habitually reserved and uncommunicative\\While the CEO enthusiastically shares his plans and agenda with all who will listen, the CFO is far more\\taciturn, rarely revealing his perspective.\\}\\
{tantamount}\\
{সমপরিমাণ,,}\\
{(adjective): being essentially equal to something\\In many situations, remaining silent is tantamount to admitting guilt, so speak to prove your innocence.\\}\\
{temperance}\\
{মিতাচার,,}\\
{(noun): the trait of avoiding excesses\\Welles wasn't known for his temperance--he usually ate enough for two and drank enough for three.\\}\\
{tempered}\\
{নির্দিষ্ট মনোভাব বা মেজাজসম্পন্ন,,}\\
{(adjective): moderated in effect\\The wide-eyed optimism of her youth was now tempered after she had worked many years in the\\criminal justice system.\\}\\
{tenacious}\\
{অনমনীয়,,}\\
{(adjective): stubbornly unyielding\\Even the most tenacious advocates for gun ownership must admit some of the dangers that firearms\\present.\\}\\
{timorous}\\
{ভীরু,,}\\
{(adjective): timid by nature or revealing fear and nervousness\\\\                                                                               \\Since this was her first time debating on stage and before an audience, Di's voice was timorous and quiet\\for the first 10 minutes.\\}\\
{torpor}\\
{অস্পষ্টতা,,}\\
{(noun): inactivity resulting from lethargy and lack of vigor or energy\\After work, I was expecting my colleagues to be enthusiastic about the outing, but I found them in a\\state of complete torpor.\\}\\
{tortuous}\\
{অসরল,,}\\
{(adjective): marked by repeated turns and bends; not straightforward\\Because the logic behind McMahon's side of the debate was so tortuous, his audience came out either\\completely confused or, worse, feeling they'd been tricked.\\}\\
{tractable}\\
{বাধ্য,,}\\
{(adjective): readily reacting to suggestions and influences; easily managed (controlled or\\taught or molded)\\Compared to middle school students, who have an untamed wildness about them, high school students\\are somewhat more tractable.\\}\\
{transient}\\
{অস্থায়ী,,}\\
{(adjective): lasting a very short time\\The unpredictable and transient nature of deja vu makes it a very difficult phenomenon to study\\properly.\\}\\
{travesty}\\
{হাস্যকর অনুকরণ,,}\\
{(noun): an absurd presentation of something; a mockery\\What I expected to be an intelligent, nuanced historical documentary turned out to be a poorly-produced\\travesty of the form.\\}\\
{treacherous}\\
{বিশ্বাসঘাতী}\\
{(adjective): tending to betray\\Even though Jesse James was an outlaw, his killer, Robert Ford, is remembered more for his treacherous\\actions than for eliminating a criminal and murder.\\}\\
{treacherous}\\
{বিশ্বাসঘাতী}\\
{(adjective): dangerously unstable and unpredictable\\The bridge built from twine and vine is treacherous to walk across, and so I think I will stay put right\\here.\\}\\
{trite}\\
{মামুলি,,}\\
{(adjective): repeated too often; overfamiliar through overuse\\Many style guides recommend not using idioms in writing because these trite expressions are\\uninteresting and show a lack of imagination on the part of the writer.\\}\\
{truncate}\\
{অগ্রভাগ ছাঁটিয়া দেত্তয়া,,}\\
{(verb): reduce the length of something\\The soccer game was truncated when the monsoon rain began to fall.\\}\\
{undermine}\\
{গোপনে ক্ষতিসাধন করা,,}\\
{(adjective): to weaken (usually paired with an abstract term)\\The student undermined the teacher's authority by questioning the teacher's judgment on numerous\\occasions.\\\\                                                                               \\}\\
{underscore}\\
{আন্ডারস্কোর}\\
{(verb): give extra weight to (a communication)\\While the hiking instructor agreed that carrying a first aid kit could be a good idea under certain\\circumstances, he underscored the importance of carrying enough water.\\}\\
{unequivocal}\\
{দ্ব্যর্থহীন,,}\\
{(adjective): admitting of no doubt or misunderstanding; having only one meaning or\\interpretation and leading to only one conclusion\\The President's first statement on the subject was vague and open to competing interpretations, so when\\he spoke to Congress about the same subject later, he was cafeful to make his position completely\\unequivocal.\\}\\
{unscrupulous}\\
{বিবেকহীন,,}\\
{(adjective): without scruples or principles\\In the courtroom, the lawyer was unscrupulous, using every manner of deceit and manipulation to secure\\a victory for himself.\\}\\
{upbraid}\\
{ভর্ত্সনা করা,,}\\
{(verb): to reproach; to scold\\Bob took a risk walking into the "Students Barbershop"--in the end he had to upbraid the apparently\\drunk barber for giving him an uneven bowl cut.\\}\\
{vacillate}\\
{স্থির না থাকা,,}\\
{(verb): be undecided about something; waver between conflicting positions or courses of\\action\\Some students vacillate between schools when deciding which to attend, while others focus only on one\\school.\\This word has other definitions but this is the most important one for the GRE\\}\\
{vehement}\\
{প্রচণ্ড,,}\\
{(adjective): marked by extreme intensity of emotions or convictions\\While the other employees responded to the bad news in a measured way, Andrew responded in a\\vehement manner, tipping over his desk and shouting at the top of his lungs.\\}\\
{venality}\\
{ঘুষ প্রদান,,}\\
{(noun): the condition of being susceptible to bribes or corruption\\Even some of the most sacrosanct sporting events are not immune to venality, as many of the officials\\have received substantial bribes to make biased calls.\\}\\
{venerate}\\
{শ্রদ্ধা করা,,}\\
{(verb): to respect deeply\\The professor, despite his soporific lectures, was venerated amongst his colleagues, publishing more\\papers yearly than all of his peers combined.\\}\\
{veracious}\\
{সত্যবাদী,,}\\
{(adjective): truthful\\While we elect our leaders in the hope that every word they speak will be veracious, history has shown\\that such a hope is naive.\\}\\
{vilify}\\
{নিন্দা করা,,}\\
{(verb): spread negative information about\\Todd was noble after the divorce, choosing to say only complimentary things about Barbara, but Barbara\\did not hesitate to vilify Todd.\\\\                                                                             \\}\\
{vindicate}\\
{খাড়া করা,,}\\
{(verb): to clear of accusation, blame, suspicion, or doubt with supporting arguments or proof\\Even seven Tour de France wins cannot vindicate Lance Armstrong in the eyes of the public--that the\\athlete used performance enhancing drugs invalidates all those wins.\\}\\
{vociferous}\\
{গলাবাজ}\\
{(adjective): conspicuously and offensively loud; given to vehement outcry\\In giving Marcia a particular vociferous response, Paul caused people at every other table in the\\restaurant to turn around an look at them angrily.\\}\\
{volubility}\\
{শব্দব্যবহার,,}\\
{(noun): the quality of talking or writing easily and continuously\\The professor's volubility knows no bounds; he could talk through a hurricane and elaborate a point from\\one St. Patrick's Day to the next.\\}\\
{wanting}\\
{ক্রটিপূর্ণ}\\
{(adjective): lacking\\She did not think her vocabulary was wanting, yet there were so many words that inevitably she found a\\few she didn't know.\\}\\
{winsome}\\
{মনোহর}\\
{(adjective): charming in a childlike or naive way\\She was winsome by nature, and many people were drawn to this free and playful spirit.\\\\}\\
\section{BASIC WORDS}
{aboveboard}\\
{অকপট,,}\\
{(adjective): open and honest\\The mayor, despite his avuncular face plastered about the city, was hardly aboveboard -- some\\concluded that it was his ingratiating smile that allowed him to engage in corrupt behavior and get away\\with it.\\}\\
{abysmal}\\
{অন্তহীন}\\
{(adjective): extremely bad\\Coach Ramsey took his newest player off the field after watching a few painful minutes of her abysmal\\performance.\\}\\
{acme}\\
{চরম উন্নতি}\\
{(noun): the highest point of achievement\\The new Cessna airplanes will be the acme of comfort, offering reclining seats and ample legroom.\\}\\
{advocate}\\
{ওকালতি করা}\\
{(verb): speak, plead, or argue in favor of\\While the senator privately approved of gay marriage, he was unwilling to advocate for the cause in a\\public venue.\\This word has other definitions but this is the most important one for the GRE\\}\\
{advocate}\\
{ওকালতি করা}\\
{(noun): a person who pleads for a cause or propounds an idea\\Martin Luther King Jr. was a tireless advocate for the rights of African-Americans in the United States.Â\\This word has other definitions but this is the most important one for the GRE\\}\\
{affable}\\
{অমায়িক}\\
{(adjective): likeable; easy to talk to\\For all his surface affability, Marco was remarkably glum when he wasn't around other people.\\}\\
{affluent}\\
{সমৃদ্ধিশালী,,}\\
{(adjective): wealthy\\The center of the city had sadly become a pit of penury, while, only five miles away, multi-million dollar\\homes spoke of affluence.\\}\\
{altruism}\\
{পরার্থবাদ}\\
{(noun): the quality of unselfish concern for the welfare of others\\Albert Schweitzer spent most of his life doing missionary work as a doctor in Africa, seeking no reward,\\apparently motivated only by altruism.\\}\\
{amiable}\\
{বন্ধুভাবাপন্ন,,}\\
{(adjective): friendly\\Amy's name was very apt: she was so amiable that she was twice voted class president.\\}\\
{amply}\\
{পর্যাপ্তভাবে,,}\\
{(adverb): more than is adequate\\The boat was amply supplied for its year at sea--no man would go hungry or thirst.\\}\\
{amuck}\\
{ক্ষিপ্তবৎ,,}\\
{(adverb): in a frenzied or uncontrolled state\\Wherever the bowl haircut teen-idol went, his legions of screaming fans ran through the streets amuck,\\hoping for a glance at his boyish face.\\}\\
{analogous}\\
{অনুরূপ}\\
{(adjective): similar in some respects but otherwise different\\\\                                                                              \\In many ways, the Internet's transformative effect on society has been analogous to that of the printing\\press.\\}\\
{animosity}\\
{শত্রুতা,,}\\
{(noun): intense hostility\\The governor's animosity toward his rival was only inflamed when the latter spread false lies regarding\\the governor's first term.\\}\\
{antedated}\\
{বাসী,,}\\
{(verb): precede in time\\Harry was so unknowledgable that he was unaware the Egyptian pharaohs antedated the American\\Revolution.\\}\\
{antiquated}\\
{মান্ধাত্যার আমলের}\\
{(adjective): old-fashioned; belonging to an earlier period in time\\Aunt Betty had antiquated notions about marriage, believing that a man should court a woman for at\\least a year before receiving a kiss.\\}\\
{apex}\\
{চূড়া}\\
{(noun): the highest point\\The Ivy League is considered the apex of the secondary education system.\\}\\
{aphorism}\\
{সূত্র,,}\\
{(noun): a short instructive saying about a general truth\\Nietzsche was known for using aphorisms, sometimes encapsulating a complex philosophical thought in\\a mere sentence.\\}\\
{aphoristic}\\
{প্রবচনাত্মক,,}\\
{(adjective): something that is a concise and instructive of a general truth or principle\\Sometimes I can't stand Nathan because he tries to impress everyone by being aphoristic, but he just\\states the obvious.\\}\\
{appreciable}\\
{উপলব্ধিজনক,,}\\
{(adjective): large enough to be noticed (usu. refers to an amount)\\There is an appreciable difference between those who say they can get the job done and those who\\actually get the job done.\\}\\
{apprehension}\\
{আশঙ্কা,,}\\
{(noun): fearful expectation\\Test day can be one of pure apprehension, as many students worry about their test scores.\\}\\
{archaic}\\
{প্রাচীন,,}\\
{(adjective): so old as to appear to belong to a different period\\Hoping to sound intelligent, Mary spoke in archaic English that was right out of Jane Austen's Pride and\\Prejudice--needless to say, she didn't have many friends.\\}\\
{ascendancy}\\
{উদয়,,}\\
{(noun): the state that exists when one person or group has power over another\\The ascendancy of the Carlsbad water polo team is clear--they have a decade of championships behind\\them.\\}\\
{ascribe}\\
{আরোপ করা}\\
{(verb): attribute or credit to\\History ascribes The Odyssey and The Illiad to Homer, but scholars now debate whether he was a\\\\                                                                               \\historical figure or a fictitious name.\\}\\
{assail}\\
{প্রাণপণ চেষ্টা করা,,}\\
{(verb): attack in speech or writing\\In the weekly paper, the editor assailed the governor for wasting hundreds of thousands of dollars in\\public projects that quickly failed.\\}\\
{assuage}\\
{প্রশমিত করা,,}\\
{(verb): make something intense less severe\\Her fear that the new college would be filled with unknown faces was assuaged when she recognized her\\childhood friend standing in line.\\}\\
{augment}\\
{বৃদ্ধি,,}\\
{(verb): enlarge or increase; improve\\Ideally, the restaurant's augmented menu will expand its clientele and increase its profits.\\}\\
{autonomously}\\
{স্বয়ংক্রিয়,,}\\
{(adverb): In an autonomous or self-governing manner.\\Many of the factory workers are worried about being replaced by machines and computers that will\\work completely autonomously.\\}\\
{avarice}\\
{অর্থলিপ্সা,,}\\
{(noun): greed (one of the seven deadly sins)\\The Spanish conquistadors were known for their avarice, plundering Incan land and stealing Incan gold.\\}\\
{avert}\\
{প্রতিহত করা,,}\\
{(verb): turn away\\Afraid to see the aftermath of the car crash, I averted my eyes as we drove by.\\}\\
{avert}\\
{প্রতিহত করা,,}\\
{(verb): ward off or prevent\\The struggling videogame company put all of its finances into one final, desperate project to avert\\bankrupcy.\\}\\
{avid}\\
{ক্ষুধিত,,}\\
{(adjective): marked by active interest and enthusiasm\\Martin is an avid birdwatcher, often taking long hikes into remote mountains to see some rare eagle.\\}\\
{badger}\\
{ব্যাজার,,}\\
{(verb): to pester\\Badgered by his parents to find a job, the 30-year-old loafer instead joined a gang of itinerant musicians.\\This word has other definitions but this is the most important one for the GRE\\}\\
{balk}\\
{কড়িকাঠ,,}\\
{(verb): refuse to comply\\The students were willing to clean up the broken glass, but when the teacher asked them to mop the\\entire floor, they balked, citing reasons why they needed to leave.\\}\\
{banish}\\
{নির্বাসিত করা,,}\\
{(verb): expel from a community, residence, or location; drive away\\The most difficult part of the fast was banishing thoughts of food.\\This word has other definitions but this is the most important one for the GRE\\}\\
{beatific}\\
{স্বর্গসুখদায়ক}\\
{(adjective): blissfully happy\\Often we imagine all monks to wear the beatific smile of the Buddha, but, like any of us, a monk can\\have a bad day and not look very happy.\\\\                                                                               \\}\\
{becoming}\\
{মানানসই,,}\\
{(adjective): appropriate, and matches nicely\\Her dress was becoming and made her look even more beautiful.\\This word has other definitions but this is the most important one for the GRE\\}\\
{begrudge}\\
{বিক্ষুব্ধ হত্তয়া,,}\\
{(verb): to envy someone for possessing or enjoying something\\Sitting all alone in his room, Harvey begrudged the happiness of the other children playing outside his\\window.\\}\\
{begrudge}\\
{বিক্ষুব্ধ হত্তয়া,,}\\
{(verb): to give reluctantly\\We never begrudge money spent on ourselves.\\}\\
{behooves}\\
{সংশিস্নষ্টরা,,}\\
{(verb): to be one's duty or obligation\\The teacher looked down at the student and said, "It would behoove you to be in class on time and\\complete your homework, so that you don't repeat freshman English for a third straight year."\\}\\
{belittle}\\
{খর্ব করা,,}\\
{(verb): lessen the importance, dignity, or reputation of\\A good teacher will never belittle his students, but will instead empower them.\\}\\
{bellicose}\\
{মারমুখো,,}\\
{(adjective): warlike; inclined to quarrel\\Known for their bellicose ways, the Spartans were once the most feared people from Peloponnesus to\\Persia.\\}\\
{benign}\\
{কৃপালু,,}\\
{(adjective): kind\\I remember my grandfather's face was wrinkled, benign, and calm.\\}\\
{benign}\\
{কৃপালু,,}\\
{(adjective): (medicine) not dangerous to health; not recurrent or progressive\\The tumor located in your ear lobe seems to be benign and should not cause you any trouble.\\}\\
{besiege}\\
{বেষ্টন করা,,}\\
{(verb): harass, as with questions or requests; cause to feel distressed or worried\\After discovering a priceless artifact in her backyard, Jane was besieged by phone calls, emails, and\\reporters all trying to buy, hold or see the rare piece of history.\\}\\
{besmirch}\\
{বদনাম করা}\\
{(verb): damage the good name and reputation of someone\\The prince's distasteful choice of words besmirched not only his own name, but the reputation of the\\entire royal family.\\}\\
{bleak}\\
{নিরানন্দ,,}\\
{(adjective): having a depressing or gloomy outlook\\Unremitting overcast skies tend to lead people to create bleak literature and lugubrious music --\\compare England's band Radiohead to any band from Southern California.\\}\\
{boon}\\
{অনুগ্রহ}\\
{(noun): a desirable state\\Modern technology has been a boon to the travel industry.\\\\                                                                                \\}\\
{boon}\\
{অনুগ্রহ}\\
{(adjective): very close and convivial\\He was a boon companion to many, and will be sadly missed.\\}\\
{boorish}\\
{চাষাড়ে,,}\\
{(adjective): ill-mannered and coarse or contemptible in behavior or appearance\\Bukowski was known for being a boorish drunk and alienating close friends and family.\\}\\
{brusquely}\\
{রেল স্টেশনে,,}\\
{(adverb): in a blunt, direct manner\\Not one for social pleasantries, the Chief of Staff would brusquely ask his subordinates anything he\\wanted, even coffee.\\}\\
{buck}\\
{ছাগ,,}\\
{(verb): resist\\The profits at our firm bucked the general downturn that effected the real estate industry.\\This word has other definitions but this is the most important one for the GRE\\}\\
{buttress}\\
{মদত দেওয়া}\\
{(verb): make stronger or defensible\\China's economy has been buttressed by a global demand for the electronic parts the country\\manufactures.\\}\\
{cadaverous}\\
{বিকৃত}\\
{(adjective): emaciated; gaunt\\Some actors take challenging roles in which they have to lose so much weight that they appear\\cadaverous.\\}\\
{candid}\\
{অকপট,,}\\
{(adjective): a straightforward and honest look at something\\Even with a perfect stranger, Charles was always candid and would rarely hold anything back.\\}\\
{candidness}\\
{সরলভাব,,}\\
{(noun): the quality of being honest and straightforward in attitude and speech\\Although I was unhappy that the relationship ended, I appreciated her candidness about why she was\\ready to move on from the relationship.\\}\\
{cardinal}\\
{অঙ্কবাচক,,}\\
{(adjective): of primary importance; fundamental\\Most cultures consider gambling a cardinal sin and thus have outlawed its practice.\\This word has other definitions but this is the most important one for the GRE\\}\\
{carping}\\
{খুঁতখুঁত,,}\\
{(adjective): persistently petty and unjustified criticism\\What seemed like incessant nagging and carping about my behavior from my mother turned out to be\\wise and useful advice that has served me well.\\}\\
{catalyst}\\
{অনুঘটক}\\
{(noun): something that speeds up an event\\Rosa Park's refusal to give up her bus seat acted as a catalyst for the Civil Rights Movement, setting into\\motion historic changes for African-Americans.\\}\\
{cavalier}\\
{বীর}\\
{(adjective): given to haughty disregard of others\\Percy dismissed the issue with a cavalier wave of his hand.\\}\\
{censor}\\
{সমালোচক,,}\\
{(verb): to examine and remove objectionable material\\\\                                                                                \\Every fall, high school English teachers are inundated by requests to censor their curriculum by removing\\The Catcher in the Rye and Scarlet Letter from their reading lists.\\}\\
{cerebral}\\
{মস্তিষ্ক - সংক্রান্ত,,}\\
{(adjective): involving intelligence rather than emotions or instinct\\A cerebral analysis of most pop music finds it to be simple and childish, but that ignores the point--the\\music's effect on the listener.\\This word has other definitions but this is the most important one for the GRE\\}\\
{champion}\\
{বিজয়ী}\\
{(verb): protect or fight for as a champion\\Martin Luther King Jr. championed civil rights fiercely throughout his short life.\\This word has other definitions but this is the most important one for the GRE\\}\\
{chauvinist}\\
{উগ্র জাতীয়তাবাদী,,}\\
{(noun): a person who believes in the superiority of their group\\The chauvinist lives on both sides of the political spectrum, outright shunning anybody whose ideas are\\not consistent with his own.\\}\\
{check}\\
{পরীক্ষা}\\
{(verb): to limit (usually modifying the growth of something)\\Deserted for six months, the property began to look more like a jungle and less like a residence--weeds\\grew unchecked in the front yard\\This word has other definitions but this is the most important one for the GRE\\}\\
{check}\\
{পরীক্ষা}\\
{(noun): the condition of being held back or limited\\When government abuses are not kept in check, that government is likely to become autocratic.\\This word has other definitions but this is the most important one for the GRE\\}\\
{checkered}\\
{চেক-কাটা}\\
{(adjective): one that is marked by disreputable happenings\\One by one, the presidential candidates dropped out of the race, their respective checkered pasts-- from\\embezzlement to infidelity--sabotaging their campaigns.\\This word has other definitions but this is the most important one for the GRE\\}\\
{chivalrous}\\
{সাহসী,,}\\
{(adjective): being attentive to women like an ideal knight\\Marco's chivalrous ways, like opening doors and pulling out chairs, was much appreciated by his date.\\}\\
{clemency}\\
{ক্ষমাশীলতা,,}\\
{(noun): leniency and compassion shown toward offenders by a person or agency charged with\\administering justice\\In the final moments of the trial, during his closing speech, Phillips was nearly begging the judge for\\clemency.\\}\\
{coalesce}\\
{একাত্ম হওয়া}\\
{(verb): fuse or cause to grow together\\Over time, the various tribes coalesced into a single common culture with one universal language.\\}\\
{cogent}\\
{প্রবল,,}\\
{(adjective): clear and persuasive\\A cogent argument will change the minds of even the most skeptical audience.\\}\\
{cohesive}\\
{সংযোজক,,}\\
{(adjective): well integrated, forming a united whole\\A well-written, cohesive essay will keep on topic at all times, never losing sight of the main argument.\\}\\
{collusion}\\
{সাজশ,,}\\
{(noun): agreement on a secret plot\\\\                                                                                \\Many have argued that Lee Harvey Oswald, JFK's assassin, was in collusion with other criminals; others\\maintain that Oswald was a lone gunman.\\}\\
{colossal}\\
{প্রকাণ্ড,,}\\
{(adjective): so great in size or force or extent as to elicit awe\\Few appreciate the colossal scale of the sun: if hollow, it could contain a million Earths.\\}\\
{commendable}\\
{প্রশংসনীয়,,}\\
{(adjective): worthy of high praise\\The efforts of the firefighters running into the burning building were commendable.\\}\\
{complacent}\\
{প্রসন্ন,,}\\
{(adjective): contented to a fault with oneself or one's actions\\After the water polo team won their sixth championship, they became complacent and didn't even make\\it to the playoffs the next year.\\}\\
{complementary}\\
{পরিপূরক,,}\\
{(adjective): enhancing each other's qualities (for two things or more).\\The head waiter was careful to tell the amateur diners that red wine was complementary with beef, each\\bringing out subtle taste notes in the other.\\}\\
{compound}\\
{যৌগ}\\
{(verb): make more intense, stronger, or more marked\\Her headache was compounded by the construction crew outside, which had six jackhammers going at\\the same time.\\This word has other definitions but this is the most important one for the GRE\\}\\
{conducive}\\
{সহায়ক,,}\\
{(adjective): making a situation or outcome more likely to happen\\Studying in a quiet room is conducive to learning; studying in a noisy environment makes learning more\\difficult.\\}\\
{conniving}\\
{উপেক্ষা করা,,}\\
{(verb): taking part in immoral and unethical plots\\The queen was so conniving that, with the help of the prince, she tried to overthrow the king.\\}\\
{consecrate}\\
{পবিত্র,,}\\
{(verb): to make holy or set apart for a high purpose\\At the church of Notre Dame in France, the new High Altar was consecrated in 1182.\\}\\
{constraint}\\
{বাধ্যতা,,}\\
{(noun): something that limits or restricts\\He found pop music a constraint on his ability to learn and preferred to listen to classical musical while\\studying.\\}\\
{consummate}\\
{সুসম্পূর্ণ,,}\\
{(adjective): having or revealing supreme mastery or skill\\Tyler was the consummate musician: he was able to play the guitar, harmonica, and the drum at the\\same time.\\This word has other definitions but this is the most important one for the GRE\\}\\
{consummate}\\
{সুসম্পূর্ণ,,}\\
{(verb): to make perfect and complete in every respect\\The restoration of the ancient church was only consummated after a twenty years of labor.\\This word has other definitions but this is the most important one for the GRE\\}\\
{contemptuous}\\
{ঘৃণাপূর্ণ,,}\\
{(adjective): scornful, looking down at others with a sneering attitude\\\\                                                                                 \\Always on the forefront of fashion, Vanessa looked contemptuously at anyone wearing dated clothing.\\}\\
{contrite}\\
{অনুতপ্ত,,}\\
{(adjective): to be remorseful\\Though he stole his little sister's licorice stick with malevolent glee, Chucky soon became contrite when\\his sister wouldn't stop crying.\\}\\
{conundrum}\\
{প্রহেলিকা,,}\\
{(noun): a difficult problem\\Computers have helped solve some of the mathematical conundrums which have puzzled man for many\\centuries.\\}\\
{convivial}\\
{পার্বণ,,}\\
{(adjective): describing a lively atmosphere\\The wedding reception was convivial; friends who hadn't seen each other for ages drank and ate\\together before heading to the dance floor.\\}\\
{convoluted}\\
{সংবর্ত,,}\\
{(adjective): highly complex or intricate\\Instead of solving the math problem in three simple steps, Kumar used a convoluted solution requiring\\fifteen steps.\\}\\
{cornucopia}\\
{প্রাচুর্যে পরিপূর্ণ ভাণ্ডার,,}\\
{(noun): an abundant supply of something good\\The International Food Expo was a cornucopia of culinary delights: gourmet foods from every continent\\were under one roof.\\}\\
{corroborate}\\
{সমর্থিত হওয়া}\\
{(verb): to confirm or lend support to (usually an idea or claim)\\Her claim that frog populations were falling precipitously in Central America was corroborated by locals,\\who reported that many species of frogs had seemingly vanished overnight.\\}\\
{cosmopolitan}\\
{বিশ্বজনীন}\\
{(adjective): comprising many cultures; global in reach and outlook\\There are few cities in the world as diverse and cosmopolitan as New York.\\}\\
{credence}\\
{প্রত্যয়,,}\\
{(noun): belief in something\\He placed no credence in psychics, claiming that they offered no special powers beyond the ability to\\make people part with their money.\\}\\
{creditable}\\
{সম্মানজনক,,}\\
{(adjective): deserving of praise but not that amazing\\Critics agreed the movie was creditable, but few gave it more than three out of five stars.\\}\\
{credulity}\\
{বিশ্বাসপ্রবণতা,,}\\
{(noun): tendency to believe readily\\Virginia's wide-eyed credulity as a five-year old was replaced by suspicion after she learned that Santa\\Claus didn't really exist.\\}\\
{cumbersome}\\
{কষ্টকর,,}\\
{(adjective): difficult to handle or use especially because of size or weight\\Only ten years ago, being an avid reader and a traveler meant carrying a cumbersome backpack stuffed\\with books--these days we need only an e-reader.\\\\                                                                                \\}\\
{dearth}\\
{অভাব, অনটন, আকাল}\\
{(noun): a lack or shortage\\I am surprised by the dearth of fast food chains; this is America and I assumed they were on every street.\\}\\
{debase}\\
{অপকৃষ্ট করা,,}\\
{(verb): reduce the quality or value of something\\The third-rate script so debased the film that not even the flawless acting could save it from being a flop.\\}\\
{debunk}\\
{অনাবৃত করা,,}\\
{(verb): expose as false ideas and claims, especially while ridiculing\\Richard Dawkins tries to debunk religious belief, but his ridicule tends to push people away from his\\points rather than convince them.\\}\\
{decimation}\\
{ধ্বংসকরণ,,}\\
{(noun): destroying or killing a large part of the population\\The decimation after atomic bombs were dropped on Hiroshima and Nagasaki is incomprehensible.\\}\\
{degrade}\\
{অধ: পতন করা,,}\\
{(verb): reduce in worth or character, usually verbally\\Jesse had mockingly pointed out all of Nancy's faults in front of their friends, publicly degrading the poor\\girl.\\}\\
{delegate}\\
{প্রতিনিধি}\\
{(verb): give an assignment to (a person)\\Since the senior manager had to go on many international business trips, she was forced to delegate\\many of her responsibilities to two lower-level managers.\\This word has other definitions but this is the most important one for the GRE\\}\\
{deliberate}\\
{সুচিন্তিত}\\
{(verb): think about carefully; weigh the pros and cons of an issue\\Emergency situations such as this call for immediate action and leave no room to deliberate over\\options.\\This word has other definitions but this is the most important one for the GRE\\}\\
{demean}\\
{হীন করা,,}\\
{(verb): to insult; to cause someone to lose dignity or respect\\At first the soccer players bantered back and forth, but as soon as one of the players became demeaning,\\calling the other's mother a water buffalo, the ref whipped out a red card.\\}\\
{demure}\\
{অচপল,,}\\
{(adjective): to be modest and shy\\The portrait of her in a simple white blouse was sweet and demure.\\}\\
{deride}\\
{উপহাস করা,,}\\
{(verb): treat or speak of with contempt\\The nun derided the students for trying to sneak insects and worms into the classroom.\\}\\
{derisive}\\
{উপহাস্য,,}\\
{(adjective): abusing vocally; expressing contempt or ridicule\\I was surprised by her derisive tone; usually, she is sweet, soft spoken, and congenial.\\}\\
{derogative}\\
{হানিকর,,}\\
{(adjective): expressed as worthless or in negative terms\\Never before have we seen a debate between two political candidates that was so derogative and filthy.\\}\\
{desecrate}\\
{অপবিত্র করা,,}\\
{(verb): to willfully violate or destroy a sacred place\\\\                                                                                \\After desecrating the pharaoh's tomb, the archaeologist soon fell victim to a horrible illness.\\}\\
{destitute}\\
{চরম অভাবগ্রস্ত}\\
{(adjective): poor enough to need help from others\\Jean Valjean, is at first destitute, but through the grace of a priest, he makes something of his life.\\}\\
{destitute}\\
{চরম অভাবগ্রস্ত}\\
{(adjective): completely wanting or lacking (usually "destitute of")\\Now that the mine is closed, the town is destitute of any economic activity.\\}\\
{deter}\\
{নিরস্ত করা,,}\\
{(verb): turn away from by persuasion\\His mother tried to deter him from joing the army, but he was too intoxicated with the idea of war to\\listen.\\}\\
{deter}\\
{নিরস্ত করা,,}\\
{(verb): try to prevent; show opposition to\\The government's primary job should invlove deterring paths to war, not finding ways to start them.\\}\\
{detrimental}\\
{ক্ষতিকারক,,}\\
{(adjective): (sometimes followed by "to") causing harm or injury\\Many know that smoking is detrimental to your health, but processed sugar in large quantities is equally\\bad.\\}\\
{devolve}\\
{নিচে গড়াইযা পড়া,,}\\
{(verb): pass on or delegate to another\\The company was full of managers known for devolving tasks to lower management, but never doing\\much work themselves.\\}\\
{devolve}\\
{নিচে গড়াইযা পড়া,,}\\
{(verb): grow worse (usually "devolve into")\\The dialogue between the two academics devolved into a downright bitter argument.\\}\\
{diabolical}\\
{নারকীয়}\\
{(adjective): to be extremely wicked like the devil\\The conspirators, willing to dispatch anyone who stood in their way, hatched a diabolical plan to take\\over the city.\\}\\
{differentiate}\\
{ভেদ করা,,}\\
{(verb): be a distinctive feature, attribute, or trait (sometimes in positive sense)\\Mozart's long melodic lines differentiate his compositions from other works of late 18th century music.\\}\\
{differentiate}\\
{ভেদ করা,,}\\
{(verb): evolve so as to lead to a new species or develop in a way most suited to the\\environment\\Animals on Madagascar differentiated from other similar animal species due to many years of isolation\\on the island.\\}\\
{dilapidated}\\
{জীর্ণ,,}\\
{(adjective): in terrible condition\\The main house has been restored but the gazebo is still dilapidated and unuseable.\\}\\
{diligent}\\
{পরিশ্রমী,,}\\
{(adjective): characterized by care and perseverance in carrying out tasks\\Michael was a diligent gardener, never leaving a leaf on the ground and regularly watering each plant.\\\\                                                                               \\}\\
{discord}\\
{অনৈক্য,,}\\
{(noun): lack of agreement or harmony\\Despite all their talented players, the team was filled with discord--some players refused to talk to\\others--and lost most of their games.\\}\\
{discreet}\\
{বিচক্ষণ}\\
{(adjective): careful to protect one's speech or actions in order to avoid offense or gain an\\advantage\\The professor thought that he was discreet, subtly wiping the stain off of his shirt, but as soon as he\\stepped off the podium a member of the audience pointed out the large ketchup stain.\\}\\
{discriminate}\\
{বৈষম্যমূলক আচরণ করা}\\
{(verb): recognize or perceive the difference\\Sarah couldn't discriminate between a good wine and a bad wine, so she avoided wine tastings.\\This word has other definitions but this is the most important one for the GRE\\}\\
{disenfranchise}\\
{ভাল থাকুন,,}\\
{(verb): deprive of voting rights\\The U.S. Constitution disenfranchised women until 1920 when they were given the right to vote.\\}\\
{disheartened}\\
{ভগ্নোত্সাহ,,}\\
{(adjective): made less hopeful or enthusiastic\\After the visiting team scored nine times, the home team's fans were disheartened, some leaving the\\game early.\\}\\
{disparate}\\
{অসম,,}\\
{(adjective): two things are fundamentally different\\With the advent of machines capable of looking inside the brain, fields as disparate as religion and\\biology have been brought together by scientists trying to understand what happens in the brain when\\people have a religious experience.\\}\\
{dispatch}\\
{প্রাণবধ,,}\\
{(noun): the property of being prompt and efficient\\She finished her thesis with dispatch, amazing her advisors who couldn't believe she hadn't written\\60 scholarly pages so quickly.\\This word has other definitions but this is the most important one for the GRE\\}\\
{dispatch}\\
{প্রাণবধ,,}\\
{(verb): dispose of rapidly and without delay and efficiently\\As soon as the angry peasants stormed the castle, they caught the king and swiftly dispatched him.\\This word has other definitions but this is the most important one for the GRE\\}\\
{docile}\\
{নিরীহ,,}\\
{(adjective): easily handled or managed; willing to be taught or led or supervised or directed\\Barnyard animals are considerably more docile than the wild animals.\\}\\
{dog}\\
{কুকুর}\\
{(verb): to pursue relentlessly; to hound\\Throughout his life, he was dogged by insecurities that inhibited personal growth.\\This word has other definitions but this is the most important one for the GRE\\}\\
{dupe}\\
{প্রতারিত ব্যক্তি,,}\\
{(verb): to trick or swindle\\Once again a get-rich-fast Internt scheme had duped Harold into submitting a \$5,000 check to a sham\\operation.\\}\\
{dupe}\\
{প্রতারিত ব্যক্তি,,}\\
{(noun): a person who is easily tricked or swindled\\\\                                                                                \\The charlatan mistook the crowd for a bunch of dupes, but the crowd was quickly on to him and decried\\his bald-faced attempt to bilk them.\\}\\
{eccentric}\\
{অদ্ভুতস্বভাব,,}\\
{(adjective): highly unconventional or usual (usually describes a person)\\Mozart was well-known for his eccentricities, often speaking words backward to confuse those around\\him.\\}\\
{egotist}\\
{আত্মশ্লাঘী,,}\\
{(noun): a conceited and self-centered person\\An egotist, Natasha had few friends because of her inability to talk about anything except her dream of\\becoming the next American Idol.\\}\\
{eke}\\
{অধিকন্তু,,}\\
{(verb): To live off meager resources, to scrape by\\Stranded in a cabin over the winter, Terry was able to eke out an existence on canned food.\\}\\
{elaborate}\\
{সম্প্রসারিত,,}\\
{(adjective): marked by complexity and richness of detail\\Thomas, on returning from Morocco, replaced his dirty gray carpet with an elaborate one he'd brought\\back with him.\\}\\
{elaborate}\\
{সম্প্রসারিত,,}\\
{(verb): explain in more detail\\Most high school physics teachers find themselves elaborating the same point over and over again, since\\many concepts confuse students.\\}\\
{elude}\\
{পালান,,}\\
{(verb): escape understanding\\Even a basic understanding of physics can elude most high schools students.\\}\\
{elusive}\\
{অধরা,,}\\
{(adjective): difficult to capture or difficult to remember\\Many first time skydivers say that describing the act of falling from the sky is elusive.\\}\\
{embellish}\\
{অলঙ্কৃত করা,,}\\
{(verb): make more attractive by adding ornament, colour, etc.; make more beautiful\\McCartney would write relatively straightforward lyrics, and Lennon would embellish them with puns\\and poetic images.\\}\\
{embroiled}\\
{বাধে,,}\\
{(adjective): involved in argument or contention\\These days we are never short of a D.C. politician embroiled in scandal--a welcome phenomenon for\\those who, having barely finished feasting on the sordid details of one imbroglio, can sink their teeth into\\a fresh one.\\}\\
{empathetic}\\
{সহানুভূতিশীল,,}\\
{(adjective): showing understanding and ready comprehension of other peoples' states and\\emotions\\Most discrimination and hatred is based on a lack of empathetic awareness of people that have the\\same aspitations and fears.\\}\\
{emulate}\\
{অনুকরণ করা}\\
{(verb): strive to equal or match, especially by imitating; compete with successfully\\\\                                                                               \\To really become fluent in a new language, emulate the speech patterns and intonation of people who\\speak the language.\\}\\
{endemic}\\
{স্থানীয়,,}\\
{(adjective): native; originating where it is found\\Irish cuisine makes great use of potatoes, but ironically, the potato is not endemic to Ireland.\\}\\
{enmity}\\
{শত্রুতা}\\
{(noun): a state of deep-seated ill-will\\Charles rude remark toward Sarah yesterday was due to his illness, not due to any real enmity toward\\Sarah.\\}\\
{entice}\\
{আকৃষ্ট করা}\\
{(verb): get someone to do something through (often false or exaggerated) promises\\Harold enticed his wife, Maude, to go on a vacation to Hawaii, with promises of luaus on the beach and\\all-you-can-eat seafood buffets.\\}\\
{enumerate}\\
{গনা,,}\\
{(verb): determine the number or amount of\\The survey enumerates the number of happy workers and the number of unhappy workers.\\}\\
{enumerate}\\
{গনা,,}\\
{(verb): specify individually, one by one\\I sat and listened as she enumerated all of the things she did not like about the past three months.\\}\\
{err}\\
{ভুল করা}\\
{(verb): to make an error\\He erred in thinking that "indigent" and "indignant" were synonyms.\\}\\
{errant}\\
{ভ্রমণরত,,}\\
{(adjective): to be wandering; not sticking to a circumscribed path\\Unlike his peers, who spent their hours studying in the library, Matthew preferred errant walks through\\the university campus.\\}\\
{erratic}\\
{অনিশ্চিত,,}\\
{(adjective): unpredictable; strange and unconventional\\It came as no surprise to pundits that the President's attempt at re-election floundered; even during his\\term, support for his policies was erratic, with an approval rating jumping anywhere from 30 to 60\\percent.\\}\\
{euphoria}\\
{রমরমা,,}\\
{(noun): a feeling of great (usually exaggerated) elation\\The euphoria of winning her first gold medal in the 100 meter dash overwhelmed Shelly-Ann Fraser and\\she wept tears of immense joy.\\}\\
{evasive}\\
{অসরল,,}\\
{(adjective): avoiding or escaping from difficulty or danger or commitment\\His responses were clearly evasive; he obviously did not want to take on any responsibility or any new\\work.\\}\\
{evasive}\\
{অসরল,,}\\
{(adjective): deliberately vague or ambiguous\\Every time I call the bank, I receive the same evasive answers about our mortgage and never get a clear\\response.\\\\                                                                                \\}\\
{evenhanded}\\
{নিরপেক্ষ,,}\\
{(adjective): without partiality\\Teachers often have trouble being evenhanded to all of their varied students.\\}\\
{exasperate}\\
{ক্রুদ্ধ করা,,}\\
{(verb): to irritate intensely\\As a child, I exasperated my mother with strings of never-ending questions.\\}\\
{excruciating}\\
{মর্মযন্ত্রণাপূর্ণ,,}\\
{(adjective): extremely painful\\After the boulder rolled a couple of feet, pinning my friend's arm, he experienced excruciating pain.\\}\\
{exemplify}\\
{উদাহরণ দ্বারা ব্যাখ্যা করা,,}\\
{(verb): be characteristic of\\Lincoln exemplified the best of not only America, but also the potential greatness that exists within each\\person.\\}\\
{exemplify}\\
{উদাহরণ দ্বারা ব্যাখ্যা করা,,}\\
{(verb): clarify by giving an example of\\Please present some case studies that exemplify the results that you claim in your paper.\\}\\
{exhort}\\
{উত্সাহিত করা,,}\\
{(verb): to strongly urge on; encourage\\Nelson's parents exhorted him to study medicine, urging him to choose a respectable profession;\\intransigent, Nelson left home to become a graffiti artist.\\}\\
{extenuating}\\
{প্রশমনকারী,,}\\
{(adjective): making less guilty or more forgivable\\The jury was hardly moved by the man's plea that his loneliness was an extenuating factor in his crime of\\dognapping a prized pooch.\\}\\
{facetious}\\
{সদানন্দ,,}\\
{(adjective): cleverly amusing in tone\\Facetious behavior will not be tolerated during sex eduation class; it's time for all of you to treat these\\matters like mature adults.\\}\\
{fawn}\\
{হরিণ-শিশু}\\
{(verb): try to gain favor by extreme flattery\\The media fawned over the handsome new CEO, praising his impeccable sense of style instead of asking\\more pointed questions.\\This word has other definitions but this is the most important one for the GRE\\}\\
{ferret}\\
{সন্ধানী,,}\\
{(verb): to search for something persistently\\Ever the resourceful lexicographer, Fenton was able to ferret out the word origin of highly obscure\\words.\\This word has other definitions but this is the most important one for the GRE\\}\\
{fete}\\
{উত্সব,,}\\
{(verb): to celebrate a person\\After World War II, war heroes were feted at first but quickly forgotten.\\}\\
{fickle}\\
{চঞ্চল}\\
{(adjective): liable to sudden unpredictable change, esp. in affections or attachments\\She was so fickle in her politics, it was hard to pinpoint her beliefs; one week she would embrace a side,\\and the next week she would denounce it.\\\\                                                                                \\}\\
{finagle}\\
{প্রতারণা করা,,}\\
{(verb): achieve something by means of trickery or devious methods\\Steven was able to finagle one of the last seats on the train by convincing the conductor that his torn\\stub was actually a valid ticket.\\}\\
{fledgling}\\
{জাতপক্ষ,,}\\
{(adjective): young and inexperienced; describing any new participant in some activity\\Murray has years of experience in family practice, but he is just a fledgling in surgery.\\}\\
{fleece}\\
{আচ্ছাদিত করা,,}\\
{(verb): to deceive\\Many people have been fleeced by Internet scams and never received their money back.\\This word has other definitions but this is the most important one for the GRE\\}\\
{flounder}\\
{হাক পাক করা}\\
{(verb): behave awkwardly; have difficulties\\Sylvia has excelled at advanced calculus, but ironically, when she has deal with taxes, she flounders.\\}\\
{flush}\\
{ফ্লাশ করা}\\
{(adjective): to be in abundance\\The exam's passage is flush with difficult words, words that you may have learned only yesterday.\\}\\
{foible}\\
{দুর্বলতা,,}\\
{(noun): a behavioral attribute that is distinctive and peculiar to an individual\\When their new roommate sat staring at an oak tree for an hour, Marcia thought it indicated a mental\\problem, but Jeff assured her it was a harmless foible.\\}\\
{foolhardy}\\
{হঠকারী,,}\\
{(adjective): marked by defiant disregard for danger or consequences\\The police regularly face dangerous situations, so for a police officer not to wear his bullet-proof vest is\\foolhardy.\\}\\
{forthright}\\
{অবিচল,,}\\
{(adjective): characterized by directness in manner or speech; without subtlety or evasion\\I did not expect the insurance agent to give us any straight answers, but I was pleasent surprised by how\\forthright he was.\\}\\
{futile}\\
{বৃথা,,}\\
{(adjective): producing no result or effect; unproductive of success\\I thought I could repair the car myself, but after two days of work with no success, I have to admit that\\my efforts were futile.\\}\\
{genial}\\
{সদয়,,}\\
{(adjective): agreeable, conducive to comfort\\Betty is a genial young woman: everyone she meets is put at ease by her elegance and grace.\\}\\
{genteel}\\
{বিনয়ী,,}\\
{(adjective): marked by refinement in taste and manners\\A live string quartet would provide a more genteel air to the wedding than would a folk singer.\\}\\
{glean}\\
{উঁচ্ছবৃত্তি,,}\\
{(verb): collect information bit by bit\\Herb has given us no formal statement about his background, but from various hints, I have gleaned that\\he grew up in difficult circumstances.\\\\                                                                               \\}\\
{glib}\\
{সাবলীল,,}\\
{(adjective): (of a person) speaking with ease but without sincerity\\I have found that the more glib the salesman, the worse the product.\\}\\
{goad}\\
{অঙ্কুশ,,}\\
{(verb): urge on with unpleasant comments\\Doug did not want to enter the race, but Jim, through a steady stream of taunts, goaded him into signing\\up for it.\\}\\
{grovel}\\
{উবুড় হয়ে থাকা,,}\\
{(verb): show submission or fear\\Every time Susan comes to the office, Frank grovels as if she were about to fire.\\}\\
{guffaw}\\
{অট্টহাসি,,}\\
{(verb): laugh boisterously\\Whenever the jester fell to the ground in mock pain, the king guffawed, exposing his yellow, fang-like\\teeth.\\}\\
{hamper}\\
{বাধা দেওয়া}\\
{(verb): prevent the progress or free movement of\\As the rain water began to collect in pools on the highway, it began to hamper the flow of traffic.\\This word has other definitions but this is the most important one for the GRE\\}\\
{hamstrung}\\
{}\\
{(verb): made ineffective or powerless\\The FBI has made so many restriction on the local police that they are absolutely hamstrung, unable to\\accomplish anything.\\}\\
{heyday}\\
{পূর্ণবিকাশ,,}\\
{(noun): the pinnacle or top of a time period or career\\During the heyday of Prohibition, bootlegging had become such a lucrative business that many who had\\been opposed to the 18th Amendment began to fear it would be repealed.\\}\\
{hodgepodge}\\
{খিচুড়ি,,}\\
{(noun): a confusing mixture or jumble\\Those in attendance represented a hodgepodge of the city's denizens: chimney sweepers could be seen\\sitting elbow to elbow with stockbrokers.\\}\\
{hound}\\
{জ্বালাতন করা,,}\\
{(verb): to pursue relentlessly\\An implacable foe of corruption, Eliot Ness hounded out graft in all forms--he even helped nab Al\\Capone.\\This word has other definitions but this is the most important one for the GRE\\}\\
{humdrum}\\
{একঘেয়ে,,}\\
{(adjective): dull and lacking excitement\\Having grown up in a humdrum suburb, Jacob relished life in New York City after moving.\\}\\
{illicit}\\
{অবৈধ,,}\\
{(adjective): contrary to or forbidden by law\\Though Al Capone was engaged in many illicit activities, he was finally arrested for income tax evasion, a\\relatively minor offense.\\}\\
{immaterial}\\
{অবাস্তব,,}\\
{(adjective): not relevant\\The judge found the defendant's comments immaterial to the trial, and summarily dismissed him from\\the witness stand.\\\\                                                                                \\}\\
{impeccable}\\
{নিষ্পাপ,,}\\
{(adjective): without fault or error\\He was impeccably dressed in the latest fashion without a single crease or stain.\\}\\
{impede}\\
{ব্যাহত করা,,}\\
{(verb): be a hindrance or obstacle to\\Since the police sergeant had to train the pair of new hires, progress in his own case was impeded.\\}\\
{impending}\\
{আসন্ন,,}\\
{(adjective): close in time; about to occur\\The impending doom of our world has been a discussed and debated for 2000 years--maybe even longer.\\}\\
{impermeable}\\
{অপ্রবেশ্য,,}\\
{(adjective): does not allow fluids to pass through\\The sand bags placed on the river formed an impermeable barrier, protecting the town from flooding.\\}\\
{implicate}\\
{জড়ানো,,}\\
{(verb): convey a meaning; imply\\By saying that some of the guests were uncomfortable, the manager implicated to the hotel staff that it\\needed to be more dilligent.\\}\\
{implicate}\\
{জড়ানো,,}\\
{(verb): to indicate in wrongdoing, usually a crime\\The crime boss was implicated for a long list of crimes, ranging from murder to disturbing the peace.\\}\\
{imponderable}\\
{ভারহীন,,}\\
{(adjective): impossible to estimate or figure out\\According to many lawmakers, the huge variety of factors affecting society make devising an efficient\\healthcare system an imponderable task.\\}\\
{impregnable}\\
{অভেদ্য}\\
{(adjective): immune to attack; incapable of being tampered with\\As a child, Amy would build pillow castles and pretend they were impregnable fotresses.\\}\\
{inadvertent}\\
{অসাবধানী,,}\\
{(adjective): happening by chance or unexpectedly or unintentionally\\Although Prohibition was rooted in noble ideals, the inadvertent and costly consequences of making\\alcohol illegal in the U.S. led its the repeal.\\}\\
{inarticulate}\\
{অস্ফুট,,}\\
{(adjective): without or deprived of the use of speech or words\\Although a brilliant economist, Professor Black was completely inarticulate, a terrible lecturer.\\}\\
{incense}\\
{ধূপ,,}\\
{(verb): make furious\\When Herb bought football tickets for a game on the day of their wedding anniversary, Jill was incensed.\\This word has other definitions but this is the most important one for the GRE\\}\\
{incessant}\\
{নিরবচ্ছিন্ন,,}\\
{(adjective): uninterrupted in time and indefinitely long continuing\\I don't mind small children in brief doses, but I think the incessant exposure that their parents have to\\them would quickly wear me down.\\}\\
{inclement}\\
{প্রচণ্ড,,}\\
{(adjective): (of weather) unpleasant, stormy\\After a week of inclement weather, we finally are able to go outside and enjoy the sun.\\\\                                                                                \\}\\
{inclement}\\
{প্রচণ্ড,,}\\
{(adjective): used of persons or behavior; showing no mercy\\Marcus Aurelius, though a fair man, was inclement to Christians during his reign, persecuting them\\violently.\\}\\
{incumbent}\\
{শায়িত্ব,,}\\
{(adjective): necessary (for someone) as a duty or responsibility\\Middle managers at times make important decisions, but real responsibility for the financial well-being\\of the corporation is ultimately incumbent on the CEO.\\}\\
{indict}\\
{অভিযুক্ত করা}\\
{(verb): to formally charge or accuse of wrong-doing\\The bankrobber was indicted on several major charges, including possession of a firearm.\\}\\
{indigenous}\\
{দেশজ,,}\\
{(adjective): originating in a certain area\\The plants and animals indigenous to Australia are notably different from those indigenous to the U.S--\\one look at a duckbill platypus and you know you're not dealing with an opossum.\\}\\
{indignant}\\
{অকৃতজ্ঞতা,,}\\
{(adjective): feeling anger over a perceived injustice\\When the cyclist swerved into traffic, it forced the driver to brake and elicited an indignant shout of "Hey,\\punk, watch where you're going!"\\}\\
{industrious}\\
{পরিশ্রমী,,}\\
{(adjective): characterized by hard work and perseverance\\Pete was an industrious student, completing every assignment thoroughly and on time.\\}\\
{inflammable}\\
{দাহ্য,,}\\
{(adjective): extremely controversial, incendiary\\It only takes one person to leave an inflammable comment on an Internet thread for that thread to blow\\up into pages upon pages of reader indignation.\\}\\
{ingenuity}\\
{অকপটতা,,}\\
{(noun): the power of creative imagination\\Daedalus was famous for his ingenuity; he was able to fashion his son Icarus with a pair of wings, using\\wax to hold them together.\\}\\
{inkling}\\
{আভাস,,}\\
{(noun): a slight suggestion or vague understanding\\Lynne speaks four Romance languages, but she doesn't have an inkling about how East Asian languages\\are structured.\\}\\
{insipid}\\
{বিরস,,}\\
{(adjective): dull and uninteresting\\The movie director was known for hiring beautiful actors in order to deflect attention away from the\\insipid scripts he would typically use.\\}\\
{insolvent}\\
{দেউলিয়া,,}\\
{(adjective): unable to pay one's bills; bankrupt\\With credit card bills skyrocketing, a shockingly large number of Americans are truly insolvent.\\}\\
{intermittent}\\
{সবিরাম,,}\\
{(adjective): stopping and starting at irregular intervals\\\\                                                                                \\The intermittent thunder continued and the night was punctuated by cracks of lightning--a surreal\\sleepless night.\\}\\
{inundate}\\
{পরিপ্লুত করা,,}\\
{(verb): to flood or overwhelm\\The newsroom was inundated with false reports that only made it more difficult for the newscasters to\\provide an objective account of the bank robbery.\\}\\
{irascible}\\
{একরোখা,,}\\
{(adjective): quickly aroused to anger\\If Arthur's dog is not fed adequately, he becames highly irascible, even growling at his own shadow.\\}\\
{irk}\\
{বিরক্ত করা}\\
{(verb): irritate or vex\\My little sister has a way of irking and annoying me like no other person.\\}\\
{irresolute}\\
{অস্থিরসঙ্কল্প,,}\\
{(adjective): uncertain how to act or proceed\\He stood irresolute at the split in the trail, not sure which route would lead back to the camp.\\}\\
{jargon}\\
{অপভাষা,,}\\
{(noun): a characteristic language of a particular group\\To those with little training in medicine, the jargon of doctors can be very difficult to understand.\\}\\
{jocular}\\
{পরিহাসপ্রিয়তা}\\
{(adjective): characterized by jokes and good humor\\My uncle was always in a jocular mood at family gatherings, messing up people's hair and telling knock-\\knock jokes to anyone who would listen.\\}\\
{junta}\\
{স্পেইনের রাষ্ট্র্রপরিষৎ,,}\\
{(noun): an aggressive takeover by a group (usually military)\\As dangerous of a threat as North Korea is, some analysts believe that were a junta suddenly to gain\\power, it could be even more unpredictable and bellicose than the current leadership\\}\\
{laborious}\\
{শ্রমসাধ্য,,}\\
{(adjective): characterized by effort to the point of exhaustion; especially physical effort\\The most laborious job I've had was working 20 hours a day as a fisherman in King Salmon, Alaska.\\}\\
{leery}\\
{ধূর্ত,,}\\
{(adjective): openly distrustful and unwilling to confide\\Without checking his references and talking to previous employers, I am leery of hiring the candidate.\\}\\
{lethargic}\\
{নিশ্চেষ্ট,,}\\
{(adjective): lacking energy\\Nothing can make a person more lethargic than a big turkey dinner.\\}\\
{lucid}\\
{স্বচ্ছ,,}\\
{(adjective): (of language) transparently clear; easily understandable\\Though Walters writes about physics and time travel, his writing is always lucid, so readers with little\\scientific training can understand difficult concepts.\\}\\
{macabre}\\
{করাল,,}\\
{(adjective): suggesting the horror of death and decay; gruesome\\Edgar Allen Poe was considered the master of the macabre; his stories vividly describe the moment\\leading up to--and often those moments after--a grisly death.\\\\                                                                               \\}\\
{malady}\\
{রোগ,,}\\
{(noun): a disease or sickness\\The town was struck by a malady throughout the winter that left most people sick in bed for two weeks.\\}\\
{malevolent}\\
{হিংসক,,}\\
{(adjective): wishing or appearing to wish evil to others; arising from intense ill will or hatred\\Villians are known for their malevolent nature, oftentimes inflicting cruetly on others just for enjoyment.\\}\\
{malleable}\\
{নমনীয়,,}\\
{(adjective): capable of being shaped or bent or drawn out\\The clay became malleable and easy to work with after a little water was added.\\}\\
{malleable}\\
{নমনীয়,,}\\
{(adjective): easily influenced\\My little brother is so malleable that I can convince him to sneak cookies from the cupboard for me.\\}\\
{malodorous}\\
{দুর্গন্ধযুক্ত,,}\\
{(adjective): having an unpleasant smell\\Some thermally active fountains spew sulfur fumes--the air around them is sometimes so malodorous\\that many have to plug their noses.\\}\\
{martial}\\
{সামরিক,,}\\
{(adjective): suggesting war or military life\\Americans tend to remember Abraham Lincoln as kindly and wise, not at all martial, despite the fact that\\he was involved in the fiercest war America has even fought.\\}\\
{maxim}\\
{প্রবচন,,}\\
{(noun): a short saying expressing a general truth\\Johnson initially suggests that the secret to business can be summarized in a single maxim but then\\requires a 300-page book to explain exactly what he means.\\}\\
{meander}\\
{আঁকাবাঁকা পথ,,}\\
{(verb): to wander aimlessly\\A casual observer might have thought that Peter was meandering through the city, but that day he was\\actually seeking out those places where he and his long lost love had once visited.\\}\\
{melancholy}\\
{বিষণ্ণতা}\\
{(noun): a deep, long-lasting sadness\\Hamlet is a figure of tremendous melancholy: he doesn't have a truly cheerful scene throughout the\\entire play.\\}\\
{melee}\\
{দাঙ্গা,,}\\
{(noun): a wild, confusing fight or struggle\\After enduring daily taunts about my name, I became enraged and pummeled the schoolyard bully and\\his sycophantic friends in a brutal melee.\\}\\
{mesmerize}\\
{সম্মোহিত করা,,}\\
{(verb): to spellbind or enthrall\\The plot and the characters were so well developed that many viewers were mesmerized, unable to\\move their eyes from the screen for even a single second.\\}\\
{misanthrope}\\
{মনুষ্যদ্বেষী,,}\\
{(noun): a hater of mankind\\Hamilton had been deceived so many times in his life that he hid behind the gruff exterior of a\\\\                                                                               \\misanthrope, lambasting perfect strangers for no apparent reason.\\}\\
{miscreant}\\
{দুরাচার,,}\\
{(noun): a person who breaks the law\\Come back you miscreant! yelled the woman who just had her purse stollen.\\}\\
{miser}\\
{কৃপণ,,}\\
{(noun): a person who doesn't like to spend money (because they are greedy)\\Monte was no miser, but was simply frugal, wisely spending the little that he earned.\\}\\
{misogynist}\\
{নারীবিদ্বেষী,,}\\
{(noun): a person who dislikes women in particular\\Many have accused Hemingway of being a quiet misogynist, but recently unearthed letters argue against\\this belief.\\}\\
{moment}\\
{মূহুর্ত}\\
{(noun): significant and important value\\Despite the initial hullabaloo, the play was of no great moment in Hampton's writing career, and within\\a few years the public quickly forgot his foray into theater arts.\\This word has other definitions but this is the most important one for the GRE\\}\\
{moot}\\
{তর্ক করা,,}\\
{(adjective): open to argument or debate; undecidable in a meaningless or irrelevant way\\Since the Board just terminated Steve as the CEO, what the finance committe might have thought of his\\proposed marketing plan for next year is now a moot point.\\This word has other definitions but this is the most important one for the GRE\\}\\
{morose}\\
{বিষণ্ণ,,}\\
{(adjective): ill-tempered and not inclined to talk; gloomy\\After Stanley found out he was no longer able to go on vacation with his friends, he sat in his room\\morosely.\\}\\
{morph}\\
{রূপান্তর,,}\\
{(verb): To undergo dramatic change in a seamless and barely noticeable fashion.\\The earnestness of the daytime talk shows of the 1970's has morphed into something far more\\sensational and vulgar: today guests actually standup and threaten to take swings at one another.\\}\\
{muted}\\
{নীরবকৃত}\\
{(adjective): softened, subdued\\Helen preferred muted earth colors, such as green and brown, to the bright pinks and red her sister liked.\\This word has other definitions but this is the most important one for the GRE\\}\\
{obdurate}\\
{বদ্ধমূল,,}\\
{(adjective): stubbornly persistent in changing an opinion or action\\No number of pleas and bribes would get him to change his obdurate attitude.\\}\\
{obliging}\\
{ভদ্র,,}\\
{(adjective): showing a cheerful willingness to do favors for others\\Even after all his success, I found him to be accommodating and obliging, sharing with me his "secret\\tips" on how to gain wealth and make friends.\\}\\
{obstinate}\\
{একগুঁয়ে,,}\\
{(adjective): resistant to guidance or discipline; stubbornly persistent\\The coach suggested improvements Sarah might make on the balance beam, but she remained\\obstinate, unwilling to modify any of the habits that made her successful in the past.\\}\\
{ornate}\\
{সুসজ্জিত}\\
{(adjective): marked by elaborate rhetoric and elaborated with decorative details\\\\                                                                                 \\The ornate Victorian and Edwardian homes spread throughout San Francisco are my favorite part of the\\city.\\}\\
{paradoxical}\\
{আপাতবিরোধী,,}\\
{(adjective): seemingly contradictory but nonetheless possibly true\\That light could be both a particle and a wave seems paradoxical, but nonetheless, it is true.\\}\\
{pastoral}\\
{যাজকসংক্রান্ত,,}\\
{(adjective): relating to the countryside in a pleasant sense\\Those who imagine America's countryside as a pastoral region are often disappointed to learn that much\\of rural U.S. is filled with cornfields extending as far as the eye can see.\\}\\
{patronize}\\
{রক্ষা করা,,}\\
{(verb): treat condescendingly\\She says she genuinely wanted to help me, but instead she patronized me, constantly pointing out how I\\was inferior to her.\\This word has other definitions but this is the most important one for the GRE\\}\\
{paucity}\\
{অনটন,,}\\
{(noun): a lack of something\\There is a paucity of jobs hiring today that require menial skills, since most jobs have either been\\automated or outsourced.\\}\\
{peevish}\\
{স্বেচ্ছাচারী,,}\\
{(adjective): easily irritated or annoyed\\Our office manager is peevish, so the rest of us tip-toe around him, hoping not to set off another one of\\his fits.\\}\\
{perennial}\\
{চিরজীবী}\\
{(adjective): lasting an indefinitely long time; eternal; everlasting\\Even at the old-timers games, Stan Musial would get the loudest cheer: he was a perennial favorite of\\the fans there.\\This word has other definitions but this is the most important one for the GRE\\}\\
{perpetuate}\\
{চিরস্থায়ী করা,,}\\
{(verb): cause to continue\\If you do not let him do things for himself, you are merely perpetuating bad habits that will be even\\harder to break in the future.\\}\\
{perquisite}\\
{উপরি,,}\\
{(noun): a right reserved exclusively by a particular person or group (especially a hereditary or\\official right)\\Even as the dishwasher at the French restaurant, Josh quickly learned that he had the perquisite of being\\able to eat terrific food for half the price diners would pay.\\}\\
{pertinent}\\
{প্রাসঙ্গিক,,}\\
{(adjective): having precise or logical relevance to the matter at hand\\While the salaries of the players might draw attention in the media, such monetary figures are not\\pertinent to the question of who plays the best on the field.\\}\\
{perturb}\\
{ক্ষুব্ধ করা,,}\\
{(verb): disturb in mind or cause to be worried or alarmed\\Now that Henry is recovering from a major illnesses, he no longer lets the little trivialities, such as late\\mail, perturb him.\\\\                                                                                \\}\\
{peruse}\\
{পুঙ্খানুপুঙ্খভাবে পরীক্ষা করা}\\
{(verb): to read very carefully\\Instead of perusing important documents, people all too often rush to the bottom of the page and\\plaster their signatures at the bottom.\\}\\
{pine}\\
{পাইন গাছ}\\
{(verb): to yearn for\\Standing forlornly by the window, she pined for her lost love.\\This word has other definitions but this is the most important one for the GRE\\}\\
{pinnacle}\\
{চূড়া}\\
{(noun): the highest point\\At its pinnacle, the Roman Empire extended across most of the landmass of Eurasia, a feat not paralleled\\to the rise of the British Empire in the 18th and 19th century.\\}\\
{piquant}\\
{তীব্র,,}\\
{(adjective): having an agreeably pungent taste\\The chef, with a mere flick of the salt shaker, turned the bland tomato soup into a piquant meal.\\}\\
{pithy}\\
{বলিষ্ঠ,,}\\
{(adjective): concise and full of meaning\\I enjoy reading the Daodejing for its pithy and insightful prose; it always gives me something to think\\about.\\}\\
{pittance}\\
{সামান্য বেতন,,}\\
{(noun): a small amount (of money)\\Vinny's uncle beamed smugly about how he'd offered his nephew fifty dollars for his Harvard tuition;\\even twice the amount would have been a mere pittance.\\}\\
{placid}\\
{শান্ত,,}\\
{(adjective): not easily irritated\\Doug is normally placid, so we were all shocked to see him yelling at the television when the Mets lost\\the game.\\}\\
{plodding}\\
{শ্রমশীল,,}\\
{(adjective): (of movement) slow and laborious\\Charlie may seem to run at a plodding pace, but he is an ultramarathoner, meaning he runs distances of\\up to 100 miles, and can run for ten hours at a stretch.\\}\\
{ploy}\\
{চাল}\\
{(noun): a clever plan to turn a situation to one's advantage\\Dennis arranged an elaborate ploy, involving 14 different people lying for him in different situations, so\\that it could appear that he was meeting Mary completely by chance at the wedding reception.\\}\\
{powwow}\\
{কুহকী,,}\\
{(noun): an informal meeting or discussion\\Before the team takes the field, the coach always calls for a powwow so that he can make sure all the\\players are mentally in the right place.\\}\\
{precarious}\\
{নিরাপত্তাহীন,,}\\
{(adjective): fraught with danger\\People smoke to relax and forget their cares, but ironically, in terms of health risks, smoking is far more\\precarious than either mountain-climbing or skydiving.\\}\\
{precedent}\\
{উদাহরণ,,}\\
{(noun): an example that is used to justify similar occurrences at a later time\\\\                                                                               \\The principal explained that even though one student had done modelling work outside of school, the\\outfits that student wore in those photographs in no way established a precedent for what could be\\worn at school dances.\\}\\
{preempt}\\
{}\\
{(verb): take the place of or have precedence over\\A governmental warning about an imminent terrorist attack would preempt ordinary network\\programming on television.\\}\\
{preemptive}\\
{}\\
{(adjective): done before someone else can do it\\Just as Martha was about to take the only cookie left on the table, Noah preemptively swiped it.\\}\\
{presumption}\\
{অনুমান}\\
{(noun): an assumption that is taken for granted\\When Mr. Baker found out the family car was gone, he acted under the presumption that his rebellious\\son had taken the car, calling his son's phone and yelling at him; only later did Mr. Baker realize that\\Mrs. Baker had simply gone out to get her nails d\\}\\
{presumption}\\
{অনুমান}\\
{(noun): audacious (even arrogant) behavior that you have no right to\\The new neighbor quickly gained a reptuation for her presumption; she had invited herself to several of\\the neighbors homes, often stopping over at inopportune times and asking for a drink.\\}\\
{presumptuous}\\
{}\\
{(adjective): excessively forward\\Many felt that Barney was presumptuous in moving into the large office before the management even\\made any official announcement of his promotion.\\}\\
{prevail}\\
{আয়ত্তে আনতে পারা}\\
{(verb): be widespread in a particular area at a particular time; be current:\\During the labor negotiations, an air of hostility prevailed in the office.\\}\\
{prevail}\\
{আয়ত্তে আনতে পারা}\\
{(verb): prove superior\\Before the cricket match, Australia was heavily favored, but India prevailed.\\}\\
{pristine}\\
{অকৃত্রিম}\\
{(adjective): Unspoiled, untouched (usu. of nature)\\The glacial lake was pristine and we filled our canteens to drink deeply.\\}\\
{pristine}\\
{অকৃত্রিম}\\
{(adjective): Immaculately clean and unused\\Drill sergeants are known for demanding pristine cabinets, uniforms, and beds, and often make new\\recruits clean and clean and clean until they meet the expected high standards.\\}\\
{profuse}\\
{অমিতব্যয়ী,,}\\
{(adjective): plentiful; pouring out in abundance\\During mile 20 of the Hawaii Marathon, Dwayne was sweating so profusely that he stopped to take off\\his shirt, and ran the remaining six miles wearing nothing more than skimpy shorts.\\Note:\\}\\
{profusion}\\
{অমিতব্যয়িতা,,}\\
{(noun): the property of being extremely abundant\\When Maria reported that she had been visited by Jesus Christ and had proof, a profusion of reporters\\\\                                                                                 \\and journalists descended on the town.\\}\\
{proponent}\\
{প্রবক্তা,,}\\
{(noun): a person who pleads for a cause or propounds an idea\\Ironically, the leading proponent of Flat-Earth Theory flies all over the world in an effort to win more\\adherents.\\}\\
{provisional}\\
{কাঁচা,,}\\
{(adjective): under terms not final or fully worked out or agreed upon\\Until the corporate office hands down a definitive decision on use of the extra offices, we will share their\\use in a provisional arrangement.\\}\\
{pugnacious}\\
{কলহপ্রিয়,,}\\
{(adjective): eager to fight or argue; verbally combative\\The comedian told one flat joke after another, and when the audience started booing, he pugnaciously\\back at them, "Hey, you think this is easy -- why don"t you buffoons give it a shot?"\\}\\
{qualm}\\
{বিবেকের দংশন,,}\\
{(noun): uneasiness about the fitness of an action\\While he could articulate no clear reason why Harkner's plan would fail, he neverless felt qualms about\\committing any resources to it.\\}\\
{quandary}\\
{মুশকিল,,}\\
{(noun): state of uncertainty or perplexity especially as requiring a choice between equally\\unfavorable options\\Steve certainly is in a quandary: if he doesn't call Elaine, she will blame him for everything, but if he does\\call her, the evidence of where he currently is could cost him his job.\\}\\
{quip}\\
{বাক্ছল,,}\\
{(noun): a witty saying or remark\\In one of the most famous quips about classical music, Mark Twain said: "Wagner's music is better than\\it sounds."\\}\\
{raffish}\\
{কুখ্যাত,,}\\
{(adjective): marked by a carefree unconventionality or disreputableness\\The men found him raffish, but the women adored his smart clothes and casual attitude.\\}\\
{raft}\\
{ভেলা,,}\\
{(noun): a large number of something\\Despite a raft of city ordinances passed by an overzealous council, noise pollution continued unabated in\\the megalopolis.\\This word has other definitions but this is the most important one for the GRE\\}\\
{rakish}\\
{লম্পটস্বভাব,,}\\
{(adjective): marked by a carefree unconventionality or disreputableness\\As soon as he arrived in the city, the rakish young man bought some drugs and headed straight for the\\seedy parts of town.\\}\\
{rankle}\\
{পচিয়া উঠা,,}\\
{(verb): gnaw into; make resentful or angry\\His constant whistling would rankle her, sometimes causing her to leave in a huff.\\}\\
{rash}\\
{ফুসকুড়ি,,}\\
{(adjective): marked by defiant disregard for danger or consequences; imprudently incurring risk\\Although Bruce was able to make the delivery in time with a nightime motorcycle ride in the rain, Susan\\\\                                                                                \\criticized his actions as rash.\\This word has other definitions but this is the most important one for the GRE\\}\\
{redress}\\
{প্রতিকার,,}\\
{(noun): an act of making something right\\Barry forgot his wife's birthday two years in a row, and was only able to redress his oversight by\\surprising his wife with a trip to Tahiti.\\This word has other definitions but this is the most important one for the GRE\\}\\
{relegate}\\
{নির্বাসিত করা,,}\\
{(verb): assign to a lower position\\When Dexter was unable to fulfill his basic duties, instead of firing him, the boss relegated him to kitchen\\cleanup.\\}\\
{remiss}\\
{শিথিল,,}\\
{(adjective): to be negligent in one"s duty\\Remiss in his duty to keep the school functioning efficiently, the principle was relieved of his position\\after only three months.\\}\\
{renege}\\
{পরিত্যাগ করা,,}\\
{(verb): fail to fulfill a promise or obligation\\We will no longer work with that vendor since it has reneged on nearly every agreement.\\}\\
{replete}\\
{পরিপূর্ণ,,}\\
{(adjective): completely stocked or furnished with something\\Only weeks after the hurricane made landfall, the local supermarket shelves were replete with goods, so\\quick was the disaster relief response.\\}\\
{reprobate}\\
{দুশ্চরিত্র,,}\\
{(noun): a person who is disapproved of\\Those old reprobates drinking all day down by the river--they are not going to amount to much.\\}\\
{reservation}\\
{সংরক্ষণ}\\
{(noun): an unstated doubt that prevents you from accepting something wholeheartedly\\I was initially excited by the idea of a trip to Washington, D.C. but now that I have read about the high\\crime statistics there, I have some reservations.\\This word has other definitions but this is the most important one for the GRE\\}\\
{resignation}\\
{ইস্তফা}\\
{(noun): the acceptance of something unpleasant that can't be avoided\\Since Jack could not think of a convincing reason why he had to miss the seminar, he attended it with a\\sense of resignation.\\This word has other definitions but this is the most important one for the GRE\\}\\
{resolve}\\
{স্থিরসঙ্কল্প করা}\\
{(verb): reach a conclusion after a discussion or deliberation\\After much thought, Ted resolved not to travel abroad this summer because he didn't have much money\\in his bank account.\\This word has other definitions but this is the most important one for the GRE\\}\\
{respite}\\
{অবকাশ,,}\\
{(noun): a pause from doing something (as work)\\Every afternoon, the small company has a respite in which workers play foosball or board games.\\}\\
{retiring}\\
{চাপা স্বভাবের,,}\\
{(adjective): to be shy, and to be inclined to retract from company\\Nelson was always the first to leave soirees--rather than mill about with "fashionable" folk, he was\\retiring, and preferred the solitude of his garret.\\\\                                                                               \\}\\
{retract}\\
{প্রত্যাহার করা}\\
{(verb): pull inward or towards a center; formally reject or disavow a formerly held belief, usually\\under pressure\\Email is wonderfully efficient, but once something awkward or damaging has been sent, there is no way\\to retract it.\\}\\
{rile}\\
{বিরক্ত করা}\\
{(verb): cause annoyance in; disturb, especially by minor irritations\\Dan is usually calm and balanced, but it takes only one intense glare from Sabrina to rile him.\\}\\
{robust}\\
{শক্তিশালী}\\
{(adjective): sturdy and strong in form, constitution, or construction\\Chris preferred bland and mild beers, but Bhavin preferred a beer with more robust flavor.\\}\\
{sanctimonious}\\
{ধর্মধ্বজী,,}\\
{(adjective): making a show of being pious; holier-than-thou\\Even during the quiet sanctity of evening prayer, she held her chin high, a sanctimonious sneer forming\\on her face as she eyed those who were attending church for the first time.\\}\\
{sanguine}\\
{অরূণ,,}\\
{(adjective): cheerful; optimistic\\With the prospect of having to learn 3,000 words during the course of the summer, Paul was anything\\but sanguine.\\}\\
{savvy}\\
{কাণ্ডজ্ঞান,,}\\
{(noun): a perceptive understanding\\Although a great CEO, he did not have the political savvy to win the election.\\}\\
{savvy}\\
{কাণ্ডজ্ঞান,,}\\
{(verb): get the meaning of something\\The student savvies the meaning of astrophysics with little effort.\\}\\
{savvy}\\
{কাণ্ডজ্ঞান,,}\\
{(adjective): well-informed or perceptive\\With his savvy business partner, the company was able to turn a profit within a year.\\}\\
{scintillating}\\
{স্ফুলিঙ্গ ছড়ান,,}\\
{(adjective): describes someone who is brilliant and lively\\Richard Feynman was renowned for his scintillating lectures--the arcana of quantum physics was made\\lucid as he wrote animatedly on the chalkboard.\\}\\
{screed}\\
{লেখা}\\
{(noun): an abusive rant (often tedious)\\Joey had difficulty hanging out with his former best friend Perry, who, during his entire cup of coffee,\\enumerated all of the government's deficiencies--only to break ranks and launch into some screed\\against big business.\\}\\
{sentimental}\\
{ভাবপ্রবণ,,}\\
{(adjective): effusively or insincerely emotional, especially in art, music, and literature\\I don't like romanticism for the same reason I don't like melodramatic acting and soap operas--overly\\sentimental.\\}\\
{serendipity}\\
{অনাকাঙ্খিত আবিস্কার}\\
{(noun): the instance in which an accidental, fortunate discovery is made\\The invention of the 3M Post It Note was serendipitous, because the scientist who had come up with the\\\\                                                                               \\idea was looking for a strong adhesive; the weak adhesive he came up with was perfect for holding a\\piece of paper in place but made it very easy for so\\}\\
{serene}\\
{শান্ত}\\
{(adjective): calm and peaceful\\I'd never seen him so serene; usually, he was a knot of stress and anxiety from hours of trading on the\\stock exchange.\\}\\
{slapdash}\\
{বেপরোয়া,,}\\
{(adjective): carelessly and hastily put together\\The office building had been constructed in a slapdash manner, so it did not surprise officials when,\\during a small earthquake, a large crack emerged on the façade of the building.\\}\\
{smattering}\\
{পল্লবগ্রাহী,,}\\
{(noun): a slight or superficial understanding of a subject; a small amount of something\\I know only a smattering of German, but Helen is able to read German newspapers and converse with\\natives.\\}\\
{smug}\\
{ফিটফাট,,}\\
{(adjective): marked by excessive complacency or self-satisfaction\\When Phil was dating the model, he had a smug attitude that annoyed his buddies.\\}\\
{snide}\\
{কটাক্ষপূর্ণ,,}\\
{(adjective): expressive of contempt; derogatory or mocking in an indirect way\\The chairman interpreted Taylor's question about promotions as a snide remark, but in all innocence\\Taylor was trying to figure out the company's process.\\}\\
{snub}\\
{তিরস্কার,,}\\
{(verb): refuse to acknowledge; reject outright and bluntly\\Wheeler was completely qualified for the committee, but the board snubbed him, choosing an obviously\\lesser qualified candidate instead.\\}\\
{sordid}\\
{নোংরা,,}\\
{(adjective): involving ignoble actions and motives; arousing moral distaste and contempt; foul\\and run-down and repulsive\\The nightly news simply announced that the senator had had an affair, but the tabloid published all the\\sordid details of the interaction.\\This word has other definitions but this is the most important one for the GRE\\}\\
{spendthrift}\\
{ডোকলা,,}\\
{(noun): one who spends money extravagantly\\Taking weekly trips to Vegas, Megan was a spendthrift whose excesses eventually caught up to her.\\}\\
{spurn}\\
{হয়রান করা}\\
{(verb): reject with contempt\\She spurned all his flattery and proposals, and so he walked off embarrassed and sad.\\}\\
{squander}\\
{অপব্যয় করা}\\
{(verb): spend thoughtlessly; waste time, money, or an opportunity\\Fearing his money would be squandered by his family, he gave all of it to charity when he died.\\}\\
{staid}\\
{রাশভারী}\\
{(adjective): characterized by dignity and propriety\\Frank came from a staid enviroment, so he was shocked that his college rooommate sold narcotics.\\\\                                                                                  \\}\\
{start}\\
{শুরু করা}\\
{(verb): to suddenly move in a particular direction\\All alone in the mansion, Henrietta started when she heard a sound.\\This word has other definitions but this is the most important one for the GRE\\}\\
{steadfast}\\
{অপলক,,}\\
{(adjective): marked by firm determination or resolution; not shakable\\A good captain needs to be steadfast, continuing to hold the wheel and stay the course even during the\\most violent storm.\\}\\
{stem}\\
{ডাঁটা,,}\\
{(verb): to hold back or limit the flow or growth of something\\To stem the tide of applications, the prestigious Ivy requires that each applicant score at least 330 on the\\Revised GRE.\\This word has other definitions but this is the most important one for the GRE\\}\\
{stipend}\\
{বৃত্তি,,}\\
{(noun): a regular allowance (of money)\\He was hoping for a monthly allowance loan from the government, but after no such stipend was\\forthcoming he realized he would have to seek other means of paying for his college tuition.\\}\\
{stolid}\\
{অটল,,}\\
{(adjective): having or revealing little emotion or sensibility; not easily aroused or excited\\Elephants may appear stolid to casual observers, but they actually have passionate emotional lives.\\}\\
{stymie}\\
{কোণঠাসা করা,,}\\
{(verb): hinder or prevent the progress or accomplishment of\\The engineers found their plans stymied at every turn and were ultimately able to make amlost no\\progress on the project.\\}\\
{summit}\\
{চূড়া}\\
{(noun): the peak or highest point\\After hiking for two days, the climbers finally reached the summit of Mount Kilimanjaro.\\}\\
{summit}\\
{চূড়া}\\
{(noun): a meeting of high-level leaders\\Since climate change policy has been mired in congressional fighting, this summit should help set the\\goals for president's next term.\\}\\
{surly}\\
{কর্কশকণ্ঠ,,}\\
{(adjective): inclined to anger or bad feelings with overtones of menace\\Every morning, Bhavin was a surly unhappy person, but once he ate breakfast, he became loving,\\laughing, and a joy to be around.\\}\\
{tact}\\
{কৌশলের সূক্ষ্মতা,,}\\
{(noun): consideration in dealing with others and avoiding giving offense\\In a tremendous display of tact, Shelly was able to maintain a strong friendship with Marcia, even\\though Marcia's husband, Frank, confessed to finding Shelley more attractive than Marcia.\\}\\
{tarnish}\\
{মরিচা,,}\\
{(verb): make dirty or spotty, as by exposure to air; also used metaphorically\\Pete Rose was one of the best baseball players of his generation, but his involvement with gambling on\\baseball games has tarnished his image in the eyes of many.\\}\\
{tawdry}\\
{রূচিহীন,,}\\
{(adjective): tastelessly showy; cheap and shoddy\\Carol expected to find New York City magical, the way so many movies had portrayed it, but she was\\\\                                                                               \\surprised how often tawdry displays took the place of genuine elegance.\\}\\
{taxing}\\
{কর আরোপিত,,}\\
{(adjective): use to the limit; exhaust\\The hike to the summit of Mt. Whitney was so taxing that I could barely speak or stand up.\\This word has other definitions but this is the most important one for the GRE\\}\\
{telling}\\
{কহন,,}\\
{(adjective): significant and revealing of another factor\\Her unbecoming dress was very telling when it came to her sense of fashion.\\}\\
{telltale}\\
{যে ব্যক্তি পরের অন্যায় জানায়,,}\\
{(adjective): revealing\\The many telltale signs of chronic smoking include yellow teeth, and a persistent, hacking cough.\\}\\
{tender}\\
{টেন্ডার}\\
{(verb): offer up something formally\\The government was loath to tender more money in the fear that it might set off inflation.\\This word has other definitions but this is the most important one for the GRE\\}\\
{thoroughgoing}\\
{সর্বব্যাপী,,}\\
{(adjective): very thorough; complete\\As a thoroughgoing bibliophile, one who had turned his house into a veritable library, he shocked his\\friends when he bought a Kindle.\\}\\
{thrifty}\\
{গোছানো}\\
{(adjective): spending money wisely\\He was economical, spending his money thriftily and on items considered essential.\\}\\
{thwart}\\
{অনুপ্রস্থ,,}\\
{(verb): hinder or prevent (the efforts, plans, or desires) of\\I wanted to spend a week in New York this autumn, but the high costs of travel and lodging thwarted my\\plans.\\}\\
{tirade}\\
{সুদীর্ঘ বক্তৃতা,,}\\
{(noun): an angry speech\\In terms of political change, a tirade oftentimes does little more than make the person speaking red in\\the face.\\}\\
{tout}\\
{দালাল,,}\\
{(verb): advertize in strongly positive terms; show off\\At the conference, the CEO touted the extraordinary success of his company's Research \& Development\\division.\\}\\
{transitory}\\
{ক্ষণস্থায়ী,,}\\
{(adjective): lasting a very short time\\If we lived forever and life was not transitory, do you think we would appreciate life less or more?\\}\\
{travail}\\
{পরিশ্রম,,}\\
{(noun): use of physical or mental energy; hard work; agony or anguish\\While they experienced nothing but travails in refinishing the kitchen, they completed the master\\bedroom in less than a weekend.\\}\\
{tribulation}\\
{দুর্দশা,,}\\
{(noun): something, especially an event, that causes difficulty and suffering\\As of 2013, nearly 1.5 million Syrians have fled their country hoping to escape the tribulations of a civil\\war tearing their country to pieces.\\\\                                                                                \\}\\
{tumult}\\
{আরাব,,}\\
{(noun): a state of chaos, noise and confusion\\Riots broke out just in front of our apartment building, and the tumult continued late into the night.\\}\\
{uncanny}\\
{ভুতুড়ে,,}\\
{(adjective): suggesting the operation of supernatural influences; surpassing the ordinary or\\normal\\Reggie has an uncanny ability to connect with animals: feral cats will readily approach him, and\\sometimes even wild birds will land on his finger.\\}\\
{uncompromising}\\
{পুরাদস্তুর,,}\\
{(adjective): not making concessions\\The relationship between Bart and Hilda ultimately failed because they were both so uncompromising,\\never wanting to change their opinions.\\}\\
{unconscionable}\\
{বিবেকবর্জিত,,}\\
{(adjective): unreasonable; unscrupulous; excessive\\The lawyer's demands were so unconscionable that rather than pay an exorbitant sum or submit himself\\to any other inconveniences, the defendant decided to find a new lawyer.\\}\\
{underwrite}\\
{অর্থলগ্নী,,}\\
{(verb): to support financially\\The latest symphony broadcast was made possible with underwriting from the Carnegie Endowment.\\}\\
{unnerve}\\
{সাহসশূন্য বলশূন্য করা,,}\\
{(verb): to make nervous or upset\\At one time unnerved by math problems, she began avidly "Magoosh-ing", and soon became adept at\\even combinations and permutations questions.\\}\\
{unprecedented}\\
{অভূতপূর্ব,,}\\
{(adjective): having never been done or known before; novel\\When America first created its national parks, the idea of setting aside the most beautiful land in a\\country was unprecedented in the history of mankind.\\}\\
{unruly}\\
{অবাধ্য,,}\\
{(adjective): (of persons) noisy and lacking in restraint or discipline; unwilling to submit to\\authority\\Walk in to any preschool and I am sure that you will find an unruly and chaotic scene--unless it's nap\\time.\\}\\
{unseemly}\\
{অনুচিত,,}\\
{(adjective): not in keeping with accepted standards of what is right or proper in polite society\\He acted in an unseemly manner, insulting the hostess and then speaking ill of her deceased husband.\\}\\
{urbane}\\
{সুসংস্কৃত,,}\\
{(adjective): showing a high degree of refinement and the assurance that comes from wide social\\experience\\Because of his service as an intelligence officer and his refined tastes, W. Somerset Maugham became\\the inspiration for the urbane and sophistcate spy James Bond.\\}\\
{vacuous}\\
{ফাঁকা,,}\\
{(adjective): devoid of intelligence, matter, or significance\\To the journalist's pointed question, the senator gave a vacuous response, mixing a few of his overall\\\\                                                                                \\campaign slogans with platitudes and completely avoiding the controversial subject of the question.\\}\\
{vanquish}\\
{পটকান,,}\\
{(verb): come out better in a competition, race, or conflict\\For years, Argentina would dominate in World Cup qualifying matches, only to be vanquished by one of\\the European countries during the late stages of the tournament.\\}\\
{variance}\\
{অনৈক্য,,}\\
{(noun): the quality of varying\\The cynic quipped, "There is not much variance in politicians; they all seem to prevaricate".\\}\\
{veneer}\\
{পাতলা আবরণ}\\
{(noun): covering consisting of a thin superficial layer that hides the underlying substance\\Mark Twain referred to the Victorian Period in America as the "Gilded Age", implying the ample moral\\corruption that lay beneath a mere veneer of respectability.\\}\\
{vicarious}\\
{প্রতিনিধি কাজ করে এমন,,}\\
{(adjective): felt or undergone as if one were taking part in the experience or feelings of\\another\\The advent of twitter is a celebrity stalker's dream, as he or she can--through hundreds of intimate\\"tweets"--vicariously live the life of a famous person.\\}\\
{vie}\\
{প্রতিযোগিতা করা,,}\\
{(verb): compete for something\\While the other teams in the division actively vie for the championship, this team seems content simply\\to go through the motions of playing.\\}\\
{vindictive}\\
{প্রতিহিংসাপরায়ণ}\\
{(adjective): to have a very strong desire for revenge\\Though the other girl had only lightly poked fun of Vanessa's choice in attire, Vanessa was so vindictive\\that she waited for an entire semester to get the perfect revenge.\\}\\
{virago}\\
{উগ্রচন্ড রমণী}\\
{(noun): an ill-tempered or violent woman\\Poor Billy was the victim of the virago's invective--she railed at him for a good 30-minutes about how he\\is the scum of the earth for speaking loudly on his cellphone in public.\\}\\
{voracious}\\
{ঔদরিক}\\
{(adjective): very hungry; approaching an activity with gusto\\Steven was a voracious reader, sometimes finishing two novels in the same day.\\}\\
{wanton}\\
{অকারণ}\\
{(adjective): without check or limitation; showing no moral restraints to one's anger, desire, or\\appetites\\Due to wanton behavior and crude language, the drunk man was thrown out of the bar and asked to\\never return.\\}\\
{wax}\\
{বৃদ্ধি পাওয়া}\\
{(verb): to gradually increase in size or intensity\\Her enthusiasm for the diva's new album only waxed with each song; by the end of the album, it was her\\favorite CD yet.\\This word has other definitions but this is the most important one for the GRE\\}\\
{whimsical}\\
{বাতিকগ্রস্ত}\\
{(adjective): determined by impulse or whim rather than by necessity or reason\\\\                                                                                  \\Adults look to kids and envy their whimsical nature at times, wishing that they could act without reason\\and play without limitation.\\}\\
{zenith}\\
{সুবিন্দু}\\
{(noun): the highest point; culmination\\At the zenith of his artistic career, Elvis was outselling any other artist on the charts.\\\\}\\
\section{ADVANCED WORDS}
{abjure}\\
{শপথ পূর্বক পরিত্যাগ করা}\\
{(verb): formally reject or give up (as a belief)\\While the church believed that Galileo abjured the heliocentric theory under threat of torture, he later\\wrote a book clearly supporting the theory.\\}\\
{abrogate}\\
{রদ করা}\\
{(verb): revoke formally\\As part of the agreement between the labor union and the company, the workers abrogated their right\\to strike for four years in exchange for better health insurance.\\}\\
{adjudicate}\\
{রায় দেওয়া}\\
{(verb): to serve as a judge in a competition; to arrive at a judgment or conclusion\\Only those with the most refined palates were able to adjudicate during the barbeque competition.\\}\\
{afford}\\
{সমর্থ হওয়া}\\
{(verb): provide with an opportunity\\The summit of Mt. Kilimanjaro affords a panoramic view that encompasses both Tanzania and Kenya.\\This word has other definitions but this is the most important one for the GRE\\}\\
{alacrity}\\
{তৎপরতা}\\
{(noun): an eager willingness to do something\\The first three weeks at his new job, Mark worked with such alacrity that upper management knew it\\would be giving him a promotion.\\}\\
{anachronism}\\
{কালবৈষম্য,,}\\
{(noun): something that is inappropriate for the given time period (usually something old).\\Dressed in 15th century clothing each day, Edward was a walking anachronism.\\}\\
{anathema}\\
{অভিশাপ,,}\\
{(noun): a detested person; the source of somebody's hate\\Hundreds of years ago, Galileo was anathema to the church; today the church is anathema to some on\\the left side of the political spectrum.\\}\\
{anemic}\\
{রক্তহীন,,}\\
{(adjective): lacking energy and vigor\\After three straight shows, the lead actress gave an anemic performance the fourth night, barely\\speaking loudly enough for those in the back rows to hear.\\}\\
{anodyne}\\
{বেদনানাশক,,}\\
{(noun): something that soothes or relieves pain\\Muzak, which is played in department stores, is intended to be an anodyne, but is often so cheesy and\\over-the-top that customers become irritated.\\}\\
{anodyne}\\
{বেদনানাশক,,}\\
{(adjective): inoffensive\\Wilbur enjoyed a spicy Mexican breakfast, but Jill prefered a far more anodyne meal in the mornings.\\}\\
{antic}\\
{অদ্ভুতদর্শন,,}\\
{(adjective): ludicrously odd\\The clown's antic act was too extreme for the youngest children, who left the room in tears.\\This word has other definitions but this is the most important one for the GRE\\}\\
{aplomb}\\
{আত্মবিশ্বাস,,}\\
{(noun): great coolness and composure under strain\\Nancy acted with aplomb during dangerous situations--she once calmly climbed up an oak tree to save a\\\\                                                                               \\cat.\\}\\
{apogee}\\
{অপভূ}\\
{(noun): the highest point\\The apogee of the Viennese style of music, Mozart's music continues to mesmerize audiences well into\\the 21st century.\\}\\
{apostate}\\
{ধর্মত্যাগী,,}\\
{(noun): a person who has abandoned a religious faith or cause\\An apostate of the Republican Party, Sheldon has yet to become affiliated with any party and dubs\\himself an independent.\\}\\
{apothegm}\\
{রিসালাত,,}\\
{(noun): a short, pithy instructive saying\\Winston Churchill is famous for many apothegms, but this might be his most famous: "It has been said\\that democracy is the worst form of government except all the others that have been tried."\\}\\
{apotheosis}\\
{নাটকের শেষ,,}\\
{(noun): exaltation to divine status; the highest point of development\\As difficult as it is to imagine, the apotheosis of Mark Zuckerberg's career, many believe, is yet to come.\\}\\
{approbatory}\\
{অনুমোদনাত্মক,,}\\
{(adjective): expressing praise or approval\\Although it might not be her best work, Hunter's new novel has received generally approbatory reviews.\\}\\
{appropriate}\\
{উপযুক্ত}\\
{(verb): to give or take something by force\\The government appropriated land that was occupied by squatters, sending them scurrying for another\\place to live.\\This word has other definitions but this is the most important one for the GRE\\}\\
{appropriate}\\
{উপযুক্ত}\\
{(verb): to allocate\\The committe appropriated the funds to its various members.\\This word has other definitions but this is the most important one for the GRE\\}\\
{appurtenant}\\
{আনুষঙ্গিক,,}\\
{(adjective): supply added support\\In hiking Mt. Everest, sherpas are appurtenant, helping climbers both carry gear and navigate\\treacherous paths.\\}\\
{arch}\\
{বক্ররেখা}\\
{(adjective): to be deliberately teasing\\The baroness was arch, making playful asides to the townspeople; yet because they couldn't pick up on\\her dry humor, they thought her supercilious.\\This word has other definitions but this is the most important one for the GRE\\}\\
{arrant}\\
{ডাহা,,}\\
{(adjective): complete and wholly (usually modifying a noun with negative connotation)\\An arrant fool, Lawrence surprised nobody when he lost all his money in a pyramid scheme that was\\every bit as transparent as it was corrupt.\\}\\
{arriviste}\\
{}\\
{(noun): a person who has recently reached a position of power; a social climber\\The city center was aflutter with arrivistes who each tried to outdo one another with their ostentatious\\sports cars and chic evening dress.\\\\                                                                                \\}\\
{arrogate}\\
{গর্ব প্রকাশ করা,,}\\
{(verb): seize and control without authority\\Arriving at the small town, the outlaw arrogated the privileges of a lord, asking the frightened citizens to\\provide food, drink, and entertainment.\\}\\
{artifice}\\
{দক্ষতা}\\
{(noun): cunning tricks used to deceive others\\The mayoral candidates both spent much of the campaign accusing each other of artifices designed to\\mislead the voting public.\\}\\
{artless}\\
{অপটু,,}\\
{(adjective): without cunning or deceit\\Despite the president's seemingly artless speeches, he was a skilled and ruthless negotiator.\\}\\
{artlessness}\\
{}\\
{(noun): the quality of innocence\\I, personally, found the artlessness of her speech charming.\\}\\
{asperity}\\
{কাটব্য,,}\\
{(noun): harshness of manner\\The editor was known for his asperity, often sending severe letters of rejection to amateur writers.\\}\\
{assiduously}\\
{পরিশ্রমী,,}\\
{(adverb): with care and persistence\\The top college football program recruits new talent assiduously, only choosing those who were the top\\in their county.\\}\\
{atavism}\\
{পূর্বগানুকৃতি,,}\\
{(noun): a reappearance of an earlier characteristic; throwback\\Much of the modern art movement was an atavism to a style of art found only in small villages through\\Africa and South America.\\}\\
{attenuate}\\
{কৃশ,,}\\
{(verb): to weaken (in terms of intensity); to taper off/become thinner.\\Her animosity towards Bob attenuated over the years, and she even went so far as to invite him to her\\party.\\}\\
{autocratic}\\
{স্বৈরাচারী,,}\\
{(adjective): characteristic of an absolute ruler or absolute rule; having absolute sovereignty\\The last true autocratic country is certainly North Korea; nowhere does a leader exercise the absolute\\control over all aspects of a people the way that Kim Jong-un does.\\}\\
{autocratic}\\
{স্বৈরাচারী,,}\\
{(adjective): offensively self-assured or given to exercising usually unwarranted power\\The manager was finally fired for his autocratic leadership, which often bordered on rude and offensive.\\}\\
{baleful}\\
{বিপর্যয়কর,,}\\
{(adjective): threatening or foreshadowing evil or tragic developments\\Movies often use storms or rain clouds as a baleful omen of evil events that will soon befall the main\\character.\\}\\
{base}\\
{ভিত্তি করা}\\
{(adjective): the lowest, class were without any moral principles\\She was not so base as to begrudge the beggar the unwanted crumbs from her dinner plate.\\This word has other definitions but this is the most important one for the GRE\\\\                                                                               \\}\\
{bastardization}\\
{}\\
{(noun): an act that debases or corrupts\\The movie World War Z is a complete bastardization of the book with little more in common than\\zombies and a title.\\}\\
{beg}\\
{ভিক্ষা করা}\\
{(verb): to evade or dodge (a question)\\By assuming that Charlie was headed to college--which he was not--Maggie begged the question when\\she asked him to which school he was headed in the Fall.\\This word has other definitions but this is the most important one for the GRE\\}\\
{bemoan}\\
{আক্ষেপ করা,,}\\
{(verb): express discontent or a stong regret\\While the CFO carefully explained all the reasons for the cuts in benefits, after the meeting employees\\bemoaned the cuts as further evidence that management was against them.\\}\\
{benighted}\\
{রাত্রিগ্রস্ত,,}\\
{(adjective): fallen into a state of ignorance\\Far from being a period of utter benightedness, The Medieval Ages produced some great works of\\theological speculation.\\}\\
{bereft}\\
{প্রি়জন - নিয়োগবিধুর,,}\\
{(adjective): unhappy in love; suffering from unrequited love\\After 64 years of marriage, William was bereft after the death of his wife.\\}\\
{bereft}\\
{প্রি়জন - নিয়োগবিধুর,,}\\
{(adjective): sorrowful through loss or deprivation\\You are not bereft if you haven't played on your Xbox in the past week, his mother said.\\}\\
{besotted}\\
{প্রমত্ত,,}\\
{(adjective): strongly affectionate towards\\Even though her father did not approve, Juliet became besotted with the young Romeo.\\}\\
{besotted}\\
{প্রমত্ত,,}\\
{(adjective): very drunk\\Never before have I seen my mom so besotted, and honestly, I hope it's the last time she drinks so much.\\}\\
{bilious}\\
{পৈত্তিক,,}\\
{(adjective): irritable; always angry\\Rex was bilious all morning, and his face would only take on a look of contentedness when he'd had his\\morning cup of coffee.\\}\\
{blinkered}\\
{}\\
{(adjective): to have a limited outlook or understanding\\In gambling, the addict is easily blinkered by past successes and/or past failures, forgetting that the\\outcome of any one game is independent of the games that preceded it.\\}\\
{bowdlerize}\\
{কোনো বইয়ের অশ্লীল অংশ বাদ দেওয়া,,}\\
{(verb): edit by omitting or modifying parts considered indelicate\\To recieve an R rating, the entire movie was bowdlerized because it contained so much violence and\\grotesque subject matter.\\}\\
{bridle}\\
{লাগাম,,}\\
{(verb): the act of restraining power or action or limiting excess\\New curfew laws have bridled people's tendency to go out at night.\\\\                                                                                  \\}\\
{bridle}\\
{লাগাম,,}\\
{(verb): anger or take offense\\The hostess bridled at the tactless dinner guests who insisted on eating before everybody had gotten\\their food.\\}\\
{bristle}\\
{কূর্চ,,}\\
{(verb): react in an offended or angry manner\\As we discussed the painting, I noticed the artitst's wife bristling at our criticisms, ready to defend her\\husband's work.\\}\\
{broadside}\\
{রণপোতের একপার্শ্বস্থিত সমুদয় কামান,,}\\
{(noun): a strong verbal attack\\Political broadsides are usually strongest in the weeks leading up to a national election.\\}\\
{bromide}\\
{ব্রোমাইড}\\
{(noun): a trite or obvious remark\\Instead of sharing his umbrella, the cheeky stranger offered Martha the following bromide: "Looks like\\it's raining."\\}\\
{brook}\\
{সহ্য করা,,}\\
{(verb): put up with something or somebody unpleasant\\While she was at the chalkboard, the teacher did not brook any form of talking--even a tiny peep\\resulted in afternoon detention.\\This word has other definitions but this is the most important one for the GRE\\}\\
{browbeat}\\
{শাসান,,}\\
{(verb): be bossy towards; discourage or frighten with threats or a domineering manner\\During the interrogation, the suspect was browbeaten into signing a false confession.\\}\\
{byzantine}\\
{কনস্ট্যাণ্টিনোপলের,,}\\
{(adjective): intricate and complex\\Getting a driver's license is not simply a matter of taking a test; the regulations and procedures are so\\byzantine that many have found themselves at the mercy of the Department of Motor Vehicles.\\}\\
{callow}\\
{অজাতশ্মশ্রু,,}\\
{(adjective): young and inexperienced\\Both Los Angeles and New York are known for callow out-of-towners hoping to make it big.\\}\\
{canard}\\
{মিথ্যা গুজব,,}\\
{(noun): a deliberately misleading fabrication\\The public will always be fooled by the media's canards.\\}\\
{capitulate}\\
{আত্মসমর্পণ করা,,}\\
{(noun): to surrender (usually under agreed conditions)\\Paul, losing 19-0 in a ping-pong match against his nimble friend, basically capitulated when he played\\the last two points with his eyes closed.\\}\\
{cataclysm}\\
{মহাযুদ্ধ}\\
{(noun): an event resulting in great loss and misfortune\\The introduction of smallpox was a cataclysm for Native Americans, killing off more than half of their\\population.\\}\\
{catholic}\\
{সার্বজনীন}\\
{(adjective): of broad scope; universal\\Jonah's friends said that Jonah's taste in music was eclectic; Jonah was quick to point out that not only\\was his taste eclectic but it was also catholic: he enjoyed music from countries as far-flung as Mali and\\\\                                                                               \\Mongolia.\\}\\
{cede}\\
{ছেড়ে দেওয়া,,}\\
{(verb): relinquish possession or control over\\Eventually, all parents must cede control of their growing childrens' educations and allow their offspring\\some autonomy.\\}\\
{celerity}\\
{ক্ষিপ্রতা}\\
{(noun): speed, rapidity\\We aim to respond to customers' questions with celerity and accuracy, with no longer than a 24 hour\\wait time.\\}\\
{chagrin}\\
{হতাশা}\\
{(noun): strong feelings of embarrassment\\Much to the the timid writer's chagrin, the audience chanted his name until he came back on the stage.\\}\\
{chagrin}\\
{হতাশা}\\
{(verb): cause to feel shame; hurt the pride of\\She never cared what others said about her appearance but was chagrined by the smallest comment\\from her mother.\\}\\
{charlatan}\\
{হাতুড়ে ডাক্তার,,}\\
{(noun): a flamboyant deceiver; one who attracts customers with tricks or jokes\\You may call him a "motivational speaker," but I call him a charlatan--he doesn't have any idea what he's\\really talking about.\\}\\
{chary}\\
{সতর্ক,,}\\
{(adjective): cautious\\Jack was wary of GRE words that looked similar, because they usually had different definitions; not so\\with chary, a word that he began to use interchangeably with wary.\\}\\
{chauvinism}\\
{উগ্র জাতীয়তাবাদ,,}\\
{(noun): fanatical patriotism; belief that one's group/cause is superior to all other\\groups/causes\\Vegetarians argue that man is chauvinistic in his belief that animals do not consciously feel the pain we\\humans do.\\This word has other definitions but this is the most important one for the GRE\\}\\
{chimera}\\
{মিথ্যা কল্পনা,,}\\
{(noun): something desired or wished for but is only an illusion and impossible to achieve\\Many believe that a world free of war is a chimera--a dream that ignores humanity's violent tendancies.\\This word has other definitions but this is the most important one for the GRE\\}\\
{choleric}\\
{দজ্জাল,,}\\
{(adjective): prone to outbursts of temper; easily angered\\While a brilliant lecturer, Mr. Dawson came across as choleric and unapproachable--very rarely did\\students come to his office hours.\\}\\
{churlish}\\
{অভদ্র,,}\\
{(adjective): lacking manners or refinement\\The manager was unnecessarily churlish to his subordinates, rarely deigning to say hello, but always\\quick with a sartorial jab if someone happened to be wearing anything even slightly mismatching.\\}\\
{complaisant}\\
{সুশীল,,}\\
{(adjective): showing a cheerful willingness to do favors for others\\On her first day at the job, Annie was complaisant, fulfilling every request of her new employee and\\\\                                                                                \\anticipating future requests.\\}\\
{complicit}\\
{}\\
{(adjective): Associated with or participating in an activity, especially one of a questionable\\nature.\\While the grand jury cleared the senator of all criminal charges, in the public mind he was still complicit\\in the corruption.\\}\\
{conciliate}\\
{শান্ত করা,,}\\
{(verb): to make peace with\\His opponents believed his gesture to be conciliatory, yet as soon as they put down their weapons, he\\unsheathed a hidden sword.\\}\\
{concomitant}\\
{সহগামী,,}\\
{(adjective): describing an event or situation that happens at the same time as or in\\connection with another\\Concomitant with his desire for nature was a desire for the culture and energy of a big city.\\}\\
{conflagration}\\
{অগ্নিকাণ্ড,,}\\
{(noun): a very intense and uncontrolled fire\\In the summer months, conflagrations are not uncommon in the southwest, do to the heat and lack of\\rain.\\}\\
{conflate}\\
{গলিয়ে মিশিয়ে দেওয়া,,}\\
{(verb): mix together different elements or concepts\\In her recent book, the author conflates several genres--the detective story, the teen thriller, and the\\vampire romance--to create a memorable read.\\}\\
{contentious}\\
{কলহপ্রিয়,,}\\
{(adjective): likely to argue\\Since old grandpa Harry became very contentious during the summer when only reruns were on T.V., the\\grandkids learned to hide from him at every opportunity.\\}\\
{corollary}\\
{অনুসিদ্ধান্ত}\\
{(noun): a practical consequence that follows naturally\\A corollary of Hurricane Sandy, which ravaged the east coast of the U.S., is a push to build higher sea\\walls to protect against future hurricanes.\\}\\
{cosseted}\\
{আদর করা,,}\\
{(verb): treat with excessive indulgence\\The king and queen cosseted the young prince, giving him a prized miniature pony for his fifth birthday.\\}\\
{coterminous}\\
{সমব্যাপ্ত,,}\\
{(adjective): being of equal extent or scope or duration\\The border of the state is coterminous with geographic limits on travel; the east and north are\\surrounded by a nearly uncrossable river and the south by a desert.\\}\\
{countermand}\\
{আদেশ রহিত করা,,}\\
{(verb): a contrary command cancelling or reversing a previous command\\By the time the colonel countermanded his soldiers not to land in enemy territory, a few helicopters had\\already touched down amid heavy gunfire.\\}\\
{cow}\\
{গাভী}\\
{(verb): to intimidate\\\\                                                                               \\Do not be cowed by a 3,000-word vocabulary list: turn that list into a deck of flashcards!\\This word has other definitions but this is the most important one for the GRE\\}\\
{crestfallen}\\
{বিষণ্ণ,,}\\
{(adjective): brought low in spirit\\I asked Maria on a date and she refused without a moment's thought; I was crestfallen.\\}\\
{crystallize}\\
{স্ফটিকে পরিণত করা,,}\\
{(verb): cause to take on a definite and clear shape\\Only after fifteen minutes of brainstorming did Samantha's ideas for the essay crystallize.\\This word has other definitions but this is the most important one for the GRE\\}\\
{cupidity}\\
{লোভ,,}\\
{(noun): greed for money\\Some believe people that amassing as much wealth as possible is the meaning to life--yet they often\\realize that cupidity brings anything but happiness.\\}\\
{curmudgeon}\\
{বদমেজাজি লোক,,}\\
{(noun): a grouchy, surly person\\Since Uncle Mike was the family curmudgeon, each Thanksgiving he was plied with copious amounts of\\wine, in the hope that she would become less grouchy.\\}\\
{debonair}\\
{প্রফুল্ল}\\
{(adjective): having a sophisticated charm\\James Bond is known for his good looks, high tech gadgets, and debonair manner.\\}\\
{decry}\\
{নিন্দা করা}\\
{(verb): express strong disapproval of\\The entire audience erupted in shouts and curses, decrying the penalty card issued by the referee.\\}\\
{defray}\\
{বহন করা,,}\\
{(verb): to help pay the cost of, either in part of full\\In order for Sean to attend the prestigious college, his generous uncle helped defray the excessive tuition\\with a monthly donation.\\}\\
{deign}\\
{প্রসন্ন হত্তয়া,,}\\
{(verb): do something that one considers to be below one's dignity\\The master of the house never deigned to answer questions from the servants.\\}\\
{demonstrative}\\
{প্রমাণদায়ক,,}\\
{(adjective): given to or marked by the open expression of emotion\\When Sally told James that she wanted to break up with him, she expected he would react\\demonstratively, but he quietly nodded his head and left without saying a word.\\}\\
{denouement}\\
{}\\
{(noun): the final resolution of the many strands of a literary or dramatic work; the\\outcome of a complex sequence of events\\At the denouement of the movie, all questions were answered, and the true identity of the robber was\\revealed.\\}\\
{derelict}\\
{পরিত্যক্ত,,}\\
{(adjective): (of a person) not doing one's duties\\The teacher was derelict in her duties because she hadn't graded a single student paper in three weeks.\\}\\
{derelict}\\
{পরিত্যক্ত,,}\\
{(noun): (of a building) abandoned\\At one time the waterfront factories were busy and productive, but now they stand derelict and will be\\\\                                                                                \\torn down.\\}\\
{desiccated}\\
{শুকিয়ে নেওয়া,,}\\
{(adjective): uninteresting, lacking vitality\\Few novelists over 80 are able to produce anything more than desiccated works--boring shadows of\\former books.\\This word has other definitions but this is the most important one for the GRE\\}\\
{desideratum}\\
{অভীষ্ট,,}\\
{(noun): something desired as a necessity\\The desideratum of the environmental group is that motorists should rely on carpooling.\\}\\
{despot}\\
{শাসক,,}\\
{(noun): a cruel and oppressive dictator\\The Emperor Claudius was regarded as a fair-minded leader; his successor, Nero, was an absolute\\despot.\\}\\
{diatribe}\\
{গালিগালাজ,,}\\
{(noun): a strong verbal attack against a person or institution\\Steve's mom launched into a diatribe during the PTA meeting, contending that the school was little more\\than a daycare in which students stare at the wall and teachers stare at the chalkboard.\\}\\
{diminutive}\\
{সঙ্কুচিত,,}\\
{(noun): to indicate smallness\\He prefers to be called a diminutive of his name: "Bill" instead of "John William."\\}\\
{diminutive}\\
{সঙ্কুচিত,,}\\
{(adjective): very small\\When he put on his father's suit and shoes, his appearance was that of a diminutive youth.\\}\\
{disabuse}\\
{ভ্রান্ত ধারণাদি হইতে মুক্ত করা,,}\\
{(verb): to persuade somebody that his/her belief is not valid\\As a child, I was quickly disabused of the notion that Santa Claus was a rotund benefactor of infinite\\largess--one night I saw my mother diligently wrapping presents and storing them under our Christmas\\tree.\\}\\
{discursive}\\
{অপ্রাসঙ্গিক,,}\\
{(adjective): (of e.g. speech and writing) tending to depart from the main point\\Many readers find it tough to read Moby Dick since the author is discursive, often cutting the action\\short to spend 20 pages on the history of a whale.\\}\\
{disingenuous}\\
{কপট,,}\\
{(adjective): not straightforward; giving a false appearance of frankness\\Many adults think that they can lie to children, but kids are smart and know when people are\\disingenuous.\\}\\
{dispensation}\\
{বিধান,,}\\
{(noun): an exemption from a rule or obligation\\Since her father is a billionaire, she is given dispensation from many of the school's policies.\\This word has other definitions but this is the most important one for the GRE\\}\\
{dissemble}\\
{ছদ্মবেশ ধারণ করা,,}\\
{(verb): conceal one's true motives, usually through deceit\\To get close to the senator, the assassin dissembled his intentions, convincing many people that he was a\\reporter for a well-known newspaper.\\\\                                                                               \\}\\
{dissipate}\\
{উড়া,,}\\
{(verb): squander or spend money frivolously\\The recent graduates dissipated their earnings on trips to Las Vegas and cruises in Mexico.\\}\\
{dissipate}\\
{উড়া,,}\\
{(verb): to disperse or scatter\\Kathleen's perfume was overwhelming in the cramped apartment, but once we stepped outside the smell\\dissipated and we could breathe once again.\\}\\
{dissolution}\\
{দ্রবণ,,}\\
{(noun): a living full of debauchery and indulgence in sensual pleasure\\Many Roman emporers were known for their dissolution, indulging in unspeakable desires of the flesh.\\This word has other definitions but this is the most important one for the GRE\\}\\
{doleful}\\
{বেদনাপূর্ণ,,}\\
{(adjective): filled with or evoking sadness\\No event is more doleful than the passing of my mother; she was a shining star in my life, and it brings\\me great sadness to think that she is now gone.\\}\\
{dolorous}\\
{শোকপূর্ণ,,}\\
{(adjective): showing sorrow\\Chopin's ballades are filled with sharp changes in moods--a dolorous melody can give way to a\\lighthearted tempo.\\}\\
{doughty}\\
{বলবান,,}\\
{(adjective): brave; bold; courageous\\I enjoy films in which a doughty group comes together to battle a force of evil.\\}\\
{dovetail}\\
{খাঁজে খাঁজে আটকান,,}\\
{(verb): fit together tightly, as if by means of a interlocking joint\\Although Darwin's evolution and Mendel's genetics were developed in isolation from one another, they\\dovetail each other very well.\\This word has other definitions but this is the most important one for the GRE\\}\\
{duplicity}\\
{কাপট্য,,}\\
{(noun): deceitfulness, pretending to want one thing but interested in something else\\A life of espionage is one of duplicity: an agent must pretend to be a totally different person than who\\she or he actually is.\\}\\
{ebullient}\\
{অত্যুত্সাহী,,}\\
{(adjective): joyously unrestrained\\Can you blame him for his ebullient mood? He just graduated from medical school.\\}\\
{effervescent}\\
{বুদ্বুদপূর্ণ,,}\\
{(adjective): marked by high spirits or excitement\\After the sales result, the manager was in an effervescent mood, letting several employees leave work\\early that day.\\This word has other definitions but this is the most important one for the GRE\\}\\
{effrontery}\\
{বেহায়াপনা,,}\\
{(noun): audacious (even arrogant) behavior that you have no right to\\The skateboarders acted with effrontery, skating through the church grounds and spray-painting signs\\warning trespassers.\\}\\
{elegiac}\\
{করুণ,,}\\
{(adjective): expressing sorrow\\Few can listen to the elegiac opening bars of the Moonlight sonata without feeling the urge to cry.\\\\                                                                                \\}\\
{embryonic}\\
{আদিম,,}\\
{(adjective): in an early stage of development\\The Board of Directors is hoping to launch a new product soon, but planning for the Z7 is in an embryonic\\stages.\\This word has other definitions but this is the most important one for the GRE\\}\\
{empiricism}\\
{প্রয়োগবাদ,,}\\
{(noun): any method that derives knowledge from experience, used in experimental science\\as a way to gain insight and knowledge\\Empiricism does not always lead to knowledge; an experience or experiment may raise more questions\\than it answers.\\}\\
{enamor}\\
{মুগ্ধ করা,,}\\
{(verb): attraction or feeling of love\\She is completely enamored with Justin Bieber, and goes to all his concerts on the East coast.\\}\\
{encumber}\\
{ভারগ্রস্ত করা,,}\\
{(verb): hold back\\The costume encumbered all my movements and caused me to sweat profusely.\\}\\
{enjoin}\\
{নির্দেশ দান করা}\\
{(verb): give instructions to or direct somebody to do something with authority\\The government agency enjoined the chemical company to clean up the hazardous dump it had created\\over the years.\\}\\
{enormity}\\
{বিশালতা,,}\\
{(noun): an act of extreme wickedness\\The enormity of Pol Pot's regime is hard to capture in words--within months hundreds of thousands of\\Cambodians lost their lives.\\This word has other definitions but this is the most important one for the GRE\\}\\
{enthrall}\\
{অভিভূত করা}\\
{(verb): hold spellbound\\She was so enthralled by the movie that she never heard people screaming, "Fire! Fire!" in the\\neighboring theater.\\}\\
{epigram}\\
{শ্লেষ,,}\\
{(noun): a witty saying\\My favorite epigram from Mark Twain is "A man who carries a cat by the tail learns something he can\\learn no other way."\\}\\
{epiphany}\\
{নিকট যীশুর আবির্ভাব,,}\\
{(noun): a sudden revelation or moment of insight\\Gary one day had an epiphany that he was a people person; he prompty quit his factory job and began\\working as a salesman.\\}\\
{eponym}\\
{}\\
{(noun): the name derived from a person (real or imaginary); the person for whom something is\\named\\Alexandria, Egypt is an eponym because it is named after Alexander the Great.\\}\\
{equitable}\\
{ন্যায়সঙ্গত,,}\\
{(adjective): fair to all parties as dictated by reason and conscience\\The equitable distribution of ice cream to a group of 5 year olds will ensure little to no fighting--at least\\until the ice cream is gone.\\\\                                                                                \\}\\
{equivocate}\\
{বাক্চাতুরী করা,,}\\
{(verb): to speak vaguely, usually with the intention to mislead or deceive\\After Sharon brought the car home an hour after her curfew, she equivocated when her parents\\pointedly asked her where she had been.\\}\\
{ersatz}\\
{বদলি লোক বস্তু,,}\\
{(adjective): not real or genuine; phony\\The car dealer's ersatz laughter was immediately followed by a price quote, one that Shelley found highly\\inflated.\\}\\
{estimable}\\
{শ্রদ্ধেয়,,}\\
{(adjective): deserving of esteem and respect\\After serving thirty years, in which he selflessly served the community, Judge Harper was one of the more\\estimable people in town.\\}\\
{ethereal}\\
{গগনচারী,,}\\
{(adjective): characterized by lightness and insubstantiality\\Because she dances with an ethereal style, ballet critics have called her Madame Butterfly.\\}\\
{evanescent}\\
{বিলীয়মান,,}\\
{(adjective): tending to vanish like vapor\\The storm flashed into existence above us and lasted only a short time--an evanescent turbulence of\\wind and cloud.\\}\\
{excoriate}\\
{ছাল ছাড়ান,,}\\
{(verb): to criticize very harshly\\Entrusted with the prototype to his company's latest smartphone, Larry, during a late night karaoke\\bout, let the prototype slip into the hands of a rival company--the next day Larry was excoriated, and\\then fired.\\}\\
{execrate}\\
{ঘৃণা করা,,}\\
{(verb): to curse and hiss at\\Though the new sitcom did decently in the ratings, Nelson railed against the show, saying that it was\\nothing more than an execrable pastiche of tired cliché's and canned laughter.\\}\\
{exegesis}\\
{ব্যাখ্যা,,}\\
{(noun): critical explanation or analysis, especially of a text\\The Bible is fertile ground for exegesis--over the past five centuries there have been as many\\interpretations as there are pages in the Old Testament.\\}\\
{exemplar}\\
{আদর্শ,,}\\
{(noun): something to be imitated\\Lena's homework is on the wall because it is an exemplar of clean, neat, and thoughtful work.\\}\\
{exiguity}\\
{দীনতা,,}\\
{(noun): the quality of being meager\\After two months at sea, the exiguity of the ship's supplies forced them to search for fresh water and\\food.\\}\\
{exorbitant}\\
{গলাকাটা,,}\\
{(adjective): greatly exceeding bounds of reason or moderation\\Shelley made one exorbitant purchase after another, buying new clothes and taking vacations even\\though she earned a limited salary.\\\\                                                                                 \\}\\
{expansive}\\
{বিস্তৃত}\\
{(adjective): communicative, and prone to talking in a sociable manner\\After a few sips of cognac, the octogenarian shed his irascible demeanor and became expansive,\\speaking fondly of the "good old days".\\This word has other definitions but this is the most important one for the GRE\\}\\
{expunge}\\
{মুছিয়া ফেলা,,}\\
{(verb): to eliminate completely\\When I turned 18, all of the shoplifting and jaywalking charges were expunged from my criminal record.\\}\\
{expurgate}\\
{শোধন করা,,}\\
{(verb): to remove objectionable material\\The censor expurgated every reference to sex and drugs, converting the rapper's raunchy flow into a\\series of bleeps.\\}\\
{extrapolate}\\
{পূর্বেই দেখা,,}\\
{(verb): draw from specific cases for more general cases\\By extrapolating from the data on the past three months, we can predict a 5% increase in traffic to our\\website.\\}\\
{facile}\\
{সহজ,,}\\
{(adjective): arrived at without due care or effort; lacking depth\\Many news shows provide facile explanations to complex politics, so I prefer to read the in-depth\\reporting of The New York Times.\\}\\
{factious}\\
{দুর্দান্ত,,}\\
{(adjective): produced by, or characterized by internal dissension\\The controversial bill proved factious, as dissension even within parties resulted\\}\\
{factitious}\\
{কপট,,}\\
{(adjective): artificial; not natural\\The defendant's story was largely factitious and did not accord with eyewitness testimonies\\}\\
{feckless}\\
{অসহায়,,}\\
{(adjective): lazy and irresponsible\\Two years after graduation, Charlie still lived with his parents and had no job, becoming more feckless\\with each passing day.\\}\\
{fecund}\\
{উর্বর,,}\\
{(adjective): intellectually productive\\The artist had entered a fecund period, producing three masterpieces in the span of two months.\\}\\
{fell}\\
{হিংস্র,,}\\
{(adjective): terribly evil\\For fans of the Harry Potter series, the fell Lord Voldemort, who terrorized poor Harry for seven lengthy\\installments, has finally been vanquished by the forces of good--unless, that is, JK Rowling decides to\\come out of retirement.\\This word has other definitions but this is the most important one for the GRE\\}\\
{firebrand}\\
{জ্বলন্ত কাষ্ঠখণ্ড,,}\\
{(noun): someone who deliberately creates trouble\\Freddie is a firebrand: every time he walks into the office, he winds up at the center of heated argument.\\}\\
{flag}\\
{পতাকা}\\
{(verb): droop, sink, or settle from or as if from pressure or loss of tautness; become less intense\\After the three crushing defeats in the last three games, the team's enthusiasm began to flag.\\This word has other definitions but this is the most important one for the GRE\\\\                                                                               \\}\\
{flippant}\\
{বাচাল,,}\\
{(adjective): showing inappropriate levity\\Although Sam was trying to honor Mark's sense of humor, many found it quite flippant that he wore a\\comic nose and glasses mask to Mark's funeral.\\}\\
{flummox}\\
{বিভ্রান্ত করা,,}\\
{(verb): be a mystery or bewildering to\\Mary's behavoir completely flummoxes me: I never have any idea what her motivations might be.\\}\\
{fractious}\\
{ঝগড়াটে,,}\\
{(adjective): irritable and is likely to cause disruption\\We rarely invite my fractious Uncle over for dinner; he always complains about the food, and usually\\launches into a tirade on some touchy subject.\\}\\
{gaffe}\\
{অসমীচীন কাজ,,}\\
{(noun): a socially awkward or tactless act\\In a famous gaffe, Vice President Quayle attempted to correct the spelling of a grade school student,\\only to find that the child was correct.\\}\\
{gambit}\\
{}\\
{(noun): a manuveur or risk in a game or conversation, designed to secure an advantage\\Randy played a gambit, telling his boss that he would leave at the end of the week if he didn't get a raise.\\}\\
{gerrymander}\\
{কূটকৌশল,,}\\
{(verb): to manipulate voting districts in order to favor a particular political party\\Years ago, savvy politicians had gerrymandered the city center to ensure their re-election.\\}\\
{graft}\\
{কলম করা,,}\\
{(noun): corruption, usually through bribery\\In countries with rampant graft, getting a driver's license can require no more than paying an official.\\This word has other definitions but this is the most important one for the GRE\\}\\
{grandiloquent}\\
{বাগাড়ম্বরপূর্ণ,,}\\
{(adjective): puffed up with vanity\\The dictator was known for his grandiloquent speeches, puffing his chest out and using big, important-\\sounding words.\\}\\
{gumption}\\
{বুঝ,,}\\
{(noun): resourcefulness and determination\\Wallace Stegner lamented the lack of gumption in the U.S. during the sixties, claiming that no young\\person knew the value of work.\\}\\
{hagiographic}\\
{}\\
{(adjective): excessively flattering toward someone's life or work\\Most accounts of Tiger Woods life were hagiographic, until, that is, his affairs made headlines.\\}\\
{hail}\\
{অভিনন্দন জানান}\\
{(verb): enthusiastically acclaim or celebrate something\\Many college superstar athletes are hailed as the next big thing, but then flop at the professional level.\\This word has other definitions but this is the most important one for the GRE\\}\\
{halcyon}\\
{মাছরাঙা}\\
{(adjective): idyllically calm and peaceful; suggesting happy tranquillity; marked by peace and\\prosperity\\The first decade after WWI was a halcyon period in America with new-found wealth and rapidly\\improving technology.\\\\                                                                                \\}\\
{hauteur}\\
{ঔদ্ধত্য,,}\\
{(noun): overbearing pride evidenced by a superior manner toward inferiors\\As soon as she won the lottery, Alice begin displaying a hauteur to her friends, calling them dirty-clothed\\peasants behind their backs.\\}\\
{hector}\\
{তর্জন - গর্জন করিয়া কর্তৃত্ব জাহির করা,,}\\
{(verb): to bully or intimidate\\The boss's hectoring manner put off many employees, some of whom quit as soon as they found new\\jobs.\\}\\
{hedge}\\
{প্রতিবন্ধক,,}\\
{(verb): to limit or qualify a statement; to avoid making a direct statement\\When asked why he had decided to buy millions of shares at the very moment the tech companies stock\\soared, the CEO hedged, mentioning something vague about gut instinct.\\This word has other definitions but this is the most important one for the GRE\\}\\
{histrionic}\\
{নাটুকে,,}\\
{(adjective): to be overly theatrical\\Though she received a B- on the test, she had such a histrionic outburst that one would have thought\\that she'd been handed a death sentence.\\}\\
{hoary}\\
{শুক্ল,,}\\
{(adjective): ancient\\Most workout gurus are young, fit people, whereas most yoga gurus are hoary men with long white\\beards.\\}\\
{hobble}\\
{মুশকিল,,}\\
{(verb): to hold back the progress of something\\Bad weather has hobbled rescue efforts, making it difficult for crews to find bodies in the wreckage.\\}\\
{hoodwink}\\
{ভাঁড়ান,,}\\
{(verb): to deceive or trick someone\\Someone tried to hoodwink Marty with an email telling him that his uncle had just passed away, and to\\collect the inheritance he should send his credit card information.\\}\\
{hubris}\\
{ঔদ্ধত্য,,}\\
{(noun): overbearing pride or presumption\\Bill Clinton was criticized for his hubris, since he believed he could get away with anything once in the\\White House.\\}\\
{illustrious}\\
{প্রসিদ্ধ,,}\\
{(adjective): widely known and esteemed; having or conferring glory\\Einstein was possibly the most illustrious scientist in recent history.\\}\\
{imbibe}\\
{পান করা}\\
{(verb): to drink or absorb as if drinking\\Plato imbibed Socrates' teachings to such an extent that he was able to write volumes of work that he\\directly attributed, sometimes word for word, to Socrates.\\}\\
{imbroglio}\\
{জট,,}\\
{(noun): a confusing and potentially embarrassing situation\\The chef cook-off featured one gourmand who had the unfortunate distinction of mixing the wrong\\broths, creating an imbroglio that diners would not soon forget.\\}\\
{immure}\\
{কয়েদ করা,,}\\
{(verb): to enclose, usually in walls\\\\                                                                               \\The modern supermarket experience makes many feel claustrophobic, as they are immured in walls upon\\walls of products.\\}\\
{impecunious}\\
{গরিব,,}\\
{(adjective): lacking money; poor\\In extremely trying times, even the moderately wealthy, after a few turns of ill-fortune, can become\\impecunious.\\}\\
{imperious}\\
{স্বেচ্ছাচারী,,}\\
{(adjective): having or showing arrogant superiority to and disdain of those one views as\\unworthy\\Children are imperious with each other before they learn that earning someone's respect is better than\\demanding.\\}\\
{impervious}\\
{অভেদ্য,,}\\
{(adjective): not admitting of passage or capable of being affected\\I am not impervious to your insults; they cause me great pain.\\}\\
{impetuous}\\
{বেগবান,,}\\
{(adjective): characterized by undue haste and lack of thought or deliberation\\Herbert is rarely impetuous, but on the spur of the moment, he spent thousands of dollars on a\\motorcycle today.\\}\\
{importuned}\\
{নিরতিশয় জিদ করা,,}\\
{(verb): beg persistently and urgently\\After weeks of importuning the star to meet for a five-minute interview, the journalist finally got what\\she wanted.\\}\\
{improvident}\\
{অপরিণামদর্শী,,}\\
{(adjective): not given careful consideration\\Marty was improvident, never putting money aside for the future but spending it on decorating the\\interior of his home.\\}\\
{impugn}\\
{সত্যতা অস্বীকার করা,,}\\
{(verb): attack as false or wrong\\Though many initially tried to impugn Darwin's theory, in scientific circles today, the is idea taken as\\truth.\\}\\
{impute}\\
{আরোপ করা,,}\\
{(verb): attribute (responsibility or fault) to something\\He imputed his subpar performance on the test to a combination of stress and poor sleep.\\}\\
{inanity}\\
{অর্থহীনতা,,}\\
{(noun): total lack of meaning or ideas\\Bill's poem was nothing more than a list of impressive sounding words, so there was no point in trying to\\take meaning from the inanity.\\}\\
{inchoate}\\
{অপরিণত,,}\\
{(adjective): only partly in existence; imperfectly formed\\Inchoate ideas about the relation of humans to other animals had been discussed since the Middle Ages\\but the modern theory really began with Darwin.\\}\\
{incontrovertible}\\
{সুনিশ্চিত,,}\\
{(adjective): necessarily or demonstrably true; impossible to deny or disprove\\\\                                                                                \\Unless you can provide incontrovertible evidence, I will remain skeptical.\\}\\
{indigent}\\
{অভাবী,,}\\
{(adjective): poor; having very little\\In the so-called Third World, many are indigent and only a privileged few have the resources to enjoy\\material luxuries.\\}\\
{indigent}\\
{অভাবী,,}\\
{(noun): a poor or needy person\\The indigents, huddled under the overpass, tried to start a small bonfire in the hope of staying warm.\\}\\
{ineffable}\\
{অবর্ণনীয়,,}\\
{(adjective): too sacred to be uttered; defying expression or description\\While art critics can occasionally pinpoint a work's greatness, much of why a piece captures our\\imaginations is completely ineffable.\\}\\
{ineluctable}\\
{অনিবার্য,,}\\
{(adjective): impossible to avoid or evade:\\For those who smoke cigarettes for years, a major health crisis brought on by smoking is ineluctable.\\}\\
{inequity}\\
{অবিচার,,}\\
{(noun): injustice by virtue of not conforming with rules or standards\\After decades of racial inequity, the "separate but equal" doctrine was successfully overturned.\\}\\
{infelicitous}\\
{অযথাযথ,,}\\
{(adjective): inappropriate\\During the executive meeting, the marketing director continued to make infelicitous comments about the\\CEO's gambling habit.\\}\\
{inimitable}\\
{অননুকরণীয়}\\
{(adjective): defying imitation; matchless\\Mozart's music follows a clear pattern that, anyone could imitate, but his music gives an overall sense of\\effortlessness that is inimitable.\\}\\
{insouciance}\\
{ঔদাসীন্য,,}\\
{(noun): lack of concern\\Surprisingly, Hank had become a high-powered CEO; his high school friends remembered him as "Hanky\\Panky", who shrugged off each failed class with insouciance.\\}\\
{insufferable}\\
{অসহ্য,,}\\
{(adjective): intolerable, difficult to endure\\Chester always tried to find some area in which he excelled above others; unsurprisingly, his co-workers\\found him insufferable and chose to exclude him from daily luncheons out.\\}\\
{internecine}\\
{মারাত্মক,,}\\
{(adjective): (of conflict) within a group or organization\\The guerilla group, which had become so powerful as to own the state police, was finally destroyed by an\\internecine conflict.\\}\\
{intimation}\\
{অন্তরঙ্গতা}\\
{(noun): an indirect suggestion\\At first the hostess tried intimation, praising the benefits of cutlery; when Cecil continued eating with his\\hands, the hostess told him to use a fork at dinner.\\\\                                                                                   \\}\\
{inure}\\
{কার্যকর করা,,}\\
{(verb): to make accustomed to something unpleasant\\Three years of Manhattan living has inured her to the sound of wailing sirens; she could probably sleep\\through the apocalypse.\\}\\
{invective}\\
{আক্রমণমূলক,,}\\
{(noun): abusive or denunciatory language\\The Internet has unleashed the invectives in many of us; many people post stinging criticism on the\\comments section underneath newspaper articles or YouTube videos.\\}\\
{invidious}\\
{বাস্তব বা কাল্পনিক অবিচারপ্রসূত,,}\\
{(adjective): likely to cause resentment\\At a time when many others in the office were about to be laid off, many considered Cheryl's fine clothes\\that day an invidious display.\\This word has other definitions but this is the most important one for the GRE\\}\\
{inviolable}\\
{অলঙ্ঘনীয়,,}\\
{(adjective): never to be broken, infringed, or dishonored\\Too many the grass at Wimbledon is inviolable and only greater tennis players are able to enjoy a game\\there.\\}\\
{inviolate}\\
{অলঙ্ঘিত,,}\\
{(adjective): must be kept sacred\\While the literary critic subjected most of the classics to the harshest reviews, he regarded Cervantes as\\inviolate, and had nothing but praise for him.\\}\\
{irrefutable}\\
{অকাট্য,,}\\
{(adjective): impossible to deny or disprove\\The existence of life on earth is irrefutable; the existence of intelligent life on earth is still hotly debated.\\}\\
{jaundice}\\
{ন্যাবা,,}\\
{(adjective): to be biased against due to envy or prejudice\\Shelly was jaundiced towards Olivia; though the two had once been best friends, Olivia had become class\\president, prom queen, and, to make matters worse, the girlfriend of the one boy Shelly liked.\\}\\
{jejune}\\
{নীরস,,}\\
{(adjective): dull; lacking flavor\\Although many top chefs have secured culinary foam's popularity in haute cuisine, Waters criticizes it for\\being jejune and unfilling.\\}\\
{jejune}\\
{নীরস,,}\\
{(adjective): immature; childish\\Her boss further cemented his reuptation for being jejune after throwing a fit when the water cooler\\wasn't refilled.\\}\\
{jingoist}\\
{}\\
{(noun): a person who thinks that their country should be at war\\In the days leading up to war, a nation typically breaks up into the two opposing camps: doves, who do\\their best to avoid war, and jingoists, who are only too eager to wave national flags from their vehicles\\and vehemently denounce those who do not do th\\}\\
{juggernaut}\\
{জগন্নাথদেব,,}\\
{(noun): a force that cannot be stopped\\Napoleon was considered a juggernaut until he decided to invade Russia in winter; after which, his once\\indomitable army was decimated by cold and famine.\\\\                                                                               \\}\\
{kowtow}\\
{সাষ্টাঙ্গ প্রণাম,,}\\
{(verb): to bow or act in a subservient manner\\Paul kowtowed to his boss so often the boss herself became nauseated by his sycophancy.\\}\\
{lacerate}\\
{বিদীর্ণ করা,,}\\
{(verb): deeply hurt the feelings of; distress\\The teacher was fired for lacerating a student who wrote a poor essay.\\This word has other definitions but this is the most important one for the GRE\\}\\
{lachrymose}\\
{ক্রন্দনরত,,}\\
{(adjective): showing sorrow\\Lachrymose and depressed, Alexei Alexandrovich walked two miles home in the rain after learning that\\his wife was having an affair.\\}\\
{lampoon}\\
{ব্যক্তিগত ব্যঙ্গ - কবিতা,,}\\
{(verb): ridicule with satire\\Mark Twain understood that lampooning a bad idea with humor was the most effective criticism.\\}\\
{languish}\\
{দুর্বলতা,,}\\
{(verb): become feeble\\Stranded in the wilderness for four days, the hiker languished, eating protein bars and nuts.\\}\\
{lascivious}\\
{লম্পট,,}\\
{(adjective): lecherous; sexually perverted\\Lolita is a challenging novel for many, not necessarily because of the elevated prose style but because of\\the depravity of the main character, Humbert Humbert, who, as an old, lascivious man, falls in love with\\a girl.\\}\\
{limpid}\\
{নির্মল,,}\\
{(adjective): clarity in terms of expression\\Her limpid prose made even the most recondite subjects accessible to all.\\}\\
{litany}\\
{প্রার্থনা - সঙ্গীত,,}\\
{(noun): any long and tedious account of something\\Mr. Rogers spoke to a Senate committee and did not give a litany of reasons to keep funding the\\program, but instead, appealed to the basic human decency of all present.\\}\\
{loath}\\
{ঘৃণা করা}\\
{(adjective): unwillingness to do something contrary to your custom (usually followed by 'to')\\I was loath to leave the concert before my favorite band finished playing.\\}\\
{lugubrious}\\
{বিষণ্ণ,,}\\
{(adjective): excessively mournful\\At the funeral, lugubrious songs filled the small church.\\}\\
{machinate}\\
{অভিসন্ধি করা,,}\\
{(verb): engage in plotting or enter into a conspiracy, swear together\\The rebels met at night in an abandoned barn to machinate.\\}\\
{magisterial}\\
{গর্বিত,,}\\
{(adjective): offensively self-assured or given to exercising unwarranted power\\Though she was only a third grade teacher, Ms. Martinet was magisterial in dealing with her class,\\lording over them like a queen.\\This word has other definitions but this is the most important one for the GRE\\}\\
{malapropism}\\
{শব্দের অপপ্রয়োগ,,}\\
{(noun): the confusion of a word with another word that sounds similar\\\\                                                                               \\Whenever I looked glum, my mother would offer to share "an amusing antidote" with me--an endearing\\malapropism of "anecdote" that never failed to cheer me up.\\}\\
{malfeasance}\\
{কুকর্ম,,}\\
{(adjective): misconduct or wrongdoing (especially by a public official)\\Not even the mayor's trademark pearly-toothed grin could save him from charges of malfeasance: while\\in power, he'd been running an illegal gambling rink in the room behind his office.\\}\\
{malingerer}\\
{রোগভানকারী,,}\\
{(noun): someone shirking their duty by pretending to be sick or incapacitated\\At one time, our country was full of hardworking respectful people, but now it seems that everyone is a\\malingerer with little inclination to work.\\}\\
{martinet}\\
{কঠোর নিয়মনিষ্ঠ ব্যক্তি,,}\\
{(noun): a strict disciplinarian\\The job seemed perfect to Rebecca, until she found out that her boss was a total martinet; after each\\project the boss would come by to scrutinize--and inevitably criticize--every little detail of the work\\Rebecca had done.\\}\\
{maudlin}\\
{মূর্খ,,}\\
{(adjective): overly emotional and sad\\Just as those who were alive during the 70's are mortified that they once cavorted about in bellbottoms,\\many who lived during the 80's are now aghast at the maudlin pop songs they used to enjoy--really, just\\what exactly is a total eclipse of the heart?\\}\\
{maunder}\\
{বক্বক্,,}\\
{(verb): wander aimlessly\\Max liked to maunder down by the seaside and pick up whatever sea shells he would stumble upon.\\}\\
{maunder}\\
{বক্বক্,,}\\
{(verb): speak (about unimportant matters) rapidly and incessantly\\After drinking two expressos each, the animated couple would maunder loudly, annoying the other\\patrons in the coffee shop.\\}\\
{mellifluous}\\
{মসৃণ,,}\\
{(adjective): smooth and sweet-sounding\\Chelsea's grandmother thought Franz Schubert's music to be the most mellifluous ever written; Chelsea\\demurred, and to her grandmother's chagrin, would blast Rihanna on the home stereo speakers.\\}\\
{mendicant}\\
{ভিক্ষাজীবী,,}\\
{(noun): a pauper who lives by begging\\Tolstoy was an aristocrat, but he strove to understand the Christianity of the Russian peasants by\\wandering among them as a mendicant.\\}\\
{meteoric}\\
{ক্ষণপ্রভ,,}\\
{(adjective): like a meteor in speed or brilliance or transience\\The early spectacular successes propelled the pitcher to meteoric stardom, but a terribly injury tragically\\cut short his career.\\}\\
{mettlesome}\\
{তেজস্বী,,}\\
{(adjective): filled with courage or valor\\For its raid on the Bin Laden's compound in Abbottabad, Seal Team Six has become, for many Americans,\\the embodiment of mettle.\\\\                                                                               \\}\\
{misattribute}\\
{}\\
{(verb): To erroneously attribute; to falsely ascribe; used especially of authorship.\\I made a mistake; I misattributed "Crime and Punishment" to Leo Tolstoy when it was actually written by\\Fyodor Dostoyevsky.\\}\\
{modicum}\\
{যৎকিঞ্চিৎ,,}\\
{(noun): a small or moderate or token amount\\If my sister had even a modicum of sense, she wouldn't be engaged to that barbarian.\\}\\
{mordant}\\
{জ্বালাময়,,}\\
{(adjective): biting and caustic in thought, manner, or style\\While Phil frequently made mordant remarks about company policy overall, he always was considerably\\gentler in discussing any person in particular.\\}\\
{moribund}\\
{মরমর,,}\\
{(adjective): being on the point of death; declining rapidly losing all momentum in progress\\Whether you like it or not, jazz as a genre is moribund at best, possibly already dead.\\}\\
{mulct}\\
{জরিমানা,,}\\
{(verb): to defraud or swindle\\The so-called magical diet cure simply ended up mulcting Maria out of hundreds of dollars, but did\\nothing for her weight.\\}\\
{nadir}\\
{কুবিন্দু}\\
{(noun): the lowest point\\For many pop music fans, the rap and alternative-rock dominated 90s were the nadir of musical\\expression.\\}\\
{nettlesome}\\
{}\\
{(adjective): causing irritation or annoyance\\Maria found her coworkers cell phone nettlesome, because every few minutes it would buzz to life with\\another text message.\\}\\
{noisome}\\
{কুদর্শন,,}\\
{(adjective): having an extremely bad smell\\Each August, when the winds moved in a south easterly direction, the garbage dump would spread\\noisome vapors through the small town.\\}\\
{nonchalant}\\
{উদাসীন,,}\\
{(adjective): coming across as cooly uninterested\\The twenty-somethings at the coffee shop always irked Sheldon, especially the way in which they acted\\nonchalantly towards everything, not even caring when Sheldon once spilled his mocha on them.\\}\\
{objurgate}\\
{তিরস্কার করা,,}\\
{(verb): express strong disapproval of\\The manager spent an hour objurgating the employee in the hopes that he would not make these\\mistakes again.\\}\\
{oblique}\\
{বাঁকানো}\\
{(adjective): not straightforward; indirect\\Herbert never explicitly revealed anything negative about Tom's past, but at times he would obliquely\\suggest that Tom was not as innocent as he seemed.\\This word has other definitions but this is the most important one for the GRE\\\\                                                                                 \\}\\
{obstreperous}\\
{দুরন্ত,,}\\
{(adjective): noisily and stubbornly defiant; willfully difficult to control\\When the teacher asked the obstreperous student simply to bus his tray, the student threw the entire\\tray on the floor, shouted an epithet, and walked out.\\}\\
{obtain}\\
{অর্জন করা}\\
{(adjective): be valid, applicable, or true\\The custom of waiting your turn in line does not obtain in some countries, in which many people try to\\rush to front of the line at the same time.\\This word has other definitions but this is the most important one for the GRE\\}\\
{obtuse}\\
{অসাড়,,}\\
{(adjective): slow to learn or understand; lacking intellectual acuity; lacking in insight or\\discernment\\Jackson was the most obtuse member of the team: the manager's subtle ironies were always lost on him.\\}\\
{officious}\\
{উপযাচক,,}\\
{(adjective): intrusive in a meddling or offensive manner\\The professor had trouble concentrating on her new theorem, because her officious secretary would\\barge in frequently reminding her of some trivial detail involving departmental paperwork.\\}\\
{ossify}\\
{শক্ত করা,,}\\
{(verb): make rigid and set into a conventional pattern\\Even as a young man, Bob had some bias against poor people, but during his years in social services, his\\bad opinions ossified into unshiftable views.\\This word has other definitions but this is the most important one for the GRE\\}\\
{overweening}\\
{দাম্ভিক,,}\\
{(adjective): arrogant; presumptuous\\Mark was so convinced of his basketball skills that in his overweening pride he could not fathom that his\\name was not on the varsity list; he walked up to the basketball coach and told her she had forgotten to\\add his name.\\}\\
{palatable}\\
{স্বাদু,,}\\
{(adjective): acceptable to the taste or mind\\MIkey didn't partake much in his friends' conversations, but found their presence palatable.\\This word has other definitions but this is the most important one for the GRE\\}\\
{palaver}\\
{ঘোঁট,,}\\
{(verb): speak (about unimportant matters) rapidly and incessantly\\During the rain delay, many who had come to see the game palavered, probably hoping that idle chatter\\would make the time go by faster.\\This word has other definitions but this is the most important one for the GRE\\}\\
{palimpsest}\\
{যে পাণ্ডুলিপিতে নূতন লেখা ঢুকাইবার জন্য মূল লেখা ঘষিয়া তুলিয়া ফেলা হইয়াছে,,}\\
{(noun): something that has been changed numerous times but on which traces of former\\iterations can still be seen\\The downtown was a palimpsest of the city's checkered past: a new Starbucks had opened up next to an\\abandoned, shuttered building, and a freshly asphalted road was inches away from a pothole large\\enough to swallow a small dog.\\}\\
{panacea}\\
{সর্বব্যাধিহর ঔষধ,,}\\
{(noun): hypothetical remedy for all ills or diseases; a universal solution\\While the company credit card has made most large purchases easier, it is no panacea: some smaller\\basic transactions still must be conducted in cash.\\}\\
{panegyric}\\
{স্তুতি,,}\\
{(noun): a formal expression of praise\\\\                                                                              \\Dave asked Andrew to do just a simple toast, but Andrew launched into a full panegyric, enumerating a\\complete list of Dave's achievements and admirable qualities.\\}\\
{paragon}\\
{উত্কর্ষের আদর্শ,,}\\
{(noun): model of excellence or perfection of a kind; one having no equal\\Even with the rise of Kobe Bryant, many still believe that Michael Jordon is the paragon for basketball\\players.\\}\\
{paragon}\\
{উত্কর্ষের আদর্শ,,}\\
{(noun): an ideal instance; a perfect embodiment of a concept\\Some say that Athens was the paragon of democracy, but these people often forget that slaves and\\women were still not allowed to vote.\\}\\
{pariah}\\
{জাতিচু্যত ব্যক্তি,,}\\
{(noun): an outcast\\The once eminent scientist, upon being found guilty of faking his data, has become a pariah in the\\research community.\\}\\
{parvenu}\\
{ভুঁইফোঁড় ব্যক্তি,,}\\
{(noun): a person who has suddenly become wealthy, but not socially accepted as part of a\\higher class\\The theater was full of parvenus who each thought that they were surrounded by true aristocrats.\\}\\
{patent}\\
{পেটেন্ট}\\
{(adjective): glaringly obvious\\Since the book had been through no fewer than six proof runs, the staff was shocked to see such a patent\\spelling mistake remaining, right in the middle of the front cover!\\}\\
{pecuniary}\\
{আর্থিক,,}\\
{(adjective): relating to or involving money\\The defendent was found guilty and had to serve a period of community service as well as pay pecuniary\\damages to the client.\\}\\
{pellucid}\\
{কাকচক্ষু,,}\\
{(adjective): transparently clear; easily understandable\\The professor had a remarkable ability make even the most difficult concepts seem pellucid.\\}\\
{penurious}\\
{নগণ্য,,}\\
{(adjective): lacking money; poor\\Truly penurious, Mary had nothing more than a jar full of pennies.\\}\\
{penurious}\\
{নগণ্য,,}\\
{(adjective): miserly\\Warren Buffett, famous multi-billionaire, still drives a cheap sedan, not because he is penurious, but\\because luxury cars are gaudy and impractical.\\}\\
{percipient}\\
{প্রত্যক্ষ করে এমন,,}\\
{(adjective): highly perceptive\\Even the most percipient editor will make an occasional error when proofreading.\\}\\
{peremptory}\\
{সুদৃঢ়,,}\\
{(adjective): bossy and domineering\\My sister used to peremptorily tell me to do the dishes, a chore I would either do perfunctorily or avoid\\doing altogether.\\\\                                                                                \\}\\
{perfunctory}\\
{ভাসা - ভাসা,,}\\
{(adjective): done routinely and with little interest or care\\The short film examines modern perfunctory cleaning rituals such as washing dishes, doing laundry and\\tooth-brushing.\\}\\
{peripatetic}\\
{ভবঘুরে,,}\\
{(adjective): traveling by foot\\Jim always preferred a peripatetic approach to discovering a city: he felt that he could see so many more\\details while walking.\\}\\
{perspicacious}\\
{তীক্ষ্নদৃষ্টি,,}\\
{(adjective): acutely insightful and wise\\Many modern observers regard Eisenhower as perspicacious, particularly in his accurate prediction of\\the growth of the military.\\}\\
{phantasmagorical}\\
{}\\
{(adjective): illusive; unreal\\Those suffering from malaria fall into a feverish sleep, their world a whirligig of phantasmagoria; if they\\recover, they are unsure of what actually took place and what was simply a product of their febrile\\imaginations.\\}\\
{philistine}\\
{সঙ্কীর্ণমনা ব্যক্তি,,}\\
{(adjective): smug and ignorant towards artistic and cultural values\\Jane considered Al completely philistine, because he talked almost exclusive about video games; she was\\entirely unaware of how well read he really was.\\This word has other definitions but this is the most important one for the GRE\\}\\
{phlegmatic}\\
{জড়,,}\\
{(adjective): showing little emotion\\Arnold is truly noble, remaining reserved until an issue of significance arises, but Walter is simply\\phlegmatic: he doesn't have the energy or inclination to care about anything.\\}\\
{picayune}\\
{}\\
{(adjective): trifling or petty (a person)\\English teachers are notorious for being picayune; however, the English language is so nuanced and\\sophisticated that often such teachers are not being contrary but are only adhering to the rules.\\}\\
{pillory}\\
{কাষ্ঠানির্মিত শাস্তিস্তম্ভবিশেষ,,}\\
{(verb): ridicule or expose to public scorn\\After the candidate confessed, the press of the opposing party took the opportunity to pillory him,\\printing editorials with the most blatantly exaggerated accusations.\\This word has other definitions but this is the most important one for the GRE\\}\\
{pith}\\
{মজ্জা,,}\\
{(noun): the most essential part of something\\When Cynthia hears a speaker presenting a complex argument, she is always able to discard the\\irrelevant details and extract the pith of what the speaker is trying to convey.\\This word has other definitions but this is the most important one for the GRE\\}\\
{plucky}\\
{সাহসী,,}\\
{(adjective): marked by courage and determination\\Some scouts initially doubted Pedroia because of his short stature, but he is a plucky player, surprising\\everyone with his boundless energy and fierce determination.\\}\\
{Pollyannaish}\\
{}\\
{(adjective): extremely optimistic\\\\                                                                                \\Even in the midst of a lousy sales quarter, Debbie remained Pollyannaish, never losing her shrill voice and\\wide smile, even when prospective customers hung up on her.\\}\\
{ponderous}\\
{ভারী,,}\\
{(adjective): weighed-down; moving slowly\\Laden with 20 kilograms of college text books, the freshman moved ponderously across the campus.\\}\\
{pontificate}\\
{উপেক্ষা করা}\\
{(verb): talk in a dogmatic and pompous manner\\The vice-president would often pontificate about economic theory, as if no one else in the room were\\qualified to speak on the topic.\\This word has other definitions but this is the most important one for the GRE\\}\\
{portentous}\\
{পরম,,}\\
{(adjective): ominously prophetic\\When the captain and more than half the officers were sick on the very first night of the voyage, many of\\the passengers felt this was portentous, but the rest of the voyage continued without any problems.\\}\\
{precipitate}\\
{থিতান,,}\\
{(adjective): hasty or rash\\Instead of conducting a thorough investigation after the city hall break-in, the governor acted\\precipitately, accusing his staff of aiding and abetting the criminals.\\}\\
{precipitate}\\
{থিতান,,}\\
{(verb): to cause to happen\\The government's mishandling the hurricane's aftermath precipitated a widespread outbreak of looting\\and other criminal activity.\\}\\
{presentiment}\\
{পূর্ব লক্ষণ,,}\\
{(noun): a feeling of evil to come\\On the night that Lincoln would be fatally shot, his wife had a presentiment about going to Ford's\\Theater, but Lincoln persuaded her that everything would be fine.\\}\\
{primacy}\\
{প্রথম স্থান,,}\\
{(noun): the state of being first in importance\\The primacy of Apple Computers is not guaranteed, as seen in the recent lawsuits and weak growth.\\}\\
{probity}\\
{সততা,,}\\
{(noun): integrity, strong moral principles\\The ideal politician would have the probity to lead, but reality gravely falls short of the ideal of morally\\upright leaders.\\}\\
{prognostication}\\
{পূর্বলক্ষণ,,}\\
{(noun): a statement made about the future\\When the Senator was asked about where the negotiations would lead, he said that any guess he could\\make would be an unreliable prognostication.\\}\\
{prolixity}\\
{বাগ্বাহুল্য,,}\\
{(noun): boring verbosity\\I loved my grandfather dearly, but his prolixity would put me to sleep, regardless of the topic.\\}\\
{promulgate}\\
{প্রচার করা,,}\\
{(verb): state or announce\\The President wanted to promulgate the success of the treaty negotiations, but he had to wait until\\Congress formally approved the agreement.\\\\                                                                                  \\}\\
{propitiate}\\
{প্রসন্ন করান,,}\\
{(verb): to placate or appease\\The two sons, plying their angry father with cheesy neckties for Christmas, were hardly able to propitiate\\him -- the father already had a drawer full of ones he had never worn before or ever planned to.\\}\\
{prosaic}\\
{গতানুগতিক,,}\\
{(adjective): dull and lacking imagination\\Unlike the talented artists in his workshop, Paul had no such bent for the visual medium, so when it was\\time for him to make a stained glass painting, he ended up with a prosaic mosaic.\\}\\
{proscribe}\\
{নির্বাসিত করা,,}\\
{(verb): command against\\My doctor proscribes that I not eat donuts with chocolate sauce and hamburger patties for breakfast.\\}\\
{proselytize}\\
{ধর্মান্তরিত করা,,}\\
{(verb): convert to another religion, philosophy, or perspective\\Lisa loves her Mac but says little about it; by contrast, Jake will proselytize, interrogating anyone with an\\Android about why she didn't purchase an iPhone.\\}\\
{protean}\\
{আকৃতি পরিবর্তনের তত্পর,,}\\
{(adjective): readily taking on different roles; versatile\\Peter Sellers was truly a protean actor--in Doctor Strangelove he played three very different roles: a\\jingoist general, a sedate President and a deranged scientist.\\}\\
{provident}\\
{মিতব্যয়ী,,}\\
{(adjective): careful in regard to your own interests; providing carefully for the future\\In a move that hardly could be described as provident, Bert spend his entire savings on a luxurious cruise,\\knowing that other bills would come due a couple months later.\\}\\
{puerile}\\
{বুড়া,,}\\
{(adjective): of or characteristic of a child; displaying or suggesting a lack of maturity\\Helen enjoyed blowing soap bubbles, but Jim regarded this as puerile, totally unworthy of a woman with\\a Ph.D.\\}\\
{puissant}\\
{প্রভাবশালী,,}\\
{(adjective): powerful\\Over the years of service, and quite to his surprise, he became a puissant advisor to the community.\\}\\
{punctilious}\\
{শিষ্টচারসম্পন্ন,,}\\
{(adjective): marked by precise accordance with details\\The colonel was so punctilious about enforcing regulations that men fell compelled to polish even the\\soles of their shoes.\\}\\
{pyrrhic}\\
{প্রাচীন গ্রীক নৃত্যবিশেষ,,}\\
{(adjective): describing a victory that comes at such a great cost that the victory is not worthwhile\\George W. Bush's win in the 2000 election was in many ways a pyrrhic victory: the circumstances of his\\win alienated half of the U.S. population.\\}\\
{quail}\\
{ভয়ে পিছাইয়া পড়া,,}\\
{(verb): draw back, as with fear or pain\\Craig always claimed to be a fearless outdoorsman, but when the thunderstorm engulfed the valley, he\\quailed at the thought of leaving the safety of his cabin.\\This word has other definitions but this is the most important one for the GRE\\\\                                                                                 \\}\\
{quisling}\\
{বিভীষণ,,}\\
{(noun): a traitor\\History looks unfavorably upon quislings; indeed they are accorded about the same fondness as Nero--he\\who watched his city burn down while playing the violin.\\}\\
{quixotic}\\
{ভাববিলাসী,,}\\
{(adjective): wildly idealistic; impractical\\For every thousand startups with quixotic plans to be the next big name in e-commerce, only a handful\\ever become profitable.\\}\\
{raconteur}\\
{গল্প - কথক,,}\\
{(noun): a person skilled in telling anecdotes\\Jude is entertaining, but he is no raconteur: beyond the handful of amusing stories he has memorized, he\\has absolutely no spontaneous story-telling ability.\\}\\
{raillery}\\
{পরিহাস,,}\\
{(noun): light teasing\\The new recruit was not bothered by the raillery, finding most of it light-hearted and good-natured.\\}\\
{rapprochement}\\
{পুনর্মিলন,,}\\
{(noun): the reestablishing of cordial relations\\Although Ann hoped that her mother and her aunt would have a rapprochement, each one's bitter\\accusations against the other made any reconciliation unlikely.\\}\\
{rarefied}\\
{তনু,,}\\
{(verb): make more subtle or refined\\Jack's vulgar jokes were not so successful in the rarefied enviroment of college professors.\\This word has other definitions but this is the most important one for the GRE\\}\\
{recapitulation}\\
{অনুচিন্তা,,}\\
{(noun): a summary (think of recap)\\Every point of the professors lesson was so clear that the students felt his concluding recapitulation was\\not necessary.\\}\\
{recrimination}\\
{প্রত্যপবাদ,,}\\
{(noun): mutual accusations\\The two brothers sat and cried, pointing fingers and making elaborate recriminations of the other's guilt\\}\\
{recrudesce}\\
{পুনরায় প্রকাশ দেত্তয়া,,}\\
{(verb): to break out or happen again\\After years of gamblers anonymous, Tony thought he'd broken his compulsive slot machine playing, but\\it took only one trip to the Atlantic City for a full recrudescence--he lost \$5k on the one armed bandit.\\}\\
{redoubtable}\\
{দুর্ধর্ষ,,}\\
{(adjective): inspiring fear or awe\\On television basketball players don't look that tall, but when you stand in front of a seven-foot tall NBA\\player, he is truly redoubtable.\\}\\
{remonstrate}\\
{তীব্র আপত্তি করা,,}\\
{(verb): to make objections while pleading\\The mothers of the kidnapped victims remonstrated to the rogue government to release their children,\\claiming that the detention violated human rights.\\}\\
{reprisal}\\
{প্রতিশোধ গ্রহণ,,}\\
{(noun): a retaliatory action against an enemy in wartime\\The Old Testament doctrine of an eye for an eye is not the kind of retaliation practiced in war; rather, an\\\\                                                                               \\arm, a leg, and both ears are the reprisal for the smallest scratch.\\}\\
{ribald}\\
{অশ্লীল,,}\\
{(adjective): humorously vulgar\\The speaker was famous for his ribald humor, but the high school principal asked him to keep the talk G-\\rated when he spoke to the student body.\\}\\
{row}\\
{সারি}\\
{(noun): an angry dispute\\The Prime Minister looked very foolish after his row with the foreign dignitary was caught on video and\\posted on youtube.\\This word has other definitions but this is the most important one for the GRE\\}\\
{sagacious}\\
{জ্ঞানী,,}\\
{(adjective): having good judgement and acute insight\\Steve Jobs is surely one of the most sagacious CEOs, making Apple the most recognizable and valuable\\companies in the world.\\}\\
{sangfroid}\\
{কামাবশায়িতা,,}\\
{(noun): calmness or poise in difficult situations\\The hostage negotiator exhibited a sangfroid that oftentimes was more menacing than the sword at his\\throat, or the gun at his head.\\}\\
{sardonic}\\
{তিক্ত,,}\\
{(adjective): disdainfully or ironically humorous; scornful and mocking\\A stand-up comedian walks a fine line when making jokes about members of the audience; such fun and\\joking can quickly become sardonic and cutting.\\}\\
{sartorial}\\
{দর্জিসংক্রান্ত,,}\\
{(adjective): related to fashion or clothes\\Monte was astute at navigating the world of finance; sartorially, however, he was found wanting--he\\typically would attempt to complement his beige tie with a gray suit and white pants.\\}\\
{saturnine}\\
{বিষণ্ণ মেজাজসম্পন্ন,,}\\
{(adjective): morose or gloomy\\Deprived of sunlight, humans become saturnine; that's why in very northerly territories people are\\encouraged to sit under an extremely powerful lamp, lest they become morose.\\}\\
{schadenfreude}\\
{পরের দুর্দশায় আনন্দ,,}\\
{(noun): joy from watching the suffering of others\\From his warm apartment window, Stanley reveled in schadenfreude as he laughed at the figures below,\\huddled together in the arctic chill.\\}\\
{sedulous}\\
{পরিশ্রমী,,}\\
{(adjective): done diligently and carefully\\An avid numismatist, Harold sedulously amassed a collection of coins from over 100 countries--an\\endeavor that took over fifteen years, and to five continents.\\}\\
{self-effacing}\\
{}\\
{(adjective): reluctant to draw attention to yourself\\The most admirable teachers and respected leaders are those who are self-effacing, directing attention\\and praise to their students and workers.\\}\\
{semblance}\\
{আভাস,,}\\
{(noun): an outward or token appearance or form that is deliberately misleading\\\\                                                                                 \\While the banker maintained a semblance of respectability in public, those who knew him well were\\familiar with his many crimes.\\}\\
{sententious}\\
{নীতিগর্ভ,,}\\
{(adjective): to be moralizing, usually in a pompous sense\\The old man, casting his nose up in the air at the group of adolescents, intoned sententiously, "Youth is\\wasted on the young."\\}\\
{simulacrum}\\
{ধ্বজা,,}\\
{(noun): a representation of a person (especially in the form of sculpture)\\The Shanghai Urban Planning Exhibition Center showcases a simulacrum of all the present and approved\\buildings in the city of Shanghai.\\}\\
{simulacrum}\\
{ধ্বজা,,}\\
{(noun): a bad imitation\\The early days of computer graphics made real people into a simalacrum that now seems comical.\\}\\
{sinecure}\\
{কর্মভারহীন পদ,,}\\
{(noun): an office that involves minimal duties\\The position of Research Director is a sinecure: the job entails almost no responsibilities, nor does the\\person in that position have to answer to anyone.\\}\\
{solecism}\\
{বাক্যগঠনপ্রণালীর নিয়মভঙ্গ,,}\\
{(noun): a socially awkward or tactless act\\Mother Anna was always on guard against any solecism from her children and scolded them\\immediately if any of them talked out of place in public.\\This word has other definitions but this is the most important one for the GRE\\}\\
{solicitous}\\
{ব্যগ্র,,}\\
{(adjective): showing hovering attentiveness\\Our neighbors are constantly knocking on our door to make sure we are ok, and I don't know how to ask\\them to stop being so solicitous about our health.\\This word has other definitions but this is the most important one for the GRE\\}\\
{solicitude}\\
{উত্কণ্ঠা,,}\\
{(noun): a feeling of excessive concern\\I walked to his house in the rain to make sure he had enough to eat while he was sick, but he seemed not\\to appreciate my solicitude.\\}\\
{spartan}\\
{স্পার্টা লোক,,}\\
{(adjective): unsparing and uncompromising in discipline or judgment; practicing great self-\\denial\\After losing everything in a fire, Tim decided to live in spartan conditions, sleeping on the floor and\\owning as little furniture as a possible.\\}\\
{splenetic}\\
{বিমর্ষ,,}\\
{(adjective): very irritable\\Ever since the car accident, Frank has been unable to walk without a cane, and so he has become\\splenetic and unpleasant to be around.\\}\\
{squelch}\\
{দমন করা,,}\\
{(verb): suppress or crush completely\\After the dictator consolidated his power, he took steps to squelch all criticism, often arresting any\\journalist who said anything that could be interpreted as negative about his regime.\\\\                                                                                \\}\\
{stalwart}\\
{সাহসী,,}\\
{(adjective): dependable; inured to fatigue or hardships\\Despite all the criticism directed at the President during this scandal, Lisa has remained his stalwart\\supporter.\\}\\
{stultify}\\
{বোকা বানান,,}\\
{(verb): cause one, through routine, to lose energy and enthusiasm\\As an undergraduate Mark felt stultified by classes outside his area of study; only in grad school, in which\\he could focus solely on literary analysis, did he regain his scholarly edge.\\}\\
{subterfuge}\\
{এড়ানর কৌশল,,}\\
{(noun): something intended to misrepresent the true nature of an activity\\Finally deciding to abandon all subterfuge, Arthur revealed to Cindy everything about his secret affair\\over the past two years.\\}\\
{supercilious}\\
{উন্নাসিক,,}\\
{(adjective): haughty and disdainful; looking down on others\\Nelly felt the Quiz Bowl director acted superciliously towards the underclassmen; really, she fumed, must\\he act so preternaturally omniscient each time he intones some obscure fact--as though everybody\\knows that Mt. Aconcagua is the highest peak in Sout\\}\\
{surfeit}\\
{আতিশয্য,,}\\
{(noun): an excessive amount of something\\There was no such thing as a surfeit of shopping for Nancy--she could stay at the outlet stores from\\opening to closing time.\\}\\
{surreptitious}\\
{গুপ্ত,,}\\
{(adjective): stealty, taking pains not to be caught or detected\\Since his mom was a light sleeper, Timmy had to tiptoe surreptitiously through the entire house, careful\\to not make the floors creak, until he at last was able to enjoy his plunder: a box of chocolate chip\\cookies.\\}\\
{sybarite}\\
{ভোগবিলাসে মগ্ন,,}\\
{(noun): a person who indulges in luxury\\Despite the fact that he'd maxed out fifteen credit cards, Max was still a sybarite at heart: when the\\police found him, he was at a \$1,000 an hour spa in Manhattan, getting a facial treatment.\\}\\
{temerity}\\
{হঠকারিতা,,}\\
{(noun): fearless daring\\No child has the temerity to go in the rundown house at the end of the street and see if it is haunted.\\}\\
{tempestuous}\\
{প্রচণ্ড,,}\\
{(adjective): as if driven by turbulent or conflicting emotions; highly energetic and wildly\\changing or fluctuating\\Chuck and Kathy had always been stable and agreeable people on their own, but when they got involved,\\it was a tempestuous relationship.\\}\\
{tendentious}\\
{উদ্দেশ্যমূলক,,}\\
{(adjective): likely to lean towards a controversial view\\Because political mudslinging has become a staple of the 24-hour media cycle, most of us, despite\\protestations to the contrary, are tendentious on many of today's pressing issues.\\}\\
{transmute}\\
{রূপান্তরিত হওয়া}\\
{(verb): change or alter in form, appearance, or nature\\\\                                                                                \\One of the goals of alchemy was to find the substance or process that would transmute lead into gold.\\}\\
{trenchant}\\
{মর্মভেদী,,}\\
{(adjective): characterized by or full of force and vigor; having keenness and forcefulness and\\penetration in thought, expression, or intellect\\Jill presented a rather superficial treatment of sales in Asia, but her trenchant analysis of sales in Europe\\inspired a number of insights into how to proceed in that market.\\}\\
{truculence}\\
{নিষ্ঠুরতা,,}\\
{(noun): defiant aggressiveness\\When the boss confronted Aaron about his earlier remarks, Aaron responded with utter truculence,\\simply throwing a glass of water in the boss' face and walking away.\\}\\
{truculent}\\
{নিষ্ঠুর,,}\\
{(adjective): having a fierce, savage nature\\Standing in line for six hours, she became progressively truculent, yelling at DMV employees and\\elbowing other people waiting in line.\\}\\
{turgid}\\
{স্ফীত,,}\\
{(adjective): (of language) pompous and tedious\\The amount of GRE vocabulary he used increased with his years--by the time he was 60, his novels were\\so turgid that even his diehard fans refused to read them.\\}\\
{turpitude}\\
{অসচ্চরিত্রতা,,}\\
{(noun): depravity; a depraved act\\During his reign, Caligula indulged in unspeakable sexual practices, so it not surprising that he will\\forever be remembered for his turpitude.\\}\\
{tyro}\\
{শিক্ষানবিস,,}\\
{(noun): someone new to a field or activity\\All great writers, athletes, and artists were tyros at one time--unknown, clumsy, and unskilled with much\\to learn.\\}\\
{umbrage}\\
{ছায়া,,}\\
{(noun): a feeling of anger caused by being offended\\Since he was so in love with her, he took umbrage at her comments, even though she had only meant to\\gently tease him.\\}\\
{unassailable}\\
{অনাক্রম্য,,}\\
{(adjective): immune to attack; without flaws\\Professor Williams is so self-assured as to seem arrogant, presenting each and every opinion as an\\unassailable fact.\\}\\
{unflappable}\\
{উদ্বেগহীন,,}\\
{(adjective): not easily perturbed or excited or upset; marked by extreme calm and\\composure\\The house shook and the ground quaked, but my dad was unflappable and comforted the family.\\}\\
{unforthcoming}\\
{}\\
{(adjective): uncooperative, not willing to give up information\\The teacher demanded to know who broke the window while he was out of the room, but the students\\understandably were unforthcoming.\\\\                                                                               \\}\\
{unimpeachable}\\
{অনিন্দ্য,,}\\
{(adjective): free of guilt; not subject to blame; beyond doubt or reproach\\After his long and unimpeachable service to the company, Sharat felt that a gold watch was a slap in the\\face rather than an honor.\\}\\
{unprepossessing}\\
{অনাকর্ষণীয়,,}\\
{(adjective): creating an unfavorable or neutral first impression\\World leaders coming to meet Gandhi would expect a towering sage, and often would be surprised by\\the unprepossessing little man dressed only in a loincloth and shawl.\\}\\
{unpropitious}\\
{অকল্যাণকর,,}\\
{(adjective): (of a circumstance) with little chance of success\\With only a bottle of water and a sandwich, the hikers faced an unpropitious task: ascending a huge\\mountain that took most two days to climb.\\}\\
{unstinting}\\
{}\\
{(adjective): very generous\\Helen is unstinting with her time, often spending hours at the house of a sick friend.\\}\\
{untenable}\\
{অসমর্থনীয়,,}\\
{(adjective): (of theories etc) incapable of being defended or justified\\With the combination of Kepler's brilliant theories and Galileo's telescopic observations, the old\\geocentric theory became untenable to most of the educated people in Europe.\\}\\
{untoward}\\
{অবাধ্য,,}\\
{(adjective): unfavorable; inconvenient\\Some professors find teaching untoward as having to prepare for lectures and conduct office hours\\prevents them from focusing on their research.\\}\\
{untrammeled}\\
{অক্ষুণ্ণ,,}\\
{(adjective): not confined or limited\\The whole notion of living untrammeled inspired the American Revolution and was enshrined in the\\Declaration of Independence and the Constitution.\\}\\
{unviable}\\
{বযারতলন্যযাগ্,,}\\
{(adjective): not able to work, survive, or succeed (also spelled inviable).\\The plan was obviously unviable considering that it lead to complete environmental destruction in the\\river valley.\\}\\
{vaunted}\\
{জাঁক করা,,}\\
{(adjective): highly or widely praised or boasted about\\For years, they had heard of New York City's vaunted skyline, and when they finally saw it, the\\spectacular cityscape did not disappoint them in the least.\\}\\
{venial}\\
{মার্জনীয়,,}\\
{(adjective): easily excused or forgiven; pardonable\\His traffic violations ran the gamut from the venial to the egregious--on one occasion he simply did not\\come to a complete stop; another time he tried to escape across state lines at speeds in excess of 140\\mph.\\}\\
{verisimilitude}\\
{আপাত সত্য,,}\\
{(noun): the appearance of truth\\All bad novels are bad for numerous reasons; all good novels are good for their verisimilitude of reality,\\placing the readers in a world that resembles the one they know.\\\\                                                                                  \\}\\
{veritable}\\
{যথার্থ,,}\\
{(adjective): truthfully, without a doubt\\Frank is a veritable life-saver -- last year, on two different occasions, he revived people using CPR.\\}\\
{vicissitude}\\
{ভাগ্যপরিবর্তন,,}\\
{(noun): change in one"s circumstances, usually for the worse\\Even great rulers have their vicissitudes--massive kingdoms have diminished overnight, and once beloved\\kings have faced the scorn of angry masses.\\}\\
{vitriol}\\
{ভিট্রিয়ল}\\
{(noun): abusive or venomous language used to express blame or bitter deep-seated ill will\\His vitriol spewed forth from a deep-seated racisim that consumed his whole life.\\}\\
{vitriolic}\\
{গন্ধকজ,,}\\
{(adjective): harsh or corrosive in tone\\While the teacher was more moderate in her criticism of the other student's papers, she was vitriolic\\toward Peter's paper, casting every flaw in the harshest light.\\}\\
{vituperate}\\
{গালাগালি দেওয়া}\\
{(adjective): to criticize harshly; to berate\\Jason had dealt with disciplinarians before, but nothing prepared him for the first week of boot camp, as\\drill sergeants vituperated him for petty oversights such as forgetting to double knot the laces on his\\boots.\\}\\
{zeitgeist}\\
{যুগের ভাবধারা,,}\\
{(noun): spirit of the times\\Each decade has its own zeitgeist--the 1990's was a prosperous time in which the promise of the\\American Dream never seemed more palpable.\\\\}\\

\end{document}
